
\section{Integrals of Parametrized Curves}

\begin{exercise}{}{}
  Calculate
  \begin{equation}
    \int_C |z| \mathrm{d} z,
  \end{equation}
  where $C$ is the line from the origin to $1 + i$.
\end{exercise}

\begin{solution}
  $\frac{\sqrt{2}}{2}(1+i)$.
\end{solution}


\section{Cauchy's Theorem}

\begin{exercise}{}{}
  Prove that
  \begin{equation}
    \int_0^{\pi} \frac{1 + 2 \cos \theta}{5 + 4 \cos \theta} \mathrm{d} \theta = 0.
  \end{equation}
  Use the value of $\int_C \frac{\mathrm{d} z}{z + 2}$, where $C$ is the unit circle $|z| = 1$.
\end{exercise}

\begin{proof}
  The pole of $\frac{1}{z+2}$ is $z = -2$, it is not contained in the unit disk.
  Then by Cauchy's theorem
  \begin{equation}
    \int_C \frac{\mathrm{d} z}{z + 2} = 0.
  \end{equation}
  Since $\cos \theta = \frac{z + z^{-1}}{2}$ and $\theta = \frac{1}{i}\ln z$,
  change variables of integral, we get
  \begin{equation}
    \int_0^{\pi} \frac{1 + 2 \cos \theta}{5 + 4 \cos \theta}\mathrm{d} \theta
    = \int_{C^{\prime}} \frac{1 + z + z^{-1}}{5 + 2z + 2z^{-1}} \frac{\mathrm{d} z}{i z},
  \end{equation}
  where $C^{\prime}$ is the upper half of the circle.
\end{proof}

\begin{exercise}{}{}
  Prove that $e^{-\pi x^2}$ is its own Fourier transform, i.e., if $\xi \in \mathbb{R}$,
  \begin{equation}
    e^{- \pi \xi^2} = \int_{-\infty}^{+\infty} e^{-\pi x^2} e^{-2\pi i x \xi} \mathrm{d} x.
  \end{equation}
\end{exercise}

\begin{proof}
  If $\xi = 0$, then the formula is precisely the known integral $1 = \int_{-\infty}^{+\infty} e^{-\pi x^2}\mathrm{d} x$
  \footnote{Consider $I^2 = \int_{-\infty}^{+\infty}\int_{-\infty}^{+\infty} e^{-\pi (x^2 + y^2)}\mathrm{d} x \mathrm{d}y$
    and let $x = r \cos \theta$, $y = r\sin \theta$.}.
  Suppose $\xi > 0$, and consider $f(z) = e^{-\pi z^2}$, which is entire.
  Consider the contour $\gamma_R$ consists of a rectangle with vertices
  $R, R+i\xi, -R+i\xi, -R$ with counterclockwise orintation.
  By Cauchy's theorem, 
  \begin{equation}
    \int_{\gamma_R} f(z)\mathrm{d} z = 0
  \end{equation}
  The integral over the real segment is $\int_{-R}^R e^{-\pi x^2} \mathrm{d} x$,
  which converges to $1$ as $R \rightarrow \infty$.
  The integral on the vertical side on the right is
  \begin{equation}
    I(R) = \int^{\xi}_0 f(R + iy)i \mathrm{d} y = \int_0^{\xi} e^{-\pi (R^2 + 2i Ry - y^2)}i \mathrm{d} y,
  \end{equation}
  which goes to $0$ as $R \rightarrow \infty$ since $|I(R)| \leq C e^{-\pi R^2}$.
  Similarly, the integral over vertical segment on the left also goes to $0$ as $R \rightarrow \infty$.
  Finally, the integral over the horizontal segment on top is
  \begin{equation}
    \int_R^{-R} e^{-\pi (x+i\xi)^2}\mathrm{d} x = - e^{\pi \xi^2} \int_{-R}^R e^{-\pi x^2}e^{-2\pi i x \xi}\mathrm{d} x.
  \end{equation}
  Therefore, the conclusion follows from the limit as $R \rightarrow \infty$.
  In the case $\xi < 0$, just consider the symmetric rectangle in the lower half-plane.
\end{proof}

\begin{exercise}{}{}
  Prove that 
  \begin{equation}
    \int_0^{\infty} \frac{1 - \cos x}{x^2}\mathrm{d} x = \frac{\pi}{2}.
  \end{equation}
\end{exercise}

\section{Cauchy's Integral Formula}

\begin{exercise}{}{}
  Calculate the following integral
  \begin{equation}
    I = \int_{|z| = 1} \frac{\left( z + \frac{1}{z} \right)^{2n}}{z} \mathrm{d} z,
  \end{equation}
  and prove that
  \begin{equation}
    J = \int_0^{2\pi} (\cos \theta)^{2n} \mathrm{d} \theta = 2 \pi \cdot \frac{(2n-1)!!}{(2n)!!}.
  \end{equation}
\end{exercise}

\begin{solution}
  Consider $f(z) = (1 + z^2)^{2n}$, by Cauchy's integral formula
  \begin{equation}
    f^{(2n)}(0) = \frac{(2n)!}{2 \pi i} \int_{|z| = 1} \frac{f(z)}{(z - 0)^{2n+1}}\mathrm{d} z
    = \frac{(2n)!}{2 \pi i} \int_{|z| = 1} \frac{(z + \frac{1}{z})^{2n}}{z}\mathrm{d} z.
  \end{equation}
  Since $(1 + x)^n = \sum_{k = 0}^{\infty} \binom{n}{k} x^k = \sum_{k = 0}^{\infty} \frac{n!}{k!(n-k)!}x^k$,
  then
  \begin{equation}
    f(z) = \sum\limits_{k = 0}^{\infty} \frac{(2n)!}{k!(2n-k)!}z^{2k}
    \Rightarrow
    f^{(2n)}(0) = \frac{(2n)! \cdot (2n)!}{n! \cdot n!}.
  \end{equation}
  Then we get $I = 2\pi i \frac{(n+1)(n+2)\cdots (2n)}{n!}$.
  On the other hand, let $z = e^{i\theta}$, then
  \begin{equation}
    I = \int_0^{2\pi} \frac{4^n (\cos \theta)^{2n}}{e^{i \theta}} \cdot i e^{i\theta} \mathrm{d} \theta
    = 4^n i \int_0^{2\pi} (\cos \theta)^{2n} \mathrm{d} \theta.
  \end{equation}
  Then $J = 2 \pi \frac{(2n)!}{(2^n n!) \cdot (2^n n!)} = 2 \pi \cdot \frac{(2n-1)!!}{(2n)!!}$.
\end{solution}

\section{Inequalities}

\begin{exercise}{}{}
  Let $\Gamma$ be the line from $-i$ to $i$, prove that
  \begin{equation}
    \left| \int_{\Gamma} (x^2 + iy^2)\mathrm{d} z \right| \leq 2.
  \end{equation}
\end{exercise}

\begin{proof}
  The curve can be represented as $z = 0 + it$, then
  $|x^2 + iy^2| = |it^2| \leq 1$.
  \begin{equation}
    \left| \int_{\Gamma} (x^2 + iy^2)\mathrm{d} z \right|
    \leq \ell(\Gamma) \cdot 1 = 2.
  \end{equation}
\end{proof}

\begin{exercise}{}{}
  Let $C$ denote the upper circle from $z = R$ to $z = -R$, prove that
  \begin{equation}
    \left| \int_C e^{iz}\mathrm{d} z \right| < \pi.
  \end{equation}
\end{exercise}

\begin{proof}
  $z = R e^{i\theta} = R \cos \theta + i R \sin \theta$, then $|e^{iz}| = |e^{iR \cos \theta - R \sin \theta}| = e^{-R \sin \theta}$.
  On the other hand, $\mathrm{d}z = Rie^{i\theta}\mathrm{d} \theta$, then $|\mathrm{d} z| = R \mathrm{d} \theta$.
  Then
  \begin{align}
    \left| \int_C e^{iz}\mathrm{d} z \right| &\leq \int_C |e^{iz}| \cdot |\mathrm{d} z|
    = \int_0^{\pi} e^{-R \sin \theta}R \mathrm{d} \theta \\
    &\leq 2 \int_0^{\frac{\pi}{2}} e^{- \frac{2}{\pi} R \theta}R \mathrm{d}\theta = \pi(1 - e^{-R}) < \pi.
  \end{align}
\end{proof}




