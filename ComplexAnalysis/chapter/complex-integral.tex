
\section{Complex Integrals}

\subsection{Integral of Parametrized Curve}

\begin{definition}{Integral of Parametrized Curve}{}
  Given a smooth curve $\gamma$ in $\mathbb{C}$ parametrized by $z(t):[a,b] \rightarrow \mathbb{C}$,
  and $f$ a continuous function on $\gamma$.
  The \emph{integral of $f$ along $\gamma$} is
  \begin{equation}
    \int_{\gamma} f(z)\mathrm{d} z = \int_a^b f(z(t)) z^{\prime}(t) \mathrm{d} t.
  \end{equation}
\end{definition}

\begin{example}{Calculation of Integrals of Parametrized Curve}{}
  \begin{enumerate}
  \item Let $C$ be the line connecting $z_0 = 0$ and $z_1 = 1 + i$, calculate
    \begin{equation}
      \int_C x - y + ix^2 \mathrm{d} z.
    \end{equation}
  \item Let $C$ be the upper semicircle with radius $r = 1$, calculate
    \begin{equation}
      \int_C |z| \mathrm{d} z.
    \end{equation}
  \end{enumerate}
\end{example}

\begin{solution}
  (1) $C$ can be parametrized as $z(t) = t + it$, $t \in [0, 1]$.
  So substituting $x = t$ and $y = t$
  \begin{equation}
    \int_C x-y+ix^2 \mathrm{d} z = \int_0^1 (t - t + t^2 i) (1 + i)\mathrm{d} t
    = - \frac{1}{3} + \frac{1}{3} i.
  \end{equation}

  (2) $C$ can be parametrized as $z(\theta) = e^{i\theta}$. Then
  \begin{equation}
    \int_C |z|\mathrm{d} z = \int_0^{\pi} 1 \cdot i e^{i\theta}\mathrm{d} \theta
    = e^{i\theta} \big|^{\pi}_0 = -1 - 1 = -2.
  \end{equation}
\end{solution}

\subsection{Newton-Leibniz Formula}

\begin{theorem}{Newton-Leibniz Formula}{}
  If a continuous function $f$ has a primitive $F$ in $\Omega$,
  and $\gamma$ is a curve in $\Omega$ that begins at $w_1$ and
  ends at $w_2$. Then
  \begin{equation}
    \int_{\gamma}f(z)\mathrm{d} z = F(w_2) - F(w_1).
  \end{equation}
\end{theorem}

\begin{example}{Applications of Newton-Leibniz Formula}{}
  Calculate the following integrals:
  \begin{enumerate}
  \item $\int_{-2}^{-2 + i} (z+2)^2 \mathrm{d} z$.
  \item $\int_0^{2\pi a} 2z^2 + 8z + 1 \mathrm{d} z$, where $x = a(\theta - \sin \theta)$,
    and $y = a (1 - \cos \theta)$.
  \end{enumerate}
\end{example}

\begin{solution}
  (1) Apply the Newton-Leibniz formula, we have
  \begin{equation}
    \int_{-2}^{-2+i} (z+2)^2 \mathrm{d} z = \frac{1}{3}(z+2)^3 \big|^{i-2}_{-2}
    = - \frac{1}{3}i.
  \end{equation}

  (2) The integral is indepedent from the curve, so apply the Newton-Leibniz formula,
  and the final answer is $\frac{16}{3}\pi^3a^3 + 16 \pi^2 a^2 + 2\pi a$.
\end{solution}

\begin{corollary}{Integral along Closed Curves}{}
  If $\gamma$ is a closed curve in an open set $\Omega$,
  and $f$ is continuous and has a primitive in $\Omega$, then
  \begin{equation}
    \int_{\gamma} f(z) \mathrm{d} z = 0.
  \end{equation}
\end{corollary}

\section{Cauchy's Theorem and its Generalizations}

\subsection{Cauchy's Theorem}

\begin{theorem}{Cauchy's Theorem}{}
  If $f$ is holomorphic in the closure of a simply connected region $D \subset \mathbb{C}$,
  and $\gamma$ is a closed curve in $D$, then
  \begin{equation}
    \int_{\gamma} f(z)\mathrm{d} z = 0.
  \end{equation}
  To be more specific, $f(z)$ has a primitive in $D$.
\end{theorem}

\begin{example}{}{}
  Prove that the value of following integrals are zero
  \begin{equation}
    \int_C \frac{\mathrm{d} z}{\cos z}, \quad
    \int_C \frac{\mathrm{d} z}{z^2 + 2z + 2}.
  \end{equation}
  where $C$ is the unit circle $|z| = 1$
\end{example}

\begin{proof}
  (1) The roots of $\cos z$ is $z = \frac{\pi}{2} + n\pi$ where $n \in \mathbb{Z}$.
  All of the roots are not contained in $|z| \leq 1$, thus by Cauchy's theorem the integral is zero.

  (2) The roots of $z^2 + 2z + 1$ are $z = -2 \pm i$, the absolute values of both roots are greater than $1$.
\end{proof}


\subsection{Cauchy's Integral Formula}

\begin{theorem}{Cauchy's Integral Formula}{}
  Suppose $f$ is holomorphic in the closure of an open set $D \subset \mathbb{C}$.
  Denote the boundary of $D$ with the positive orintation as $C$, then
  \begin{equation}
    f(z) = \frac{1}{2 \pi i} \int_C \frac{f(\zeta)}{\zeta - z}\mathrm{d} \zeta, \quad
    z \in D.
  \end{equation}
\end{theorem}

\begin{example}{Applications of Cauchy's Integral Formula}{}
  Calculate
  \begin{equation}
    \int_C \frac{z^2 - z + 1}{z - 1}\mathrm{d} z,
  \end{equation}
  where $C$ is the circle centered at the origin with a radius of $2$.
\end{example}

\begin{solution}
  Denote $f(z) = z^2 - z + 1$, $f(1) = 1$.
  Since $z = 1$ is in the region enclosed by $C$, so by Cauchy's integral formula
  \begin{equation}
    f(1) = \frac{1}{2\pi i}\int_C \frac{z^2 - z + 1}{z - 1}\mathrm{d} z
    \Rightarrow \int_C \frac{z^2 - z + 1}{z - 1}\mathrm{d}z = 2 \pi i.
  \end{equation}
\end{solution}

\begin{corollary}{Cauchy's Derivative Formula}{}
  If $f$ is holomorphic in an open set $\Omega$,
  then $f$ has infinitely many complex derivatives in $\Omega$.
  Moreover, if $C \subset \Omega$ is a circle whose interior is also contained in $\Omega$,
  then
  \begin{equation}
    f^{(n)}(z) = \frac{n!}{2 \pi i} \int_C \frac{f(\zeta)}{(\zeta - z)^{n+1}}\mathrm{d} \zeta
  \end{equation}
  for all $z$ in the interior of $C$.
\end{corollary}

\begin{example}{Applications of Cauchy's Derivative Formula}{}
  Calculate
  \begin{equation}
    \int_C \frac{z^2 - z + 1}{(z-1)^2}\mathrm{d} z,
  \end{equation}
  where $C$ is the circle centered at the origin with a radius of $2$.
\end{example}

\begin{solution}
  
\end{solution}

\begin{corollary}{Cauchy Inequality}{}
  If $f$ is holomorphic in an open set that contains the closure of a disc $D$
  centered at $z_0$ and of radius of $R$, then
  \begin{equation}
    |f^{(n)}(z_0)| \leq \frac{n! \|f\|_C}{R^n},
  \end{equation}
  where $\|f\|_C = \sup \limits_{z \in C}|f(z)|$ denotes the supremum of $|f|$
  on the boundary circle $C$.
\end{corollary}

\begin{theorem}{Taylor Expansion of Holomorphic Functions}{}
  If $f$ is holomorphic in an open set $\Omega$,
  then $f$ has a power series expansion at any point $z_0 \in \Omega$,
  \begin{equation}
    f(z) = \sum\limits_{n = 0}^{\infty} \frac{f^{(n)}(z_0)}{n!}(z-z_0)^n, \quad n \in \mathbb{N}.
  \end{equation}
   And $f$ is infinitely differentiable.
\end{theorem}

\subsection{Liouville's Theorem}

\begin{theorem}{Liouville's Theorem}{}
  If a complex-valued function $f$ is entire and bounded, then $f$ is constant.
\end{theorem}

\begin{proof}
  It suffices to prove that $f^{\prime} = 0$.
  For each $z_0 \in \mathbb{C}$, and all $R > 0$, the Cauchy inequalities yield
  \begin{equation}
    |f^{\prime}(z_0)| \leq \frac{B}{R},
  \end{equation}
  where $B$ is a bound for $f$. Letting $R \rightarrow +\infty$ gives the desired result.
\end{proof}

\begin{theorem}{Fundamental Theorem for Algebra}{}
  Every polynomial $P(z) = a_nz^n + \cdots + a_0$ of degree $n \geq 1$
  has precisely $n$ roots in $\mathbb{C}$.
  If these roots are denoted by $w_1,\cdots,w_n$, then $P$ can be factored as
  \begin{equation}
    P(z) = a_n(z-w_1)(z-w_2)\cdots (z-w_n).
  \end{equation}
\end{theorem}

\begin{proof}
  We first prove that for all non-constant polynomial $P(z)$,
  there is at least one root in $\mathbb{C}$.
  If $P$ has no roots, then $\frac{1}{P(z)}$ is entire and bounded in $\mathbb{C}$.
  By Liouville's theorem, $\frac{1}{P(z)}$ is constant,
  this contradicts the assumption that $P(z)$ is non-constant.

  Then we can get the factorization $P(z) = (z-w_1)Q(z)$,
  where $Q(z)$ is a polynomial with degree $n-1$.
  By induction, we can conclude that
  \begin{equation}
    P(z) = c(z-w_1)\cdots (z-w_n),
  \end{equation}
  and comparing the coefficients yields $c = a_n$.
\end{proof}


\subsection{Morera's Theorem}

\begin{theorem}{Morera's Theorem}{}
  Suppose $f$ is a continuous function in the simply connected region $D$,
  and for any closed curve $C$ in $D$,
  \begin{equation}
    \int_C f(z)\mathrm{d} z = 0,
  \end{equation}
  then $f$ is holomorphic.
\end{theorem}


