

\section{Poles and Zeros}

\subsection{Singularities and Poles}

\begin{definition}{Isolated Singularity}{}
  Let $f(z)$ be a complex-valued function that is holomorphic
  in some deleted neighborhood of $z_0$, but not defined at $z_0$.
  Then $z_0$ is said to be an \emph{isolated singularity of $f(z)$}.
\end{definition}

\begin{definition}{Pole Singularity and Essential Singularity}{}
  An isolated singularity $z_0$ is called a \emph{pole of $f(z)$}
  if $\lim \limits _{z \rightarrow z_0}f(z)$ goes up to infinity.
  If there exists $n \in \mathbb{Z}^+$ such that
  \begin{equation}
    g(z) = (z-z_0)^n f(z)
  \end{equation}
  is holomorphic and nonzero at $z_0$,
  then $z_0$ is said to be a \emph{pole of order $n$}.
  If no finite $n$ exists, then $z_0$ is said to be an \emph{essential singularity}.
\end{definition}

\begin{definition}{Principal Part and Residue}{}
  If $f$ has a pole of order $n$ at $z_0$, and
  \begin{equation}
    f(z) = \frac{a_{-n}}{(z-z_0)^n} + \frac{a_{-n+1}}{(z-z_0)^{n-1}} + \cdots + \frac{a_{-1}}{z-z_0} + G(z)
    := H(z) + G(z),
  \end{equation}
  then $H(z)$ is said to be the \emph{principal part of $f$ at $z_0$},
  and $a_{-1}$ the \emph{residue of $f$ at $z_0$}.
\end{definition}

\subsection{Zeros}

\begin{definition}{Zero}{}
  A complex number $z_0$ is a \emph{zero of $f(z)$} if
  \begin{equation}
    f(z_0) = 0.
  \end{equation}
  Then there exists a neighborhood $U \subset \Omega$ of $z_0$,
  and a unique positive integer $n$ such that
  \begin{equation}
    f(z) = (z - z_0)^n g(z), \quad z \in U,
  \end{equation}
  where $g(z)$ is a non-vanishing holomorphic function on $U$.
  Then $z_0$ is said to be a \emph{zero of order $n$}.
\end{definition}

\section{The Residue Formula}

\subsection{The Residue Formula}

\begin{theorem}{The Residue Formula}{}
  Suppose that $f$ is holomorphic in an open set containing a circle $C$
  and its interior, except for a pole at $z_0$ inside $C$. Then
  \begin{equation}
    \int_C f(z)\mathrm{d} z = 2\pi i \operatorname{res}_{z_0} f.
  \end{equation}
\end{theorem}






