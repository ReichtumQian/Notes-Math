

\section{The Laurent Series}

\begin{theorem}{Laurent Series}{}
  Let $f(z)$ be a complex-valued function that is holomorphic in $A = \{z \in \mathbb{C}: r < |z-z_0| < R\}$.
  Then $f(z)$ can be uniquely represented by
  \begin{equation}
    f(z) = \sum\limits_{n = -\infty}^{+\infty} a_n(z-z_0)^n, \quad z \in A,
  \end{equation}
  where $a_n = \frac{1}{2\pi i} \oint_C \frac{f(\zeta)}{(\zeta - z_0)^{n+1}}\mathrm{d} \zeta$,
  $C$ is any positively oriented, simple closed contour lying within $A$ and enclosing $z_0$.
\end{theorem}

\begin{note}
  Laurent series vs Taylor series:
  (1) $f(z)$ can be expanded to Laurent series within a region contains singularities;
  (2) Laurent series converges globally instead of locally.
\end{note}


\section{Singularities and Zeros}

\begin{definition}{Principal Part and Analytic Part}{}
  Let $f(z)$ be a complex-valued function of the form $f(z) = \sum_{n = -\infty}^{+\infty}a_n(z-z_0)^n$.
  Then
  \begin{equation}
    \sum\limits_{n = 0}^{+\infty}a_n(z-z_0)^n, \quad
    \sum\limits_{n = -\infty}^{-1}a_n (z-z_0)^n
  \end{equation}
  are called the \emph{analytic part}
  and the \emph{principal part} of $f(z)$ at $z_0$, respectively.
\end{definition}

\begin{definition}{Singularities}{}
  If a complex-valued function $f(z)$ is defined in a punctured neighborhood of $z_0$,
  then $z_0$ is said to be an \emph{isolated singularity of $f(z)$}.
  \begin{enumerate}
  \item If $f(z) = \sum_{n = 0}^{+\infty}a_n(z-z_0)^n$, $z_0$ is said to be an \emph{removable singularity};
  \item If $f(z) = \sum_{n=-m}^{+\infty}a_n(z-z_0)^n$, $z_0$ is said to be a \emph{pole singularity of order $m$};
  \item If $f(z) = \sum_{n = -\infty}^{+\infty}a_n(z-z_0)^n$, $z_0$ is said to be an \emph{essential singularity}.
  \end{enumerate}
\end{definition}

\begin{definition}{Zero}{}
  If a complex-valued function $f(z)$ can be represented as $f(z) = \sum_{n = m}^{+\infty} a_n(z-z_0)^n$,
  then $z_0$ is said to be a \emph{zero of order $m$ of $f(z)$}.
\end{definition}

\section{The Residue Formula}

\begin{definition}{Residue}{}
  Given the Laurent series of a holomorphic function
  \begin{equation}
    f(z) = \sum_{n = -\infty}^{+\infty}a_n(z-z_0)^n,
  \end{equation}
  then $a_{-1}$ is said to be the \emph{residue of $f(z)$ at $z_0$},
  denoted as $\operatorname{res}_{z_0} f$.
\end{definition}

\begin{theorem}{The Residue Formula}{}
  Suppose that $f$ is holomorphic in an open set containing a circle $C$
  and its interior, except for a pole at $z_0$ inside $C$. Then
  \begin{equation}
    \operatorname{res}_{z_0} f = \frac{1}{2 \pi i} \int_C f(z)\mathrm{d} z.
  \end{equation}
\end{theorem}

\section{Meromorphic Functions}

\begin{definition}{Meromorphic Function}{}
  A complex-valued function $f(z)$ on an open set $\Omega$ is \emph{meromorphic} if there exists
  $\{z_n\}_{n=1}^{+\infty}$ that has on limit points in $\Omega$ such that
  \begin{enumerate}
  \item $f$ is holomorphic in $\Omega - \{z_1,z_2,\cdots\}$;
  \item $f$ has poles at $\{z_0,z_1,\cdots\}$.
  \end{enumerate}
\end{definition}


