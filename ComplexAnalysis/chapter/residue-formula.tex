

\section{The Laurent Series}

\begin{theorem}{Laurent Series}{}
  Let $f(z)$ be a complex-valued function that is holomorphic in $A = \{z \in \mathbb{C}: r < |z-z_0| < R\}$.
  Then $f(z)$ can be uniquely represented by
  \begin{equation}
    f(z) = \sum\limits_{n = -\infty}^{+\infty} a_n(z-z_0)^n, \quad z \in A,
  \end{equation}
  where $a_n = \frac{1}{2\pi i} \oint_C \frac{f(\zeta)}{(\zeta - z_0)^{n+1}}\mathrm{d} \zeta$,
  $C$ is any positively oriented, simple closed contour lying within $A$ and enclosing $z_0$.
\end{theorem}

\begin{note}
  Laurent series vs Taylor series:
  (1) $f(z)$ can be expanded to Laurent series within a region contains singularities;
  (2) Laurent series converges globally instead of locally.
\end{note}


\section{Singularities and Zeros}

\begin{definition}{Principal Part and Analytic Part}{}
  Let $f(z)$ be a complex-valued function of the form $f(z) = \sum_{n = -\infty}^{+\infty}a_n(z-z_0)^n$.
  Then
  \begin{equation}
    \sum\limits_{n = 0}^{+\infty}a_n(z-z_0)^n, \quad
    \sum\limits_{n = -\infty}^{-1}a_n (z-z_0)^n
  \end{equation}
  are called the \emph{analytic part}
  and the \emph{principal part} of $f(z)$ at $z_0$, respectively.
\end{definition}

\begin{definition}{Singularities}{}
  If a complex-valued function $f(z)$ is defined in a punctured neighborhood of $z_0$,
  then $z_0$ is said to be an \emph{isolated singularity of $f(z)$}.
  \begin{enumerate}
  \item If $f(z) = \sum_{n = 0}^{+\infty}a_n(z-z_0)^n$, $z_0$ is said to be an \emph{removable singularity};
  \item If $f(z) = \sum_{n=-m}^{+\infty}a_n(z-z_0)^n$, $z_0$ is said to be a \emph{pole singularity of order $m$};
  \item If $f(z) = \sum_{n = -\infty}^{+\infty}a_n(z-z_0)^n$, $z_0$ is said to be an \emph{essential singularity}.
  \end{enumerate}
\end{definition}

\begin{definition}{Zero}{}
  If a complex-valued function $f(z)$ can be represented as $f(z) = \sum_{n = m}^{+\infty} a_n(z-z_0)^n$,
  then $z_0$ is said to be a \emph{zero of order $m$ of $f(z)$}.
\end{definition}

\section{The Residue Formula}

\subsection{The Concept and Representation of Residue}

\begin{definition}{Residue}{}
  Given the Laurent series of a holomorphic function
  \begin{equation}
    f(z) = \sum_{n = -\infty}^{+\infty}a_n(z-z_0)^n,
  \end{equation}
  then $a_{-1}$ is said to be the \emph{residue of $f(z)$ at $z_0$},
  denoted as $\operatorname{res}_{z_0} f$.
  If $z_0 = \infty$, then $- a_{-1}$ is said to be the residue.
\end{definition}

\begin{proposition}{Representation of Residue}{}
  If $z_0$ is an $m$-order pole of $f(z)$, i.e.,
  $\varphi(z) = (z-z_0)^mf(z) $, $\varphi(z_0) \neq 0$, then
  \begin{equation}
    \operatorname{res}_{z_0} f = \lim \limits _{z \rightarrow z_0}\frac{\varphi^{(m-1)}(z)}{(m-1)!}.
  \end{equation}
  If $z_0 = \infty$, then the residue of $f$ at $\infty$ is the negative of the sum of the residues at
  all other finite singularities
  \begin{equation}
    \operatorname{res}_{z = \infty} f = - \sum\limits_{k = 1}^n \operatorname{res}_{z_k} f(z).
  \end{equation}
\end{proposition}

\begin{proof}
  $\varphi(z) = (z-z_0)^m f(z) = \sum _{n = 0}^{+\infty} a_{n-m}(z - z_0)^n$.
  The term with coefficent $a_{-1}$ is when $n = m-1$, that is $a_{-1}(z - z_0)^{m-1}$, so
  \begin{equation}
    \varphi^{(m-1)}(z_0) = (m-1)! a_{-1}.
  \end{equation}
  This implies the conclusion.
\end{proof}

\begin{example}{Practice on Finding Residue}{}
  Find the residue of
  \begin{equation}
    (1) \frac{z}{(z-1)(z+1)^2}, \quad z = \pm 1, \infty, \quad
    (2) \frac{1}{\sin z}, \quad z = n \pi, \quad
    (3) \frac{1}{e^{z-1}}, \quad z = 1, \infty.
  \end{equation}
\end{example}

\begin{solution}
  (1) $z = 1$ is a $1$-order pole, and $\varphi(z) = (z-1)f(z)= \frac{z}{(z+1)^2}$.
  Then the residue is $\varphi(1) = \frac{1}{4}$.
  $z = -1$ is a $2$-order pole, $\varphi(z) = (z+1)^2f(z) = \frac{z}{z-1}$.
  Then the residue is $\varphi^{\prime}(-1) = - \frac{1}{4}$.
  $z = \infty$ is $- [\operatorname{res}_1f + \operatorname{res}_{-1}f] = 0$.

  (2) $z = n\pi$ are $1$-order poles, $\varphi(z) = \frac{z - n\pi}{\sin z}$.
  Apply the L'Hospital formula,
  \begin{equation}
    \varphi(z_0) = \frac{1}{\cos n\pi } = (-1)^n,
  \end{equation}
  so the residue is $(-1)^n$ at $z = n \pi$.

  (3) $z = 1$ is a infinite-order pole, but $e^x = \sum _{n = 0}^{\infty} \frac{x^n}{n!}$,
  so the residue is $1$.
  The residue of $z = \infty$ is $-1$.
\end{solution}

\subsection{The Residue Formula}

\begin{theorem}{The Residue Formula}{}
  Suppose that $f$ is holomorphic in an open set containing a circle $C$
  and its interior, except for a pole at $z_0$ inside $C$. Then
  \begin{equation}
    \int_C f(z)\mathrm{d} z
    =
    2 \pi i\operatorname{res}_{z_0} f.
  \end{equation}
\end{theorem}

\begin{proof}
  By the coefficent of Laurent series.
\end{proof}

\begin{example}{Apply Residue Formula to Calculate Integrals}{}
  Calculate the following integrals
  \begin{equation}
    \int_{|z| = 1} \frac{1}{z \sin z}\mathrm{d} z, \quad
  \end{equation}
\end{example}

\begin{solution}
  By Residue formula, $\int_{|z| = 1} \frac{1}{z \sin z}\mathrm{d} z = 2 \pi i \operatorname{res}_{z = 0}f(z)$.
  $z = 0$ is a $2$-order pole, and $\varphi(z) = z^2 f(z) = \frac{z}{\sin z}$.
  Then
  \begin{equation}
    \operatorname{res}_{z = 0} f(z)
    = \lim \limits _{z \rightarrow 0} \varphi^{\prime}(z)
    = \frac{\sin z - z \cos z}{\sin^2 z}.
  \end{equation}
  Apply the Taylor expansion, we get the limit is equal to
  $\lim \limits _{z \rightarrow 0} - \frac{2}{3} \cdot \frac{z^3}{z^2} = 0$.
  Then the value of the integral is $0$.
\end{solution}

\begin{corollary}{Cauchy Residue Formula}{}
  Let $U$ be an open, simply connected domain, and $C$ a positively oriented, piecewise smooth,
  simple closed contour in $U$.
  If $f(z)$ is holomorphic in $U$ except isolated singularities $z_1,z_2,\cdots,z_n$, then
  \begin{equation}
    \oint_C f(z)\mathrm{d} z
    =
    2 \pi i \sum\limits_{k = 1}^n \operatorname{res}_{z_k}f(z).
  \end{equation}
\end{corollary}

\begin{example}{Apply Cauchy Residue Fomula to Calculate Integrals}{}
  Calculate the following integrals:
  \begin{equation}
    \frac{1}{2\pi i} \int_{|z| = 2} \frac{e^{iz}}{1 + z^2}\mathrm{d} z.
  \end{equation}
\end{example}

\begin{solution}
  $z = \pm i$ are $1$-order poles. Consider $z = i$, $\varphi(z) := \frac{e^{iz}}{z + i}$,
  then $\operatorname{res}_{z = i} f = - \frac{i}{2e}$.
  Similarly $\operatorname{res}_{z = -i} f = - \frac{e i}{2}$.
  By Cauchy residue formula, the integral is
  \begin{equation}
    - \frac{i}{2e} - \frac{e i}{2}.
  \end{equation}
\end{solution}

\subsection{Integrateing Trigonometric Rational Functions}

\begin{proposition}{Integrating Rational Functions of $\sin \theta$ and $\cos \theta$}{}
  Let $R(\cos \theta, \sin \theta)$ be a rational function of $\sin \theta$ and $\cos \theta$,
  then
  \begin{equation}
    \int_0^{2\pi} R(\cos \theta, \sin \theta) \mathrm{d} \theta
    = \int_{|z| = 1} R(\frac{z + z^{-1}}{2}, \frac{z - z^{-1}}{2i}) \cdot \frac{1}{iz}\mathrm{d} z.
  \end{equation}
\end{proposition}

\begin{proof}
  Consider $z = e^{i\theta}$, then $\cos \theta = \frac{z + z^{-1}}{2}$,
  $\sin \theta = \frac{z - z^{-1}}{2i}$, $\mathrm{d} \theta = \frac{\mathrm{d} z}{iz}$.
  Then change of variation yields the conclusion.
\end{proof}

\begin{example}{Practice on Integrating Rational Functions of $\sin \theta$ and $\cos \theta$}{}
  Calculate the following integrals
  \begin{equation}
    \int_0^{2\pi} \frac{1}{a + \cos \theta}\mathrm{d} \theta ~~ (a > 1), \quad
    \int_0^{2\pi} \frac{1}{(2 + \sqrt{3} \cos \theta)^2}\mathrm{d} \theta
  \end{equation}
\end{example}

\begin{solution}
  (1) Let $z = e^{i\theta}$, $\cos \theta = \frac{z + z^{-1}}{2}$, $\mathrm{d} \theta = \frac{1}{iz}\mathrm{d} z$, then
  \begin{equation}
    \int_0^{2\pi} \frac{1}{a + \cos \theta}\mathrm{d} \theta = \int_{|z| = 1} \frac{-2i}{z^2 + 2az + 1}\mathrm{d} z.
  \end{equation}
  Poles of the above integral are $\alpha = -a + \sqrt{a^2 - 1}$ and $\beta = -a - \sqrt{a^2 - 1}$.
  Since $a > 1$, then $|\alpha| < 1$ and $|\beta| > 1$, so only $\alpha$ lies in the unit disk.
  Consider $\varphi(z) = \frac{-2i}{z - \beta}$, then the residue at $\alpha$ is $\varphi(\alpha) = \frac{2i}{\beta - \alpha}$.
  Therefore
  \begin{equation}
    \int_{|z| = 1} \frac{-2i}{z^2 + 2az + 1}\mathrm{d} z = 2 \pi i \operatorname{res}_{z = \alpha}f
    = 2 \pi i \frac{2i}{\beta - \alpha} = \frac{2\pi}{\sqrt{a^2 - 1}}.
  \end{equation}

  (2) Similarly, we can calculate in following steps
  \begin{align}
    \int_0^{2\pi} \frac{1}{(2 + \sqrt{3} \cos \theta)^2} \mathrm{d} \theta
    & = \frac{4}{3i}\int_{|z| = 1} \frac{z}{(z^2 + \frac{4}{\sqrt{3}}z + 1)^2}\mathrm{d} z
      = \frac{4}{3i} \int_{|z| = 1} \frac{z}{(z + \sqrt{3})^2(z + \frac{1}{\sqrt{3}})^2} \mathrm{d} z
  \end{align}
  Denote $\alpha = -\sqrt{3}$ and $\beta = - \frac{1}{\sqrt{3}}$, then $|\alpha| > 1$, $|\beta| < 1$.
  Define $\varphi(z) = \frac{z}{(z+\sqrt{3})^2}$, then $\varphi^{\prime}(z) = \frac{\sqrt{3} - z}{(z + \sqrt{3})^3}$,
  $\varphi(\beta) = \frac{3}{2}$.
  \begin{equation}
    \int_0^{2\pi} \frac{1}{(2 + \sqrt{3} \cos \theta)^2} \mathrm{d} \theta
    = \frac{4}{3i} \cdot 2 \pi i \cdot \frac{3}{2} = 4 \pi.
  \end{equation}
\end{solution}

\begin{note}
  Similarly, we can calculate integrals involve $\tan \theta$ by 
  \begin{equation}
    \tan \theta = \frac{\frac{e^{i\theta} - e^{-i\theta}}{2i}}{\frac{e^{i\theta} + e^{-i\theta}}{2}}
    = \frac{1}{i} \frac{e^{2i\theta} - 1}{e^{2i\theta} + 1}.
  \end{equation}
\end{note}

\begin{example}{Calculate Integrals involve $\tan \theta$}{}
  Calculate the followin integrals
  \begin{equation}
    \int_0^{\pi} \tan (\theta + ia) \mathrm{d} \theta ~~ (a \neq 0).
  \end{equation}
\end{example}

\subsection{Integrating Rational Functions}

\begin{proposition}{The Residue Formula for Rational Functions}{}
  Let $f(z) = \frac{P(z)}{Q(z)}$ where $P(z), Q(z)$ are polynomials satisfying
  (1) $(f(z), Q(z)) = 1$;
  (2) $\operatorname{deg}(Q(z)) \geq \operatorname{deg}(P(z)) + 2$;
  (3) $Q(z)$ has no roots on $\mathbb{R}$.
  Then 
  \begin{equation}
    \int_{-\infty}^{+\infty} f(x) \mathrm{d} x
    = 2\pi i \sum \operatorname{res}_{z = a_k}f(z),
  \end{equation}
  where $a_k$ are residues of $f(z)$ satisfying $\operatorname{Im}(a_k) > 0$.
\end{proposition}

\begin{example}{Apply Residue Formula to Calculate Integrals of Rational Functions}{}
  Calculate the following integrals:
  \begin{equation}
    \int_0^{+\infty} \frac{x^2}{(x^2 + 1)(x^2 + 4)}\mathrm{d} x
  \end{equation}
\end{example}

\begin{solution}
  (1) 
\end{solution}


