

\section{Complex Numbers}

\subsection{Algebra Form of Complex Numbers}

\begin{definition}{Complex Number}{}
  A \emph{complex number} takes the form $z = x + iy$, where $x$ and $y$ are real,
  and $i$ is an imaginary number that satisfies
  \begin{equation}
    i^2 = -1.
  \end{equation}
  We call $x$ and $y$ the \emph{real part} and \emph{imaginary part} of $z$
  respectively, denoted as
  \begin{equation}
    x = \ReOp{z}, \quad y = \ImOp{z}.
  \end{equation}
\end{definition}

\begin{definition}{Complex Conjugate of Complex Number}{}
  The \emph{complex conjugate} of $z = x + iy$ is
  \begin{equation}
    \overline{z} := x - iy.
  \end{equation}
\end{definition}

\begin{definition}{Addition and Multiplication of Complex Number}{}
  Let $z_1 = x_1 + iy_1$ and $z_2 = x_2 + iy_2$, define
  \begin{equation}
    z_1 + z_2 = (x_1 + x_2) + i(y_1 + y_2), \quad
    z_1 z_2 = (x_1 + iy_1)(x_2 + iy_2).
  \end{equation}
\end{definition}

\begin{example}{Practice on Definition of Complex Numbers}{}
  Let $\frac{\overline{z}}{z} = a + bi$, prove that $a^2 + b^2 = 1$.
\end{example}

\begin{proof}
  Assume $z = x + iy$,
  multiply the denominator by $\overline{z}$, we get
  \begin{equation}
    \frac{\overline{z}}{z} = \frac{\overline{z}^2}{z \cdot \overline{z}}
    = \frac{(x-iy)^2}{x^2 + y^2} = \frac{x^2 - 2ixy - y^2}{x^2 + y^2},
  \end{equation}
  then $a = \frac{x^2 - y^2}{x^2 + y^2}$, $b = \frac{-2xy}{x^2 + y^2}$,
  it is not hard to verify that $a^2 + b^2 = 1$.
\end{proof}

\begin{definition}{Absolute Value of Complex Number}{}
  Let $z = x + iy$, then the \emph{absolute value of $z$} is
  \begin{equation}
    |z| := \sqrt{x^2 + y^2}.
  \end{equation}
\end{definition}

\begin{proposition}{Properties of Absolute Value}{}
  Given $z, w$ two complex numbers, then
  \begin{equation}
    (1) |z|^2 = z \cdot \overline{z}, \quad
    (2) \frac{1}{z} = \frac{\overline{z}}{|z|^2}, \quad
    (3) |z| \leq |x| + |y|, \quad
    (4) |z + w| \leq |z| + |w|.
  \end{equation}
\end{proposition}


\subsection{Polar Form of Complex Numbers}

\begin{definition}{Polar Form of Complex Numbers}{}
  The \emph{polar form} of complex numbers is
  \begin{equation}
    z = r e^{i\theta}.
  \end{equation}
  where $r = |z|$ and $e^{i\theta} = \cos \theta + i \sin \theta$.
\end{definition}

\begin{example}{Application of Polar Form}{}
  Prove that for any complex number $z \neq -1$, if $|z| = 1$,
  then there exists $t \in \mathbb{R}$ such that
  \begin{equation}
    z = \frac{1 + it}{1 - it}.
  \end{equation}
\end{example}

\begin{proof}
  Consider $z = \cos \theta + i \sin \theta$, then by universal formula of
  trigonometric function
  \begin{equation}
    z = \frac{1 - t^2}{1 + t^2} + \frac{2i t}{1 + t^2} = \frac{1+it}{1 - it},
  \end{equation}
  where $t = \tan \frac{\theta}{2}$.
\end{proof}

\begin{proposition}{Solutions of $z^n = re^{i\theta}$}{}
  The roots of $z^n = re^{i\theta}$ are
  \begin{equation}
    z_k = \sqrt[n]{r} e^{\frac{2k\pi + \theta}{n}i}, \quad k = 0,1,\cdots,n-1.
  \end{equation}
\end{proposition}

\begin{example}{Applications of Solutions of $z^n = re^{i\theta}$}{}
  Find the solutions of
  \begin{equation}
    (1+z)^5 = (1-z)^5.
  \end{equation}
\end{example}

\begin{solution}
  The equation is equivalent to 
  \begin{equation}
    \left( \frac{1+z}{1-z} \right)^5 = 1 \Rightarrow
    \frac{1+z}{1-z} = e^{\frac{2k\pi}{5}i} := w_k, \quad k = 0,1,2,3,4.
  \end{equation}
  then $\frac{1}{1-z} - 1 = w_k$, and $z = \frac{w_k}{w_k + 1}$.
\end{solution}

\section{Holomorphic Functions}

\subsection{Concept of Holomorphic Function}

\begin{definition}{Holomorphic Function}{}
  Let $\Omega$ be an open set in $\mathbb{C}$ and $f$ a complex-valued function
  on $\Omega$. The function $f$ is \emph{holomorphic at $z_0 \in \Omega$} if
  \begin{equation}
    f^{\prime}(z_0) := \lim \limits _{h \rightarrow 0}\frac{f(z_0 + h) - f(z_0)}{h}, \quad h \in \mathbb{C}
  \end{equation}
  exists. We call the limit \emph{derivative of $f$ at $z_0$}, denoted as $f^{\prime}(z_0)$.
  If $f$ is holomorphic in $\mathbb{C}$, then it is said to be an \emph{entire function}.
\end{definition}

\begin{example}{Examples of Non-Holomorphic Functions}{}
  Prove that $f(z) = \overline{z}$ is not holomorphic.
\end{example}

\begin{proof}
  Consider the limit
  \begin{equation}
    \lim \limits _{h \rightarrow 0} \frac{ \overline{z_0 + h} - \overline{z_0}}{h}
    = \lim \limits _{h \rightarrow 0}\frac{\overline{h}}{h}.
  \end{equation}
  The limit of $\frac{\overline{h}}{h}$ does not exist, so $f(z)$ is not holomorphic.
\end{proof}

\begin{proposition}{Representation of Complex Derivative}{}
  If $f(z)$ is holomorphic at $z_0$, then
  \begin{equation}
    f^{\prime}(z_0) = u_x(z_0) + v_x(z_0) i = v_y(z_0) - iu_y(z_0).
  \end{equation}
\end{proposition}

\begin{proof}
  By the definition of derivative, let $h \in \mathbb{R}$, then
  \begin{equation}
    f^{\prime}(z_0) = \lim \limits _{h \rightarrow 0} \frac{u(x_0+h, y_0) + i v(x_0 + h, y_0) - u(x_0, y_0) - iv(x_0, y_0)}{h}
    = u_x(z_0) + v_x(z_0)i.
  \end{equation}
  Similarly, let $h = ik$, then
  \begin{equation}
    f^{\prime}(z_0) = \lim \limits _{k \rightarrow 0} \frac{u(x_0,y_0+k) + i v(x_0, y_0 + k) - u(x_0,y_0) - iv(x_0,y_0)}{ik}
    = v_y(z_0) - iu_y(z_0).
  \end{equation}
  This completes the proof.
\end{proof}

\begin{example}{Application of Complex Derivatives}{}
  If $f(z)$ is holomorphic in $D$, and $f^{\prime}(z) = 0$ for all $z \in D$.
  Prove that $f(z)$ is constant in $D$.
\end{example}

\begin{proof}
  We know $0 = f^{\prime}(z) = u_x(z)+iv_x(z) = v_y(z)-iu_y(z)$,
  which implies that
  \begin{equation}
    u_x(z) = u_y(z) = v_x(z) = v_y(z) = 0, \quad z \in D.
  \end{equation}
  This means $f(z)$ is contant in $D$.
\end{proof}

\subsection{Cauchy-Riemann Equation}

\begin{definition}{Partial Derivatives with Respect to $z$ and $\overline{z}$}{}
  Let $f$ be a complex-valued function,
  then the derivatives of $f$ with respect to $z$ and $\overline{z}$ are
  \begin{equation}
    \frac{\partial f}{\partial z} := \frac{1}{2} \left( \frac{\partial f}{\partial x} + \frac{1}{i} \frac{\partial f}{\partial y} \right), \quad
    \frac{\partial f}{\partial \overline{z}} := \frac{1}{2} \left( \frac{\partial f}{\partial x} - \frac{1}{i} \frac{\partial f}{\partial y} \right).
  \end{equation}
\end{definition}

\begin{note}
  Why we define the above partial derivatives?
  Consider $f(x,y) = f(z, \overline{z})$,
  with $x(z, \overline{z}) = \frac{1}{2}(z + \overline{z})$ and $y(z,
  \overline{z}) = \frac{1}{2}(z - \overline{z})$, then
  \begin{equation}
    f_z = f_x x_z + f_y y_z = \frac{1}{2}(f_x + \frac{1}{i}f_y), \quad
    f_{\overline{z}} = f_x x_{\overline{z}} + f_y y_{\overline{z}} = \frac{1}{2}(f_x - \frac{1}{i}f_y).
  \end{equation}
\end{note}

\begin{theorem}{Cauchy-Riemann Equation}{}
  A complex-valued function $f(z) = u(z) + iv(z)$ is holomorphic at $z_0 \in \mathbb{C}$ if and
  only if
  \begin{equation}
    \frac{\partial u}{\partial x} = \frac{\partial v}{\partial y},
    \frac{\partial u}{\partial y} = - \frac{\partial v}{\partial x}, \quad
    \text{or} \quad
    \frac{\partial f}{\partial \overline{z}} = 0,
  \end{equation}
  and $u, v$ have continuous derivatives at $z_0$.
\end{theorem}

\begin{example}{Counterexample of Cauchy-Riemann Equation}{}
  Prove the following $f(z)$ satisfies Cauchy-Riemann equation,
  but is not differentiable at the origin.
  \begin{equation}
    f(z) =
    \begin{cases}
      \frac{x^3 - y^3 + i(x^3 + y^3)}{x^2 + y^2} & \text{ if } z \neq 0;\\
      0 & \text{ if } z = 0.
    \end{cases}
  \end{equation}
\end{example}

\begin{proof}
  Write $f(z) = u(x,y) + v(x,y)i$, then
  \begin{equation}
    u(x, 0) = x, u_x(x, 0) = 1, u_x(0, 0) = 1; \quad
    u(0, y) = -y, u_y(0, y) = -1, u_y(0, 0) = -1.
  \end{equation}
  \begin{equation}
    v(x,0) = x, v_x(x,0) = 1, v_x(0,0) = 1; \quad
    v(0,y) = y, v_y(0, y) = 1, v_y(0, 0) = 1.
  \end{equation}
  The above derivatives satisfy the Cauchy-Riemann equation.
  Let $y = kx$, then the limit $\lim \limits _{(x,y) \rightarrow (0,0)}\frac{f(x,y) - f(0, 0)}{x+iy}$
  does not exist, so it is not differentiable at the origin.
\end{proof}

\begin{note}
  Cauchy-Riemann equation implies that any function that relies on
  $\overline{z}$ is not holomorphic.
  E.g., $\overline{z}$, $|z|^2 = z\overline{z}$.
\end{note}

\begin{example}{What Happens When Conjugate is not Zero}{}
  Suppose $f(z)$ is holomorphic in $D$.
  If (1) $\overline{f(z)}$ is holomorphic,
  or (2) $|f(z)|$ is constant,
  or (3) $\operatorname{Re}(f(z)) \equiv 0$,
  prove that $f(z)$ is constant in $D$.
\end{example}

\begin{proof}
  (1) Since $\overline{f(z)} = u - iv$ is holomorphic, so it follows Cauchy-Riemann equation, i.e.,
  \begin{equation}
    u_x = -v_y, \quad u_y = v_x.
  \end{equation}
  At the same time, $f(z)$ also follows Cauchy-Riemann equation.
  We can obtain that $u_x = u_y = v_x = v_y = 0$ for all $z \in D$,
  which means $f(z)$ is constant in $D$.

  (2) $\overline{f(z)} = \frac{|f(z)|^2}{f(z)}$, since the numerator is constant,
  $\overline{f(z)}$ is holomorphic, so $f(z)$ is constant.

  (3) $u(x,y) \equiv 0$ means that $u_x(x,y)$ and $u_y(x,y)$ are zero functions.
  Then Cauchy-Riemann equation implies that $v_x$ and $v_y$ are also zero functions.
\end{proof}



