

\section{Complex Numbers}

\subsection{Algebra Form of Complex Numbers}

\begin{definition}{Complex Number}{}
  A \emph{complex number} takes the form $z = x + iy$, where $x$ and $y$ are real,
  and $i$ is an imaginary number that satisfies
  \begin{equation}
    i^2 = -1.
  \end{equation}
  We call $x$ and $y$ the \emph{real part} and \emph{imaginary part} of $z$
  respectively, denoted as
  \begin{equation}
    x = \ReOp{z}, \quad y = \ImOp{z}.
  \end{equation}
\end{definition}

\begin{definition}{Addition and Multiplication of Complex Number}{}
  Let $z_1 = x_1 + iy_1$ and $z_2 = x_2 + iy_2$, define
  \begin{equation}
    z_1 + z_2 = (x_1 + x_2) + i(y_1 + y_2), \quad
    z_1 z_2 = (x_1 + iy_1)(x_2 + iy_2).
  \end{equation}
\end{definition}

\begin{definition}{Absolute Value of Complex Number}{}
  Let $z = x + iy$, then the \emph{absolute value of $z$} is
  \begin{equation}
    |z| := \sqrt{x^2 + y^2}.
  \end{equation}
\end{definition}

\begin{proposition}{Properties of Absolute Value}{}
  Given $z, w$ two complex numbers, then
  \begin{equation}
    (1) |z| \leq |x| + |y|, \quad
    (2) |z + w| \leq |z| + |w|.
  \end{equation}
\end{proposition}

\begin{proof}
  
\end{proof}

\begin{definition}{Complex Conjugate of Complex Number}{}
  The \emph{complex conjugate} of $z = x + iy$ is
  \begin{equation}
    \overline{z} := x - iy.
  \end{equation}
\end{definition}

\subsection{Polar Form of Complex Numbers}

\begin{definition}{Polar Form of Complex Numbers}{}
  The \emph{polar form} of complex numbers is
  \begin{equation}
    z = r e^{i\theta}.
  \end{equation}
  where $r = |z|$ and $e^{i\theta} = \cos \theta + i \sin \theta$.
\end{definition}

\subsection{Sequence of Complex Numbers}

\section{Complex Functions}


\section{Complex Integrals}





