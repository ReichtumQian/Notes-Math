
\section{Cauchy's Theorem}

\begin{theorem}{Cauchy's Theorem}{}
  If $f$ is holomorphic in the closure of a simply connected region $D \subset \mathbb{C}$,
  and $\gamma$ is a closed curve in $D$, then
  \begin{equation}
    \int_{\gamma} f(z)\mathrm{d} z = 0.
  \end{equation}
\end{theorem}

\begin{example}{Applications of Cauchy's Theorem}{}
  
\end{example}

\begin{example}{Apply Cauchy's Theorem to Calculate Improper Integrals}{}
  Apply Cauchy's theorem to 
  \begin{enumerate}
  \item Prove that if $\xi \in \mathbb{R}$,
    \begin{equation}
      e^{- \pi \xi^2} = \int_{-\infty}^{+\infty} e^{-\pi x^2} e^{-2\pi i x \xi} \mathrm{d} x.
    \end{equation}
  \end{enumerate}
\end{example}

\section{Cauchy's Integral Formula}

\begin{theorem}{Cauchy's Integral Formula}{}
  Suppose $f$ is holomorphic in the closure of an open set $D \subset \mathbb{C}$.
  Denote the boundary of $D$ with the positive orintation as $C$, then
  \begin{equation}
    f(z) = \frac{1}{2 \pi i} \int_C \frac{f(\zeta)}{\zeta - z}\mathrm{d} \zeta, \quad
    z \in D.
  \end{equation}
\end{theorem}

\begin{corollary}{Cauchy's Derivative Formula}{}
  If $f$ is holomorphic in an open set $\Omega$,
  then $f$ has infinitely many complex derivatives in $\Omega$.
  Moreover, if $C \subset \Omega$ is a circle whose interior is also contained in $\Omega$,
  then
  \begin{equation}
    f^{(n)}(z) = \frac{n!}{2 \pi i} \int_C \frac{f(\zeta)}{(\zeta - z)^{n+1}}\mathrm{d} \zeta
  \end{equation}
  for all $z$ in the interior of $C$.
\end{corollary}

\begin{corollary}{Cauchy Inequality}{}
  If $f$ is holomorphic in an open set that contains the closure of a disc $D$
  centered at $z_0$ and of radius of $R$, then
  \begin{equation}
    |f^{(n)}(z_0)| \leq \frac{n! \|f\|_C}{R^n},
  \end{equation}
  where $\|f\|_C = \sup \limits_{z \in C}|f(z)|$ denotes the supremum of $|f|$
  on the boundary circle $C$.
\end{corollary}

\section{Liouville's Theorem}

\begin{theorem}{Liouville's Theorem}{}
  If a complex-valued function $f$ is entire and bounded, then $f$ is constant.
\end{theorem}

\begin{corollary}{Fundamental Theorem for Algebra}{}
  
\end{corollary}


\section{Morera's Theorem}

\begin{theorem}{Morera's Theorem}{}
  Suppose $f$ is a continuous function in the simply connected region $D$,
  and for any closed curve $C$ in $D$,
  \begin{equation}
    \int_C f(z)\mathrm{d} z = 0,
  \end{equation}
  then $f$ is holomorphic.
\end{theorem}


