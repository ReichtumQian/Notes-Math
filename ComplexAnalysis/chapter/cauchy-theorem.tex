
\section{Cauchy's Theorem}

\subsection{Goursat's Theorem}

\begin{theorem}{Goursat's Theorem}{}
  If $\Omega$ is an open set in $\mathbb{C}$,
  and $T \subset \Omega$ a triangle whose interior is also contained in $\Omega$,
  then
  \begin{equation}
    \int_T f(z)\mathrm{d} z = 0
  \end{equation}
  whenever is holomorphic in $\Omega$.
\end{theorem}

\subsection{Cauchy's Theorem for a Disc}

\begin{theorem}{Existence of Primitives in a Disc}{}
  A holomorphic function in an open disc has a primitive in that disc.
\end{theorem}

\begin{theorem}{Cauchy's Theorem for a Disc}{}
  If $f$ is holomorphic in a disc, then
  \begin{equation}
    \int_{\gamma} f(z)\mathrm{d} z = 0
  \end{equation}
  for any closed curve $\gamma$ in that disc.
\end{theorem}

\subsection{Cauchy's Theorem for any Closed Curve}

\begin{theorem}{Cauchy's Theorem}{}
  If $f$ is holomorphic in a connected region $D \subset \mathbb{C}$,
  and $\gamma$ is a closed curve in $D$, then
  \begin{equation}
    \int_{\gamma} f(z)\mathrm{d} z = 0.
  \end{equation}
\end{theorem}

\section{Cauchy's Integral Formula}

\begin{theorem}{Cauchy's Integral Formula}{}
  Suppose $f$ is holomorphic in an open set that contains the closure of a disc $D$.
  If $C$ denotes the boundary circle of this disc with the positive orintation, then
  \begin{equation}
    f(z) = \frac{1}{2 \pi i} \int_C \frac{f(\zeta)}{\zeta - z}\mathrm{d} \zeta, \quad
    z \in D.
  \end{equation}
\end{theorem}

\begin{corollary}{Cauchy's Derivative Formula}{}
  If $f$ is holomorphic in an open set $\Omega$,
  then $f$ has infinitely many complex derivatives in $\Omega$.
  Moreover, if $C \subset \Omega$ is a circle whose interior is also contained in $\Omega$,
  then
  \begin{equation}
    f^{(n)}(z) = \frac{n!}{2 \pi i} \int_C \frac{f(\zeta)}{(\zeta - z)^{n+1}}\mathrm{d} \zeta
  \end{equation}
  for all $z$ in the interior of $C$.
\end{corollary}

\begin{corollary}{Cauchy Inequality}{}
  If $f$ is holomorphic in an open set that contains the closure of a disc $D$
  centered at $z_0$ and of radius of $R$, then
  \begin{equation}
    |f^{(n)}(z_0)| \leq \frac{n! \|f\|_C}{R^n},
  \end{equation}
  where $\|f\|_C = \sup \limits_{z \in C}|f(z)|$ denotes the supremum of $|f|$
  on the boundary circle $C$.
\end{corollary}

