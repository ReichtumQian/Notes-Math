
\section{Existence of Solutions}

\begin{proposition}{Conditions for the Existence of Solutions}{}
  A system of linear equations $Ax = b$ over $\mathbb{P}$ has a solution
  if and only if 
  \begin{equation}
    r(A) = r(\overline{A}),
  \end{equation}
  where $\overline{A} := (A|b)$.
\end{proposition}

\begin{proposition}{Conditions for the Same Solutions}{}
  Two system of linear equations $Ax = 0$ and $Bx = 0$ have same solutions
  if and only if $A$ and $B$ are row-equivalent.
\end{proposition}

\begin{example}{Applications of Same Solutions}{}
  Given that the following two systems of linear equations have the same
  solutions, find the value of $a,b$ and $c$.
  \begin{equation}
    \begin{cases}
      x_1 + 2x_2 + 3x_3 = 0\\
      2x_1 + 3x_2 + 5x_3=0\\
      x_1 + x_2 + ax_3 = 0
    \end{cases}
    \quad
    \begin{cases}
      x_1 + bx_2 + cx_3 = 0\\
      2x_1 + b^2x_2 + (c+1)x_3 = 0
    \end{cases}
  \end{equation}
\end{example}

\begin{solution}
  Hint: Denote the first system of linear equations as $Ax = 0$, and the second
  $Bx = 0$. Then apply elementary row operations to the following matrix
  \begin{equation}
    C =
    \begin{bmatrix}
      A\\
      B
    \end{bmatrix}.
  \end{equation}
  Make sure that all the rows corresponding to $B$ are reduced to zero rows.
\end{solution}

\begin{proposition}{Number of Solutions}{}
  For a system of linear equation $Ax = b$ satisfying $r(A) = r(\overline{A})$,
  \begin{enumerate}
  \item if $r(A) = n$, then $Ax = b$ has a unique solution;
  \item if $r(A) < n$, then $Ax = b$ has infinitely many solutions.
  \end{enumerate}
\end{proposition}


\section{Solving Systems of Linear Equations}

\subsection{Solving $Ax = 0$}

\begin{definition}{Free Variables}{}
  Let $A$ be a matrix with a rank of $r$,
  and $Ax = 0$ be a system of linear equations where $x = [x_1,\cdots,x_n]^T$.
  The variables $x_{r+1},\cdots,x_n$ are said to be \emph{free variables of $Ax = 0$}.
\end{definition}

\begin{definition}{Fundamental Solution System}{}
  Let $A$ be a matrix with a rank of $r$,
  and $Ax = 0$ be a system of linear equations where $x = [x_1,\cdots,x_n]^T$.
  If there exist vectors
  \begin{equation}
    \begin{cases}
      \eta_1 = (*,\cdots,*,1,0,\cdots,0)\\
      \eta_2 = (*,\cdots,*,0,1,\cdots,0)\\
      \quad \vdots\\
      \eta_{n-r} = (*,\cdots,*,0,0,\cdots,1)
    \end{cases},
  \end{equation}
  satisfying $A\eta_i = 0$.
  Then we call $\eta_1,\cdots,\eta_{n-r}$ the \emph{fundamental solution system of
    $Ax = 0$}.
\end{definition}

\begin{proposition}{Solution of Homogeneous Systems of Linear Equations}{}
  Any solution of the homogeneous system of linear equations $Ax = 0$ is
  a linear combination of its fundamental solution system.
\end{proposition}

\subsection{Solving $Ax = b$}

\begin{definition}{Particular Solution}{}
  Given a non-homogenous system of linear equations $Ax = b$,
  if a vector $\gamma$ satisfying 
  \begin{equation}
    A \gamma = b,
  \end{equation}
  then it is called a \emph{particular solution of $Ax = b$}.
\end{definition}

\begin{proposition}{Solution of Non-homogeneous Systems of Linear Equations}{}
  The general solution of the non-homogeneous system of linear equations $Ax = b$
  is the sum of its particular solution $\gamma$ and the general solution of
  the corresponding homogeneous system of linear equations $Ax = 0$.
  That is
  \begin{equation}
    x = \gamma + \operatorname{span} (\eta_1,\cdots,\eta_{n-r}).
  \end{equation}
\end{proposition}

\begin{example}{Solve Systems of Linear Equations}{}
  
\end{example}

\section{Least Square Problems}




