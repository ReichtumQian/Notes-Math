
\section{Basic Concepts and Properties of Determinants}

\subsection{Definition of Determinants}

\begin{definition}{Determinants}{}
  Let $A = (a_{ij})$ be an $n \times n$ square matrix,
  its \emph{determinant} is defined by
  \begin{equation}
    |A| := \sum\limits_{\sigma \in S_n} \mathrm{sgn}(\sigma) \prod\limits_{i = 1}^n a_{i, \sigma(i)},
  \end{equation}
  where $S_n$ is the set of all permutations of $\{1, 2, \cdots, n\}$,
  $\sigma$ is a permutation in $S_n$,
  $\mathrm{sgn}(\sigma)$ is the sign of $\sigma$,
  which is $+1$ if $\sigma$ is an even permutation,
  and $-1$ if $\sigma$ is an odd permutation,
  $a_{i, \sigma(i)}$ is the entry of $A$ in the $i$-th row and $\sigma(i)$-th column.
\end{definition}

\begin{example}{Basic Calculation of Determinants}{}
  Find the determinant of $A$ in the form of
  \begin{equation}
    A =
    \begin{bmatrix}
      0 & 0 & \cdots & 0 & a_1\\
      0 & 0 & \cdots & a_2 & 0 \\
      \vdots & \vdots & \ddots & \vdots & \vdots \\
      a_n & 0 & \cdots & 0 & 0
    \end{bmatrix}
  \end{equation}
\end{example}

\begin{solution}
  By the definition of the determinant, here the permutation $\sigma$ is
  \begin{equation}
    \sigma(i) = n - i + 1, \quad i = 1,2,\cdots,n.
  \end{equation}
  And $\operatorname{inv}(\sigma) = (n-1) + (n-2) + \cdots + 1 = \frac{n(n-1)}{2}$.
  Therefore,
  \begin{equation}
    |A| = (-1)^{\frac{n(n-1)}{2}} a_1a_2\cdots a_n,
  \end{equation}
  which completes the solution.
\end{solution}

\begin{proposition}{Elementary Transformation of Determinants}{}
  When performing elementary rwo-operations on a matrix,
  the corresponding changes to the determinant are as follows:
  \begin{enumerate}
  \item \textbf{Interchange}: Swapping two rows (or columns) changes the sign of
    the determinant.
  \item \textbf{Multiplication}: Multiplying a row (or column) by a scalar $k$
    multiplies the determinant by $k$.
  \item \textbf{Addition}: Adding a multiple of one row (or column) to another
    row (or column) does not change the value of the determinant.
  \end{enumerate}
\end{proposition}

\begin{proof}
  Hint: use the definition of determinants.
\end{proof}

\begin{definition}{Minor and Cofactor}{}
  Consider an $n \times n$ matrix $A = (a_{ij})$.
  Given two positive integers $i,j$,
  the \emph{minor of $a_{ij}$}, denoted as $M_{ij}$,
  is the determinant obtained by deleting the $i$-th row
  and $j$-th column of $A$.
  The \emph{cofactor of $a_{ij}$} is then defined as $A_{ij} = (-1)^{i+j}M_{ij}$.
\end{definition}

\begin{note}
  The concept of a minor can be generalized beyond the deletion of a single row
  and column.
  Specifically, a \emph{minor} can refer to the determinant of any square submatrix
  obtained by deleting multiple rows and multiple columns from $A$.
\end{note}

\subsection{Expansion of Determinants}

\begin{proposition}{Expansion of Determinants}{}
  Fix column $j$, the determinant $|A|$ can be expanded as
  \begin{equation}
    |A| = \sum\limits_{i = 1}^n a_{ij} A_{ij},
  \end{equation}
  where $A_{ij}$ is the cofactor of $a_{ij}$.
  Similarly, fix row $i$, the determinant $|A|$ can be expanded as
  \begin{equation}
    |A| = \sum\limits_{j = 1}^n a_{ij} A_{ij}.
  \end{equation}
\end{proposition}

\begin{proposition}{Expansion by Two Distinct Rows}{}
  Let $A = (a_{ij})$ be an $n \times n$ square matrix,
  and let $i, j$ be two distinct positive integers.
  Then
  \begin{equation}
    \sum\limits_{k = 1}^n a_{ik} A_{jk} = 0,
  \end{equation}
  where $A_{jk}$ is the cofactor of $a_{jk}$.
\end{proposition}

\begin{proof}
  Define a new matrix $B$ by replacing the $j$-th row of $A$ with the
  $i$-th row of $A$.
  Since the $i$-th and $j$-th rows of $B$ are identical, then $|B| = 0$.
  The determinant of $B$ can be computed using cofactor expansion along the
  $j$-th row, and we find that the above equation is equivalent to $|B|$,
  and the value is $0$.
\end{proof}

\section{Cramer's Rule}

\begin{theorem}{Cramer's Rule}{}
  Consider a system of $n$ linear equations in $n$ unknowns $Ax = b$.
  If the determinant of the coefficient matrix $|A| \neq 0$,
  then the system has a unique solution given by
  \begin{equation}
    x = \left( \frac{|B_1|}{|A|}, \frac{|B_2|}{|A|}, \cdots, \frac{|B_n|}{|A|} \right).
  \end{equation}
  Here, $B_i$ is the matrix obtained by replacing the $i$-th column of $A$
  with the column vector $b$.
\end{theorem}

\begin{proof}
  Hint: unique + $x$ is the solution.
\end{proof}


% \section{Special Determinants}

% \subsection{Claw-shaped Determinants}






