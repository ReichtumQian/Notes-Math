
\section{Theory of Divisibility}

\begin{definition}{Divisibility}{}
  Let $g(x)$ and $f(x)$ be polynomials over a number field $\mathbb{P}$.
  If there exists a polynomial $h(x)$ such that
  \begin{equation}
    f(x) = g(x) h(x),
  \end{equation}
  then $g(x)$ is said to \emph{divide $f(x)$}, denoted as $g(x)| f(x)$.
  Here, $g(x)$ is called a \emph{divisor (factor) of $f(x)$},
  and $f(x)$ is a \emph{multiple of $g(x)$}.
\end{definition}

\begin{proposition}{Division Algorithm with Remainder}{}
  For any $f(x), g(x) \in \mathbb{P}[x]$,
  there exist unique $q(x), r(x) \in \mathbb{P}[x]$ such that
  \begin{equation}
    f(x) = q(x) g(x) + r(x),
  \end{equation}
  where either $r(x) = 0$ or $\operatorname{deg}(r(x)) < \operatorname{deg} (g(x))$.
  Here $q(x)$ and $r(x)$ are said to be the \emph{quotient} and \emph{remainder} respectively.
\end{proposition}

\begin{example}{Computation of the Division Algorithm with Remainder}{}
  Given $f(x) = x^4 + 2x^3 - 5x + 7$ and $g(x) = x^2 - 3x + 1$,
  find the quotient and the remainder when $g(x)$ divides $f(x)$.
\end{example}

\begin{solution}
  Compute like the integer division, the answer is $q(x) = x^2 + 5x + 14$, $r(x)
  = 32x - 7$.
\end{solution}

\begin{example}{Application of Division Algorithm with Remainder}{}
  Prove the following propositions:
  \begin{enumerate}
  \item Let $g(x) = ax + b$, $f(x)$ be a polynomial, prove that $g(x) | f(x)$ if
    and only if $g(x) | f^2(x)$.
  \item Let $f(x), g(x), h(x)$ be three non-zero polynomials, prove that
    $h(x) | [f(x) - g(x)]$ if and only if the remainder of $f(x)$ and $g(x)$
    divided by $h(x)$ are the same.
  \end{enumerate}
\end{example}

\begin{proof}
  (1) Left-to-right is direct, so we only prove the right-to-left.
  Suppose $f(x) = q(x)g(x) + r(x)$, then we get
  \begin{equation}
    f^2(x) = p(x)g(x) + r^2(x).
  \end{equation}
  Due to $g(x) | r^2(x)$, we know that $r(x) = 0$.
\end{proof}

\section{Greatest Common Divisor}

\begin{definition}{Common Divisor}{}
  Let $f(x), g(x), d(x) \in \mathbb{P}[x]$,
  if $d(x)$ divides both $f(x)$ and $g(x)$,
  then $d(x)$ is said to be a \emph{common divisor of $f(x)$ and $g(x)$}.
\end{definition}

\begin{definition}{Greatest Common Divisor}{}
  Let $d(x)$ be a common divisor of $f(x)$ and $g(x)$,
  if $d(x)$ can not divide any other common divisor of $f(x)$ and $g(x)$,
  then it is called the \emph{greatest common divisor of $f(x)$ and $g(x)$},
  denoted as
  \begin{equation}
    d(x) = \left( f(x), g(x) \right).
  \end{equation}
\end{definition}

\begin{theorem}{Representation of Greatest Common Divisor}{}
  For any $f(x), g(x) \in \mathbb{P}[x]$,
  there exists $d(x) = (f(x), g(x))$, and there exist $u(x), v(x) \in \mathbb{P}[x]$
  such that
  \begin{equation}
    d(x) = u(x) f(x) + v(x) g(x).
  \end{equation}
\end{theorem}

\begin{proof}
  We prove the proposition using the Euclidean algorithm,
  \begin{equation}
    f(x) = q_1(x)g(x) + r_1(x), \quad g(x) = q_2(x)r_1(x) + r_2(x), \quad r_1(x) = q_3(x)r_2(x) + r_3(x),
  \end{equation}
  repeat this process, and finally get
  \begin{equation}
    r_{s-3}(x) = q_{s-1}(x)r_{s-2}(x) + r_{s-1}(x), \quad
    r_{s-2}(x) = q_s(x)r_{s-1}(x) + r_s(x), \quad
    r_{s-1}(x) = q_{s+1}(x)r_s(x) + 0.
  \end{equation}
  At this point $r_s(x)$ is the GCD of $f(x)$ and $g(x)$.
  And it can be represented as
  \begin{align}
    r_s(x) &= r_{s-2}(x) - q_s(x)r_{s-1}(x)\\
           &= r_{s-2}(x) - q_s(x)(r_{s-3}(x) - q_{s-1}(x)r_{s-2}(x))\\
           &= (r_{s-4}(x) - q_{s-2}(x)r_{s-3}(x)) - q_s(x)(r_{s-3}(x) - q_{s-1}(x)r_{s-2}(x))\\
           &= \cdots\\
           &= u(x)f(x) + v(x)g(x)
  \end{align}
  which completes the proof.
\end{proof}

\begin{note}
  The polynomial $d(x) = u(x)f(x) + v(x)g(x)$ is not necessarily the greatest
  common divisor of $f(x), g(x)$.
  The representation is also not unique.
\end{note}

\begin{example}{Find the Representation of GCD}{}
  Let $f(x) = x^4 + 3x^3 - x^2 - 4x - 3$ and $g(x) = 3x^3 + 10x^2 + 2x - 3$,
  find their GCD $d(x)$, and $u(x), v(x)$ such that
  \begin{equation}
    d(x) = u(x)f(x) + v(x) g(x).
  \end{equation}
\end{example}

\begin{solution}
  By Euclidean algorithm, we first have
  $f(x) = q_1(x)g(x) + r_1(x)$, where
  \begin{equation}
    q_1(x) = \frac{1}{3}x - \frac{1}{9}, \quad
    r_1(x) = - \frac{5}{9}x^2 - \frac{25}{9}x - \frac{10}{3}.
  \end{equation}
  Repeat the step, we have $g(x) = q_2(x)r_1(x) + r_2(x)$ where
  \begin{equation}
    q_2(x) = - \frac{27}{5}x + 9, \quad
    r_2(x) = 9x + 27.
  \end{equation}
  Similarly $r_1(x) = q_3(x)r_2(x) + r_3(x)$, where
  \begin{equation}
    q_3(x) = - \frac{5}{81}x - \frac{10}{87}, \quad
    r_3(x) = 0.
  \end{equation}
  Conclude the above equations, we know that $r_1(x) = f(x) - q_1(x)g(x)$ and
  $r_2(x) = g(x) - q_2(x)r_1(x)$, which yield
  \begin{equation}
    r_2(x) = g(x) - q_2(x)[f(x) - q_1(x)g(x)]
    = [1 - q_1(x)]g(x) - q_2(x)f(x).
  \end{equation}
  which completes the solution.
\end{solution}

\begin{proposition}{Invariance of Greatest Common Divisor}{}
  Let $f(x), g(x) \in \mathbb{P}[x]$. If $a,b,c,d \in \mathbb{P}$ satisfy $ad-bc
  \neq 0$, then
  \begin{equation}
    \left( f(x), g(x) \right) = (af(x) + bg(x), cf(x) + dg(x)).
  \end{equation}
\end{proposition}

\begin{proof}
  Denote the left hand side $d(x)$ and right hand side $c(x)$,
  it is equivalent to prove that $d(x)$ divides $c(x)$ and $c(x)$ divides $d(x)$.
  It is direct to know that $d(x)$ divides $c(x)$, since it divides $f(x),
  g(x)$, and then $af(x) + bg(x)$ and $cf(x) + dg(x)$.
  To prove that $c(x)$ divides $d(x)$, by Cramer's rule, we know that
  \begin{equation}
    f(x) = \frac{df_1(x) - bg_1(x)}{ad - bc}, \quad
    g(x) = \frac{ag_1(x) - cf_1(x)}{ad - bc},
  \end{equation}
  which yields the solution.
\end{proof}

\section{Relatively Prime Polynomials}

\begin{definition}{Relatively Prime Polynomials}{}
  If the greatest common divisor of $f(x), g(x)$ is $1$,
  then $f(x)$ and $g(x)$ are said to be \emph{relatively prime}.
\end{definition}

\begin{proposition}{Equivalent Condition for Relatively Prime Polynomials}{}
  Two polynomials $f(x), g(x)$ are relatively prime if and only if
  there exist $u(x), v(x) \in \mathbb{P}[x]$ such that
  \begin{equation}
    u(x)f(x) + v(x)g(x) = 1.
  \end{equation}
\end{proposition}

\begin{proof}
  Left-to-right is direct since $1$ is the GCD of $f(x), g(x)$.
  To prove the right-to-left direction, assume that $(f(x), g(x)) = d(x)$,
  $d(x)$ divides $u(x)f(x) + v(x)g(x)$ (which is $1$).
  Thus $d(x)$ is a contant, which yields that $f(x), g(x)$ are relatively prime.
\end{proof}

\begin{example}{Application of Relatively Prime}{}
  Prove the following propositions
  \begin{enumerate}
  \item Let $f(x), g(x)$ be relatively prime, prove that $f(x^m), g(x^m)$
    are relatively prime.
  \item Suppose $(f(x), g(x)) = d(x)$, prove that $(f(x^m), g(x^m)) = d(x^m)$.
  \end{enumerate}
\end{example}

\begin{proof}
  (1) By the equivalent condition for relatively prime, we know
  \begin{equation}
    u(x)f(x) + v(x)g(x) = 1 \Rightarrow u(x^m)f(x^m) + v(x^m)g(x^m) = 1,
  \end{equation}
  which implies that $f(x^m)$ and $g(x^m)$ are relatively prime.

  (2) Let $f(x) = d(x)m(x)$ and $g(x) = d(x)n(x)$, here $m(x)$ and $n(x)$ are
  relatively prime.
  Therefore, $f(x^k) = d(x^k)m(x^k)$ and $g(x^k) = d(x^k)n(x^k)$ are relatively prime.
\end{proof}

\section{Irreducible Polynomials and the Remainder Theorem}

\begin{definition}{Irreducible Polynomial}{}
  Let $p(x) \in \mathbb{P}[x]$ with $\operatorname{deg}(p(x)) \geq 1$.
  If $p(x)$ cannot be expressed as a product of two polynomials in
  $\mathbb{P}[x]$
  with degree less than that of $p(x)$,
  then $p(x)$ is called an \emph{irreducible polynomial in $\mathbb{P}[x]$}.
\end{definition}

\begin{example}{Application of Irreducible Polynomials}{}
  Let $f(x), g(x)$ be relatively prime, prove that
  \begin{equation}
    (f(x)g(x), f(x) + g(x)) = 1.
  \end{equation}
\end{example}

\begin{proof}
  Assume that $f(x)g(x)$ and $f(x) + g(x)$ are not relatively prime,
  there exists an irreducible polynomial $p(x)$ that either divides $f(x)$ or
  $g(x)$. This contradicts the given conditions.
\end{proof}

\begin{theorem}{Uniqueness of Factorization}{}
  If $f(x)$ is a polynomial over $\mathbb{P}$ with degree greater than or equal
  to $1$, then $f(x)$ can be uniquely decomposed into a product of finite number
  of irreducible polynomials over $\mathbb{P}$:
  \begin{equation}
    f(x) = cp_1^{r_1}(x) p_2^{r_2}(x) \cdots p_s^{r_s}(x).
  \end{equation}
\end{theorem}

\begin{theorem}{The Remainder Theorem}{}
  For $f(x) \in \mathbb{P}[x]$ and for any $\alpha \in \mathbb{P}$,
  we always have
  \begin{equation}
    f(x) = (x - \alpha)q(x) + f(\alpha).
  \end{equation}
\end{theorem}

\begin{proof}
  By the division of polynomial, we get $f(x) = (x-\alpha)q(x) + c$.
  Substituting $x = \alpha$ yields the conclusion.
\end{proof}

\begin{example}{Applications of the Remainder Theorem}{}
  Suppose $(x - 1)| f(x^m)$, prove that $(x^m - 1) | f(x^m)$.
\end{example}

\begin{proof}
  By the remainder theorem, we get $f(x^m) = (x-1)q(x) + f(1)$.
  Since $x-1$ divides $f(x^m)$, we know that $f(1) = 0$.
  Using the remainder theorem to $f(x)$ and $x-1$, then
  \begin{equation}
    f(x) = (x-1)p(x) + f(1) = (x-1)p(x).
  \end{equation}
  Substituting the original $x$ with $x^m$ yields the conclusion.
\end{proof}





