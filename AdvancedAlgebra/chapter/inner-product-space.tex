
\section{Euclidean Space}

\subsection{Euclidean Space}

\begin{definition}{Euclidean Space}{}
  Let $V$ be a linear space over $\mathbb{R}$.
  If a bilinear function $\langle \alpha, \beta \rangle$ in $V$ satisfies
  \begin{enumerate}
  \item Linearity: $\langle k_1\alpha_1 + k_2\alpha_2, \beta \rangle = k_1
    \langle \alpha_1, \beta\rangle + k_2 \langle \alpha_2, \beta \rangle$;
  \item Symmetry: $\langle \alpha, \beta \rangle = \langle \beta, \alpha \rangle$;
  \item Positive-definiteness: $\langle \alpha, \alpha \rangle \geq 0$;
    and $\langle \alpha, \alpha \rangle = 0$ if and only
    if $\alpha = 0$,
  \end{enumerate}
  then it is said the \emph{inner product in $V$}.
\end{definition}

\begin{example}{}{}
  Let $C_{[a,b]}$ denote the set containing all the continuous functions on $[a,
  b]$, prove that it is an Euclidean space with the inner product 
  \begin{equation}
    \langle f(x), g(x) \rangle := \int_a^b f(x)g(x)\mathrm{d} x.
  \end{equation}
\end{example}

\begin{definition}{Metric/Gram Matrix}
  Let $\epsilon_1,\cdots,\epsilon_n$ be a basis of an Euclidean space $V$,
  define a $n$-order square matrix
  \begin{equation}
    G = \begin{pmatrix}
      \langle\epsilon_1,\epsilon_1\rangle & \langle\epsilon_1,\epsilon_2\rangle & \cdots & \langle\epsilon_1,\epsilon_n\rangle \\
      \langle\epsilon_2,\epsilon_1\rangle & \langle\epsilon_2,\epsilon_2\rangle & \cdots & \langle\epsilon_2,\epsilon_n\rangle \\
      \vdots & \vdots & \ddots & \vdots \\
      \langle\epsilon_n,\epsilon_1\rangle & \langle\epsilon_n,\epsilon_2\rangle & \cdots & \langle\epsilon_n,\epsilon_n\rangle
    \end{pmatrix}
  \end{equation}
  as its \emph{metric matrix}.
\end{definition}

\begin{proposition}{Representation of Inner Product}{}
  If the coordinate vectors of vectors $\alpha, \beta$ with respect to the basis
  $(\epsilon_1,\cdots,\epsilon_n)$ are $X, Y$ respectively.
  Then 
  \begin{equation}
    \langle \alpha, \beta \rangle = X^TGY,
  \end{equation}
  where $G$ is the metric matrix of $(\epsilon_1,\cdots,\epsilon_n)$.
\end{proposition}

\begin{proof}
  See bilinear functions.
\end{proof}

\begin{example}{Find the Metric Matrix and Inner Product}{}
  Given a basis of Euclidean space $V$:
  \begin{equation}
    \epsilon_1=e_1+e_2,\epsilon_2=e_1+e_3,\epsilon_3=e_4-e_1,\epsilon_4=e_1-e_2-e_3+e_4,
  \end{equation}
  where $\{e_i\}$ are standard basis vectors.
  The coordinates of two vectors $\alpha, \beta$ under
  $(\epsilon_1,\cdots,\epsilon_4)$ are $(1,2,3,4)^T$ and $(2,0,1,0)^T$ respectively.
  Find (1) The metric matrix under $(\epsilon_1,\cdots,\epsilon_4)$
  (2) $\langle \alpha, \beta \rangle$.
\end{example}

\begin{solution}
  (1)
\end{solution}

\subsection{Orthogonality}

\begin{definition}{Orthogonality}{}
  Let $\alpha, \beta$ be two vectors in an Euclidean space $V$,
  if
  \begin{equation}
    \langle \alpha, \beta \rangle = 0,
  \end{equation}
  then we say that $\alpha, \beta$ are \emph{orthogonal}.
\end{definition}

\begin{proposition}{Orthogonality Implies Linearly Independent}{}
  Let $\alpha_1,\cdots,\alpha_n$ be a set of vectors in an Euclidean space $V$,
  if they are pairwise orthogonal, then they are linearly independent.
\end{proposition}

\begin{proof}
  Without loss of generality,
  assume that $k_1 \alpha_1 + \cdots + k_n\alpha_n = 0$.
  Then taking the inner product of both sides with $\alpha_i$ yields
  \begin{equation}
    k_i \langle \alpha_i, \alpha_i \rangle = 0.
  \end{equation}
  Therefore for any $i = 1,\cdots,n$, we have $k_i = 0$.
\end{proof}




\section{Unitary Space}









