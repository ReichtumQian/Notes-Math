
\section{Rank of Matrices}

\begin{definition}{Rank of a Matrix}{}
  The \emph{rank} of a matrix is equal to:
  (1) The rank of row-vector group of the matrix;
  (2) The order of the largest non-zero submatrix of the matrix.
\end{definition}

\begin{proposition}{Elementary Transformation Standard Form}{}
  Let $A$ be an $s \times n$ matrix with rank $r$.
  Then there exist an $s$-order invertible matrix $P$
  and an $n$-order invertible matrix $Q$ such that
  \begin{equation}
    A = P \left[
      \begin{array}{cc}
        I_r & O\\
         O  & O
      \end{array}
    \right] Q.
  \end{equation}
\end{proposition}

\begin{corollary}{Standard Forms for Column-full-rank and Row-full-rank Matrices}{}
  For an $m \times n$ matrix $A$,
  $A$ is column-full-rank if and only if there exists an $m$-order invertible
  matrix $P$ such that
  \begin{equation}
    A = P \left[
      \begin{array}{c}
        I_n\\
        O
      \end{array}
    \right].
  \end{equation}
  Similarly, $A$ is row-full-rank if and only if there exists an $n$-order invertible
  matrix $Q$ such that
  \begin{equation}
    A = \left[
      \begin{array}{cc}
        I_m&O
      \end{array}
    \right]Q.
  \end{equation}
\end{corollary}


\section{Inverse and Adjugate Matrices}

\subsection{Inverse of Matrices}

\begin{definition}{Inverse of a Matrix}{}
  If two matrices $A, B$ satisfy $AB = BA = I$,
  then $B$ is called the \emph{inverse} of $A$, denoted as $A^{-1}$.
\end{definition}

\begin{proposition}{Basic Properties of Matrix Inverses}{}
  Consider two matrices $A, B$ and a real number $k$, then
  \begin{enumerate}
  \item $(kA)^{-1} = k^{-1}A^{-1}$;
  \item $(AB)^{-1} = B^{-1} A^{-1}$;
  \item $(A^T)^{-1} = (A^{-1})^T$;
  \item $|A^{-1}| = |A|^{-1}$.
  \end{enumerate}
\end{proposition}

\begin{proof}
  (1) is obvious.
  (2) by $(AB)^{-1}AB = I$ then $(AB)^{-1} = B^{-1}A^{-1}$.
\end{proof}

\begin{proposition}{Necessary and Sufficient Condition for Invertibility}{}
  A square matrix $A$ is invertible if and only if:
  (1) $|A| \neq 0$;
  (2) or $A$ can be expressed as a product of elementary matrices.
\end{proposition}

\begin{proposition}{Finding the Inverse Using Elementary Row Operations}{}
  If a matrix $A$ is invertible, then we can apply a sequence of
  elementary row operations such that
  \begin{equation}
    (A, I) \rightarrow (I, A^{-1}).
  \end{equation}
  Similarly, for any matrix $B$ with compatible dimensions,
  we can apply elementary row operations such that $(A, B)
  \rightarrow (I, A^{-1}B)$.
\end{proposition}

\subsection{Adjugate Matrix}


\begin{definition}{Adjugate Matrix}{}
  Let $A = (a_{ij})$ be an $n \times n$ matrix, and let $A_{ij}$ denote
  the cofactor of the entry $a_{ij}$.
  The \emph{adjugate matrix} of $A$, denoted as $A^{\ast}$,
  is defined as the transpose of the cofactor matrix of $A$, i.e.,
  \begin{equation}
    A^*=\begin{bmatrix}A_{11}&A_{21}&\cdots&A_{n1}\\A_{12}&A_{22}&\cdots&A_{n2}\\\vdots&\vdots&\ddots&\vdots\\A_{1n}&A_{2n}&\cdots&A_{nn}\end{bmatrix}.
  \end{equation}
\end{definition}

\begin{proposition}{Finding the Inverse Using the Adjugate Matrix}{}
  Let $A$ be an $n \times n$ matrix, and let $A^{\ast}$ denote its adjugate matrix.
  Then
  \begin{equation}
    AA^{\ast} = |A| I.
  \end{equation}
\end{proposition}

\begin{proof}
  Define $B = AA^{\ast}$.
  The $(i,j)$-entry of $B$, denoted as $b_{ij}$, is given by 
  \begin{equation}
    b_{ij} = \sum\limits_{k = 1}^n a_{ik} A_{jk} =
    \begin{cases}
      |A|, & i = j;\\
      0, & i \neq j.
    \end{cases}
  \end{equation}
  This result follows directly from the expansion of the determinant along the
  $j$-th column.
  Therefore, $AA^{\ast} = |A|I$.
\end{proof}




