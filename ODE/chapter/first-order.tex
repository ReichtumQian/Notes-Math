
\section{Separable Equation and its Generalization}

\subsection{Separable Equation}

\begin{definition}{Separable Equation}{}
  A \emph{separable equation} is an ODE of the form
  \begin{equation}
    \frac{\mathrm{d} y}{\mathrm{d} x} = f(x)\varphi(y),
  \end{equation}
  where $f$ and $\varphi$ are continuous functions.
\end{definition}

\begin{proposition}{Solution for Separable Equations}{}
  If $\varphi(y) \neq 0$,
  then the solution for separable equations satisfies
  \begin{equation}
    \int \frac{\mathrm{d} y}{\varphi(y)} = \int f(x)\mathrm{d} x + c.
  \end{equation}
\end{proposition}

\begin{proof}
  Omitted since it is direct to see.
\end{proof}

\begin{example}{Applications of Separable Equations}{}
  Solve the following ODEs
  \begin{equation}
    (1) \frac{\mathrm{d}y}{\mathrm{d} x} = P(x)y, \quad
    (2) 2yy^{\prime} - y^2 - 2 = 0, y(0) = 1.
  \end{equation}
\end{example}

\begin{solution}
  (1) $y = c e^{\int P(x)\mathrm{d} x}$.
  (2) $y = \sqrt{3e^x - 2}$.
\end{solution}

\subsection{Homogeneous Equation}

\begin{definition}{Homogeneous Equation}{}
  A \emph{homogeneous equation} is an ODE of the form
  \begin{equation}
    \frac{\mathrm{d} y}{\mathrm{d}x} = g \left( \frac{y}{x} \right),
  \end{equation}
  where $g$ is continuous.
\end{definition}

\begin{proposition}{Homogeneous Equation to Separable Equation}{}
  Any homogeneous equation can be transformed into a separable equation
  by substituting $y(x) = u(x)x$, that is
  \begin{equation}
    \frac{\mathrm{d} y}{\mathrm{d} x} = g \left( \frac{y}{x} \right)
    \Rightarrow
    u + x \frac{\mathrm{d}u}{\mathrm{d} x} = g(u).
  \end{equation}
\end{proposition}

\begin{example}{Solve Homogeneous Equations}{}
  Solve the following ODEs
  \begin{equation}
    xy^{\prime} + y(\ln x - \ln y) = 0, y(1) = e^3.
  \end{equation}
\end{example}

\begin{solution}
  $y^{\prime} = - \frac{y}{x} \ln \frac{x}{y} = \frac{y}{x} \ln \frac{y}{x}$, so it is a homogeneous equation.
  Let $y(x) = x u(x)$, then the ODE can be transformed into
  \begin{equation}
    \frac{\mathrm{d} u}{u(\ln u - 1)} = \frac{\mathrm{d} x}{x}
    \Rightarrow
    u = e^{1 \pm e^c x}.
  \end{equation}
  When $x = 1$, $y(1) = e^3$, so $u(1) = e^3$. So $e^c = 2$, $u = e^{1\pm 2x}$
  and $y = x e^{1 \pm 2x}$.
\end{solution}

% \begin{proposition}{Generalization of Homogeneous Equation}{}
%   The generalization of homogeneous equation
%   \begin{equation}
%     \frac{\mathrm{d} y}{\mathrm{d} x} = \frac{a_1 x + b_1 y + c_1}{a_2x + b_2 y + c_2},
%   \end{equation}
%   can be solved using the substitution:
%   \begin{equation}
%     u = 
%   \end{equation}
% \end{proposition}

\section{First-Order Linear ODEs}

\subsection{Variation of Parameters}

\begin{definition}{First-Order Linear ODE}{}
  A \emph{first-order linear ODE} is an ODE of the form
  \begin{equation}
    \frac{\mathrm{d} y}{\mathrm{d} x} = P(x) y + Q(x),
  \end{equation}
  where $P(x), Q(x)$ are continuous functions.
\end{definition}

\begin{proposition}{Variation of Parameters}{}
  The general solutions for $y^{\prime} = P(x)y$ and $y^{\prime}(x) = P(x)y + Q(x)$ are
  \begin{equation}
    y = C e^{\int P(x)\mathrm{d} x}, \quad
    y = \left( \int Q(x) e^{-\int P(x)\mathrm{d} x}\mathrm{d} x + C \right) e^{\int P(x)\mathrm{d} x}
  \end{equation}
  respectively.
\end{proposition}

\begin{proof}
  The homogeneous case is direct. To solve the non-homogeneous case,
  consider $y = C(x)e^{\int P(x) \mathrm{d} x}$, then
  \begin{equation}
    y^{\prime}(x) = C^{\prime}(x) e^{\int P(x)\mathrm{d} x} + C(x) P(x) e^{\int P(x)\mathrm{d} x}.
  \end{equation}
  Meanwhile, the given condition indicates
  \begin{equation}
    y^{\prime}(x) = C(x)P(x)e^{\int P(x)\mathrm{d} x} + Q(x).
  \end{equation}
  Comparing the above two equations, we get $Q(x) = C^{\prime}(x) e^{\int P(x)\mathrm{d} x}$,
  which implies
  \begin{equation}
    C(x) = \int Q(x) e^{-\int P(x)\mathrm{d} x} \mathrm{d} x + C.
  \end{equation}
\end{proof}

\begin{example}{Applications of Variation of Parameters}{}
  Find the solution for
  \begin{equation}
    (1) y^{\prime} = \frac{y + \sin^3 x}{\sin x \cos x}, \quad
    (2) y^{\prime} = \frac{y}{x + y^3}.
  \end{equation}
\end{example}

\begin{solution}
  (1) Apply the formula.

  (2) Take reciprocal on both sides, then $x^{\prime} = \frac{x}{y} + y^2$. Then apply the formula.
\end{solution}

\subsection{Bernoulli's Equation}

\begin{proposition}{Bernoulli's Equation}{}
  The solution for $y^{\prime} = P(x) y + Q(x) y^n$ is the solution of
  \begin{equation}
    \frac{\mathrm{d} z}{\mathrm{d} x} = (1-n)P(x)z + (1-n)Q(x), \quad y = z^{n-1},
  \end{equation}
  or $y = 0$.
\end{proposition}

\begin{proof}
  Multiplying both sides by $y^{-n}$ yields
  \begin{equation}
    y^{-n}y^{\prime} = P(x)y^{1-n} + Q(x).
  \end{equation}
  Let $z(x) = y^{1-n}(x)$, then $z^{\prime}(x) = (1-n)y^{-n}y^{\prime}$. Then the equation is equivalent to
  \begin{equation}
    z^{\prime}(x) = (1-n)P(x)z(x) + Q(x).
  \end{equation}
\end{proof}

\begin{example}{Practice on Bernoulli's Equation}{}
  Find the solution for
  \begin{equation}
    y^{\prime} = \frac{x^4 + y^3}{xy^2}.
  \end{equation}
\end{example}

\begin{solution}
  $y^{\prime} = \frac{y}{x} + \frac{x^3}{y^2}$, so it is a Bernoulli's equation with $n = -2$.
\end{solution}

\section{Exact Equation}

\begin{definition}{Exact Equation}{}
  A \emph{exact equation} is an ODE of the form
  \begin{equation}
    M(x,y)\mathrm{d} x + N(x,y)\mathrm{d} y = 0,
  \end{equation}
  where $M_y = N_x$.
\end{definition}

\section{First-Order Implicit Equation}

\begin{definition}{First-Order Implicit Equation}{}
  A \emph{first-order implicit equation} is an ODE of the form
  \begin{equation}
    F(x, y, y^{\prime}) = 0.
  \end{equation}
\end{definition}






