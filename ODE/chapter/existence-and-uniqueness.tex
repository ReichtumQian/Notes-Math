
\section{Picard's Existence and Uniqueness Theorem}


\begin{lemma}{Local Solution of IVP}{}
  A function $y(x)$ follows
  $y(x) = y_0 + \int_{x_0}^xf(t,y(t))\mathrm{d} t$ ($x \in [x_0, x_0 + h]$)
  if and only if it follows
  \begin{equation}
    \frac{\mathrm{d} y}{\mathrm{d} x} = f(x,y), \quad x \in [x_0, x_0 + h],
    \quad \quad y(x_0) = y_0.
  \end{equation}
\end{lemma}

\begin{proof}
  By taking derivative or integral, it is obvious.
\end{proof}

\begin{definition}{Picard Sequence}{}
  A sequence of functions $\varphi_0,\cdots,\varphi_n(x)$ satisfying
  \begin{equation}
    \varphi_n(x) = y_0 + \int_{x_0}^x f(t, \varphi_{n-1}(t))\mathrm{d} t, \quad
    x \in [x_0, x_0 + h], n \geq 1,
  \end{equation}
  with $\varphi_0(x) = y_0$ is called the \emph{Picard sequence}.
\end{definition}

\begin{definition}{Lipschitz Continuous}{}
  Let $f(x,y)$ be a function defined on
  the rectangular domain $R = [x_0-a,x_0+a] \times [y_0-b, y_0 + b]$.
  If there exists $L > 0$ such that
  \begin{equation}
    \left| f(x,y_1) - f(x,y_2)  \right| \leq L |y_1 - y_2|, \quad
    \forall (x, y_1), (x, y_2) \in R.
  \end{equation}
  Then $f(x,y)$ is said to be \emph{Lipschitz continuous with respect to $y$ in $R$}.
\end{definition}

\begin{lemma}{Properties of Picard Sequence}{}
  If $f(x,y)$ is Lipschitz continuous with respect to $y$.
  Then the Picard sequence $\varphi_0(x),\cdots,\varphi_n(x)$ is
  uniformly bounded and uniformly convergent on $[x_0, x_0 + h]$.
\end{lemma}

\begin{proof}
  Uniformly bounded: By induction, if there exists $b \geq Mh$ such that $|\varphi_n(x) - y_0| \leq b$,
  where $M = \max _{x \in [x_0, x_0 + h]}|f(x)|$. Then
  \begin{equation}
    |\varphi_{n+1}(x) - y_0| \leq \int_{x_0}^x |f(t, \varphi_n(t))|\mathrm{d} t
    \leq M h \leq b.
  \end{equation}

  Uniformly convergent: Consider function series
  $\varphi_0(x) + \sum\limits_{n = 1}^{\infty}[\varphi_n(x) - \varphi_{n-1}(x)]$.
  We have $|\varphi_1(x) - \varphi_0(x)| \leq M(x-x_0)$, by induction we can prove
  \begin{equation}
    |\varphi_n(x) - \varphi_{n-1}(x)| \leq \frac{ML^{n-1}}{n!}(x-x_0)^n \leq \frac{ML^{n-1}}{n!}h^n.
  \end{equation}
  Then by M-test we know the series is uniformly convergent.
\end{proof}


\begin{theorem}{Picard's Existence and Uniqueness Theorem}{}
  If $f(x,y)$ is Lipschitz continuous in
  the rectangular domain $R = [x_0-a,x_0+a] \times [y_0-b, y_0 + b]$.
  Then there exists a unique solution $y = \varphi(x)$ satisfying
  \begin{equation}
    \frac{\mathrm{d} y}{\mathrm{d} x} = f(x,y), \quad x\in (x_0 -h, x_0 + h),
    \quad \quad \varphi(x_0) = y_0
  \end{equation}
  where $h = \min \left( a, \frac{b}{M} \right)$, $M := \max \limits_{(x,y) \in R}|f(x,y)|$.
\end{theorem}

\begin{proof}
  (1) Existence: By taking limit of the Picard sequence
  and applying the limit properties of uniform convergence,
  it is obvious that the limit function satisfies
  $\varphi(x) = y_0 + \int_{x_0}^x f(t,\varphi(t))\mathrm{d} t$,
  so it is also the solution of IVP.

  (2) Uniqueness: If $\psi(x)$ is another continuous solution of IVP,
  then by induction we also have
  \begin{equation}
    |\varphi_n(x) - \psi(x)| \leq \frac{M L^n}{(n+1)!}(x-x_0)^{n+1} \rightarrow 0,
  \end{equation}
  then the limit is unique.
\end{proof}

\section{Extension of Solutions}

\begin{theorem}{Extension of Solutions}{}
  Let $f(x,y)$ be a continuous function in a bounded region $G$,
  and is Lipschitz continuous with respect to $y$.
  Then the solution of 
  \begin{equation}
    \frac{\mathrm{d}y}{\mathrm{d} x} = f(x,y),
  \end{equation}
  can be extended from any $(x_0, y_0) \in G$ to be arbitrarily close to the
  boundary of $G$.
\end{theorem}



