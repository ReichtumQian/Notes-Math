
\section{Solution Structure of Higher-Order Linear ODEs}

\subsection{Linear Dependence and Wronsky Determinant}

\begin{definition}{Linear Independence}{}
  Let $x_1(t),\cdots,x_k(t)$ be functions defined on $t \in [a, b]$,
  if there exist $c_1,c_2,\cdots,c_k$ that are not all zero such that
  \begin{equation}
    c_1x_1(t) + \cdots + c_kx_k(t) \equiv 0, \quad t \in [a, b].
  \end{equation}
  Then $x_1(t), \cdots, x_k(t)$ are said to be \emph{linearly dependent}.
  Otherwise, they are said to be \emph{linearly independent}.
\end{definition}

\begin{definition}{Wronsky Determinant}{}
  Let $x_1(t),\cdots,x_k(t)$ be functions that are $k-1$ times differentiable on $[a, b]$,
  then the \emph{Wronsky determinant} is
  \begin{equation}
    W =
    \begin{vmatrix}
      x_1(t) & x_2(t) & \cdots & x_k(t)\\
      x^{\prime}_1(t) & x^{\prime}_2(t) & \cdots & x^{\prime}_k(t)\\
      \vdots & \vdots & & \vdots\\
      x^{(k-1)}_1(t) & x^{(k-1)}_2(t) & \cdots & x^{(k-1)}_k(t)
    \end{vmatrix}.
  \end{equation}
\end{definition}

\begin{proposition}{Wronsky Determinant to Determine Linear Dependence}{}
  Given $x_1(t),\cdots,x_k(t)$, if their Wronsky determinant is zero,
  then they are linearly dependent. Otherwise, they are linearly independent.
\end{proposition}

\subsection{Solution Structure of $x^{(n)}(t) + a_1(t)x^{(n-1)}(t) + \cdots + a_n(t)x(t) = 0$}

\begin{proposition}{Solution Structure of Homogeneous Higher-Order Linear ODEs}{}
  If $x_1(t), \cdots, x_k(t)$ are $k$ linearly independent solutions of
  $x^{(n)}(t) + a_1(t)x^{(n-1)}(t) + \cdots + a_{n-1}(t)x^{\prime}(t) + a_n(t)x(t) = 0$.
  Then the general solution is
  \begin{equation}
    c_1x_1(t) + \cdots + c_kx_k(t), \quad c_1,c_2,\cdots,c_k \in \mathbb{R}.
  \end{equation}
\end{proposition}

\subsection{Solution Structure of $x^{(n)}(t) + a_1(t)x^{(n-1)}(t) + \cdots + a_n(t)x(t) = f(t)$}

\begin{proposition}{Variation of Parameters}{}
  Let $x_1(t), \cdots, x_n(t)$ be $n$ linearly independent solutions of homogeneous ODE.
  Then the solution of the inhomogeneous ODE
  $x^{(n)}(t) + a_1(t)x^{(n-1)}(t) + \cdots + a_{n-1}(t)x^{\prime}(t) + a_n(t)x(t) = f(t)$
  is
  \begin{equation}
    x(t) = c_1(t)x_1(t) + \cdots + c_n(t)x_n(t),
  \end{equation}
  where $c_1(t),\cdots,c_n(t)$ follows
  \begin{align}
    & c^{\prime}_1(t)x_1(t) + \cdots + c_n^{\prime}(t)x_n(t) = 0\\
    & c^{\prime}_1(t)x_1^{\prime}(t) + \cdots + c_n^{\prime}(t)x_n^{\prime}(t) = 0\\
    & \quad \quad \quad \vdots\\
    & c^{\prime}_1(t)x_1^{(n-1)}(t) + \cdots +  c_n^{\prime}(t)x_n^{(n-1)}(t) = f(t).
  \end{align}
\end{proposition}

\begin{proof}
  Let us validate the solution. With the given conditions
  \begin{align}
    & x^{\prime}(t) = c_1(t)x^{\prime}_1(t) + \cdots + c_n(t)x^{\prime}_n(t)\\
    & x^{\prime\prime}(t) = c_1(t)x^{\prime\prime}_1(t) + \cdots + c_n(t)x_n^{\prime\prime}(t)\\
    & \quad \quad \quad \vdots\\
    & x^{(n-1)}(t) = c_1(t)x_1^{(n-1)}(t) + \cdots + c_n(t) x_n^{(n-1)}(t)\\
    & x^{(n)}(t) = c_1(t)x_1^{(n)}(t) + \cdots + c_n(t) x_n^{(n)}(t) + f(t).
  \end{align}
  Then consider
  \begin{align}
    &x^{(n)}(t) + a_1(t)x^{(n-1)}(t) + \cdots + a_{n-1}(t)x^{\prime}(t) + a_n(t)x(t)\\
    =& f(t) + c_1(t)[x_1^{(n)}(t) + a_1(t) x_1^{(n-1)}(t) + \cdots] + \cdots + c_n(t)[x_n^{(n)}(t) + a_1(t)x_n^{(n-1)}(t) + \cdots]\\
    =& f(t).
  \end{align}
  Then the $x(t)$ is a solution of the inhomogeneous ODE.
\end{proof}

\section{Solving Higher-Order Constant-Coefficient Linear ODEs}

\subsection{Solving $x^{(n)}(t) + a_1x^{(n-1)}(t) + \cdots + a_nx(t) = 0$}

\begin{proposition}{Method of Characteristic Equation}{}
  The solution of $x^{(n)}(t) + a_1x^{(n-1)}(t) + \cdots + a_nx(t) = 0$ is determined by
  the roots of
  \begin{equation}
    \lambda^n + a_1 \lambda^{n-1} + \cdots + a_{n-1} \lambda + a_n = 0.
  \end{equation}
  Specifically, the fundamental solution system is composed of
  \begin{enumerate}
  \item For each \textbf{distinct real root} $\lambda$, include a term $c e^{\lambda t}$;
  \item For each \textbf{repeated real root} $\lambda$ of multiplicity $k$,
    include $c_1 e^{\lambda t}, c_2 t e^{\lambda t},\cdots, c_k t^{k-1} e^{\lambda t}$;
  \item For each pair of \textbf{complex conjugate roots} $\alpha \pm \beta i$,
    include terms $c_1 e^{\alpha t} \cos(\beta t)$ and $c_2 e^{\alpha t} \sin (\beta t)$;
  \item For \textbf{repeated complex roots}, analogously multiply $t^m$ to the base terms.
  \end{enumerate}
\end{proposition}

\subsection{Solving $x^{(n)}(t) + a_1x^{(n-1)}(t) + \cdots + a_nx(t) = f(t)$}

\begin{proposition}{Method of Undetermined Coefficients}{}
  If the $n$-order ODE is given by
  $x^{(n)}(t) + \cdots + a_nx(t) = (b_0t^m + \cdots + b_{m-1}t + b_m)e^{\lambda t}$.
  Then a particular solution $x_p(t)$ follows
  \begin{equation}
    x_p(t) = t^k e^{\lambda t} (c_0 t^m + c_1 t^{m-1} + \cdots + c_{m-1}t + c_m),
  \end{equation}
  where $k$ is the multiplicity of $\lambda$ as a root of the characteristic equation,
  and $c_0,\cdots,c_m$ are determined by substituting $x_p(t)$ into the ODE.
\end{proposition}

\begin{note}
  Method of undertermined coefficients only works for some special right-hand side functions,
  but it offers higher computation efficiency.
  Method of variation parameters offer a more general method, but it requires a ton of computation.
\end{note}





