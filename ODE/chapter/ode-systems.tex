
\section{Solution Structure of Systems of ODEs}

\subsection{Linear Dependence and Wronsky Determinant}

\begin{definition}{Linear Dependence}{}
  Let $\mathbf{x}_1(t),\cdots, \mathbf{x}_k(t)$ be vector functions $\mathbf{x}_i(t): [a,b] \rightarrow \mathbb{R}^n$.
  If there exists $c_1,\cdots,c_k$ that are not all zero such that
  \begin{equation}
    c_1\mathbf{x}_1(t) + \cdots + c_k\mathbf{x}_k(t) = 0, \quad t \in [a, b].
  \end{equation}
  Then $\mathbf{x}_1(t),\cdots,\mathbf{x}_k(t)$ are said to be \emph{linearly dependent}.
  Otherwise, they are said to be \emph{linearly independent}.
\end{definition}

\begin{definition}{Wronsky Determinant}{}
  Let $\mathbf{x}_1(t),\cdots,\mathbf{x}_n(t)$ be $n$ vector functions $\mathbf{x}_i(t): [a,b] \rightarrow \mathbb{R}^n$.
  Then the \emph{Wronsky determinant} is
  \begin{equation}
    W(t) :=
    \begin{vmatrix}
      x_{11}(t) & x_{12}(t) & \cdots & x_{1n}(t)\\
      x_{21}(t) & x_{22}(t) & \cdots & x_{2n}(t)\\
      \vdots & \vdots & & \vdots\\
      x_{n1}(t) & x_{n2}(t) & \cdots & x_{nn}(t)
    \end{vmatrix},
  \end{equation}
  where $x_{ij}$ is the $j$-th component of $\mathbf{x}_i$.
\end{definition}

\subsection{Solution Structure of $\mathbf{x}^{\prime}(t) = A(t) \mathbf{x}(t)$}

\begin{definition}{Fundamental System of Solutions}{}
  Let $\mathbf{x}_1(t),\cdots,\mathbf{x}_n(t)$ be linearly independent vector functions.
  If for any solution $\mathbf{x}(t)$ of $\mathbf{x}^{\prime}(t) = A(t) \mathbf{x}(t)$,
  it can be represented as
  \begin{equation}
    \mathbf{x}(t) = c_1\mathbf{x}_1(t) + \cdots + c_n\mathbf{x}_n(t),
  \end{equation}
  then $\mathbf{x}_1(t),\cdots,\mathbf{x}_n(t)$ are said to be a \emph{fundamental system of solutions}.
\end{definition}

\begin{definition}{Fundamental Solution Matrix}{}
  Let $\mathbf{x}_1(t),\cdots,\mathbf{x}_n(t)$ be the fundamental system of solutions of
  $\mathbf{x}^{\prime}(t) = A(t) \mathbf{x}(t)$,
  then the \emph{fundamental solution matrix} is
  \begin{equation}
    \Phi(t) := 
    \begin{bmatrix}
      x_{11}(t) & x_{12}(t) & \cdots & x_{1n}(t)\\
      x_{21}(t) & x_{22}(t) & \cdots & x_{2n}(t)\\
      \vdots & \vdots & & \vdots\\
      x_{n1}(t) & x_{n2}(t) & \cdots & x_{nn}(t)
    \end{bmatrix},
  \end{equation}
  where $x_{ij}$ is the $j$-th component of $\mathbf{x}_i$.
\end{definition}

\begin{proposition}{Matrix Form of System of ODEs}{}
  Let $\Phi(t)$ be the fundamental solution matrix of $\mathbf{x}^{\prime}(t) = A(t)\mathbf{x}(t)$,
  then
  \begin{equation}
    \Phi^{\prime}(t) = A(t) \Phi(t).
  \end{equation}
\end{proposition}

\begin{proposition}{Solution Structure of $\mathbf{x}^{\prime}(t) = A(t) \mathbf{x}(t)$}{}
  Let $\mathbf{x}_1(t),\cdots,\mathbf{x}_n(t)$ be linearly independent solutions
  of $\mathbf{x}^{\prime}(t) = A(t) \mathbf{x}(t)$.
  Then they are fundamental system of solutions if and only if
  \begin{equation}
    |\Phi(t)| \neq 0, \quad t \in [a,b].
  \end{equation}
  Furthermore, any solution of $\mathbf{x}^{\prime}(t) = A(t)\mathbf{x}(t)$ can be written into matrix forms
  \begin{equation}
    \mathbf{x}(t) = \Phi(t)C, \quad C \in \mathbb{R}^{n \times 1}.
  \end{equation}
\end{proposition}

\begin{example}{}{}
  Define
  \begin{equation}
    \Phi(t) =
      \begin{bmatrix}
        e^t & te^t\\
        0 & e^t
      \end{bmatrix},
      \quad A = 
      \begin{bmatrix}
        1 & 1\\
        0 & 1
      \end{bmatrix}.
  \end{equation}
  Prove that $\Phi(t)$ is a fundamental solution matrix of $\mathbf{x}^{\prime}(t) = A(t)\mathbf{x}(t)$.
\end{example}

\begin{proof}
  Let $\Phi(t) = [\mathbf{x}_1(t), \mathbf{x}_2(t)]$,
  it is not hard to verify that $\mathbf{x}_1(t)$ and $\mathbf{x}_2(t)$ are solutions of
  $\mathbf{x}^{\prime}(t) = A(t) \mathbf{x}(t)$.
  Meanwhile, $|\Phi(t)| \neq 0$, then $\Phi(t)$ is a fundamental solution matrix of $\mathbf{x}^{\prime}(t) = A(t)\mathbf{x}(t)$.
\end{proof}

\subsection{Solution Structure of $\mathbf{x}^{\prime}(t) = A(t) \mathbf{x}(t) + \mathbf{f}(t)$}

\begin{proposition}{Variation of Parameters}{}
  Let $\Phi(t)$ be the fundamental solution matrix of $\mathbf{x}^{\prime}(t) = A(t)\mathbf{x}(t)$,
  then the general solution of $\mathbf{x}^{\prime}(t) = A(t) \mathbf{x}(t) + \mathbf{f}(t)$ is
  \begin{equation}
    \mathbf{x}(t) = \Phi(t)C(t), \quad C(t): [a, b] \rightarrow \mathbb{R}^{n \times 1},
  \end{equation}
  where $C(t)$ follows $C^{\prime}(t) = \Phi^{-1}(t)\mathbf{f}(t)$
  and $C(t_0) = \Phi^{-1}(t_0) \mathbf{x}(t_0)$.
\end{proposition}

\begin{proof}
  Consider $\mathbf{x}(t) = \Phi(t)C(t)$, then
  \begin{equation}
    \mathbf{x}^{\prime}(t) = \Phi^{\prime}(t) C(t) + \Phi(t)C^{\prime}(t), \quad
    A(t)\mathbf{x}(t) + \mathbf{f}(t) = A(t) \Phi(t) C(t) + \mathbf{f}(t).
  \end{equation}
  Since $\Phi^{\prime}(t) = A(t)\Phi(t)$, then
  \begin{equation}
    \mathbf{x}^{\prime}(t) = A(t) \Phi(t)C(t) + \Phi(t)C^{\prime}(t).
  \end{equation}
  Compare the right-hand side and the left-hand side, we get
  \begin{equation}
    \Phi(t)C^{\prime}(t) = \mathbf{f}(t) \Rightarrow C^{\prime}(t) = \Phi^{-1}(t) \mathbf{f}(t).
  \end{equation}
\end{proof}

\begin{example}{}{}
  Find the solutions of
  \begin{equation}
    \mathbf{x}^{\prime} =
    \begin{bmatrix}
      1 & 1\\
      0 & 1
    \end{bmatrix} \mathbf{x} +
    \begin{bmatrix}
      e^{-t}\\
      0
    \end{bmatrix}, \quad
    \mathbf{x}(0) =
    \begin{bmatrix}
      -1 \\
      1
    \end{bmatrix}
  \end{equation}
\end{example}

\begin{solution}
  First, we consider the fundamental solution matrix of $\mathbf{x}^{\prime} = A\mathbf{x}$,
  \begin{equation}
    \Phi(t) =
    \begin{bmatrix}
      e^t & te^t\\
      0 & e^t
    \end{bmatrix} \Rightarrow
    \Phi^{-1}(t) =
    \begin{bmatrix}
      e^{-t} & -te^{-t}\\
      0 & e^{-t}
    \end{bmatrix}
  \end{equation}
  Since $C^{\prime}(t) = \Phi^{-1}(t)\mathbf{f}(t)$ and $C(t_0) = \Phi^{-1}(t_0)\mathbf{x}(t_0)$, i.e.,
  \begin{equation}
    C^{\prime}(t) =
    \begin{bmatrix}
      e^{-2t}\\
      0
    \end{bmatrix}, \quad
    C(0) =
    \begin{bmatrix}
      -1\\
      1
    \end{bmatrix}.
  \end{equation}
  Then $C(t) = \int_{t_0}^t \Phi^{-1}(t)\mathbf{f}(t) \mathrm{d} t + C(t_0)$,
  so $C(t) = (-\frac{1}{2}e^{-2t} - \frac{1}{2}, 1)^T$.
  Then the solution is in the form of
  \begin{equation}
    \mathbf{x}(t) = \Phi(t)C(t) = 
    \begin{bmatrix}
      (t-\frac{1}{2})e^t - \frac{1}{2}e^{-t}\\
      e^t
    \end{bmatrix}.
  \end{equation}
\end{solution}

\section{Solving Constant-Coefficient Systems of ODEs}

\begin{definition}{Matrix Exponential}{}
  Let $A$ be an $n$-order matrix, then its \emph{matrix exponential} is
  \begin{equation}
    e^A := \sum\limits_{n = 0}^{\infty} \frac{A^n}{n!}.
  \end{equation}
\end{definition}

\begin{example}{}{}
  Calculate $e^{A}$, where
  \begin{equation}
    A =
    \begin{bmatrix}
      2 & 1\\
      0 & 2
    \end{bmatrix}
  \end{equation}
\end{example}

\begin{solution}
  Consider $A = B + C$, where
  \begin{equation}
    B =
    \begin{bmatrix}
      2 & 0\\
      0 & 2
    \end{bmatrix}, \quad
    C =
    \begin{bmatrix}
      0 & 1\\
      0 & 0
    \end{bmatrix}.
  \end{equation}
  Then $e^A = e^{B + C} = e^B e^C$, where
  \begin{equation}
    e^B =
    \begin{bmatrix}
      e^2 & 0\\
      0 & e^2
    \end{bmatrix}, \quad
    e^C =
    \begin{bmatrix}
      0 & 1\\
      0 & 0
    \end{bmatrix}
    \Rightarrow
    e^A =
    \begin{bmatrix}
      0 & e^2\\
      0 & 0
    \end{bmatrix}.
  \end{equation}
\end{solution}

\subsection{Solving $\mathbf{x}^{\prime}(t) = A \mathbf{x}(t)$}

\begin{proposition}{Solution Matrix of $\mathbf{x}^{\prime}(t) = A \mathbf{x}(t)$}{}
  The solution matrix of $\mathbf{x}^{\prime}(t) = A \mathbf{x}(t)$ is
  \begin{equation}
    \mathbf{x}(t) = e^{At} c,
  \end{equation}
  where $c \in \mathbb{R}^n$.
\end{proposition}

\begin{example}{}{}
  Find the fundamental solution matrix of
  \begin{equation}
    \mathbf{x}^{\prime} =
    \begin{bmatrix}
      2 & 1\\
      0 & 2
    \end{bmatrix} \mathbf{x}(t).
  \end{equation}
\end{example}

\begin{solution}
  First consider $e^{At}$, let $A = B + C$, then
  \begin{equation}
    e^{At} = e^{Bt}e^{Ct}, \quad
    \text{where} \quad
    B =
    \begin{bmatrix}
      2 & 0\\
      0 & 2
    \end{bmatrix}, C =
    \begin{bmatrix}
      0 & 1\\
      0 & 0
    \end{bmatrix}
    \Rightarrow e^{At} =
    \begin{bmatrix}
      e^{2t} & t\\
      0 & e^{2t}
    \end{bmatrix}.
  \end{equation}
  The fundamental solution matrix $\Phi(t) = e^{At}$.
\end{solution}

\subsection{Solving $\mathbf{x}^{\prime}(t) = A \mathbf{x}(t) + \mathbf{f}(t)$}





