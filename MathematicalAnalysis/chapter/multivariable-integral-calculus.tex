
\section{Parametric Integrals}

\subsection{Proper Parametric Integrals}

\begin{definition}{Proper Parametric Integrals}{}
  Let $f(x, t)$ be a bivariate function defined on $G = \{\alpha(t) \leq x \leq
  \beta(t), a \leq t \leq b\}$,
  then
  \begin{equation}
    F(t) = \int_{\alpha(t)}^{\beta(t)} f(x, t)\mathrm{d} x, \quad t \in [a, b]
  \end{equation}
  is said to be a \emph{proper parametric integral}.
\end{definition}

\begin{proposition}{Properties of Proper Parametric Integrals over Regular Regions}{}
  If $f(x, t), f_t(x, t)$ are continuous on $[a, b] \times [c, d]$, then
  \begin{equation}
    \lim \limits _{t \rightarrow t_0} \int_a^b f(x, t)\mathrm{d} x
    = \int_a^b f(x, t_0) \mathrm{d} x,
  \end{equation}
  \begin{equation}
    \int_{c}^{d}\mathrm{d}t\int_{a}^{b}f(x,t)\mathrm{d}x=\int_{a}^{b}\mathrm{d}x\int_{c}^{d}f(x,t)\mathrm{d}x,
  \end{equation}
  \begin{equation}
    \frac{\mathrm{d}}{\mathrm{d} t} \left( \int_a^b f(x, t)\mathrm{d}x \right)
    = \int_a^b f_t(x, t)\mathrm{d} x.
  \end{equation}
\end{proposition}

\begin{example}{Use Parametric Integrals to Calculate Definite Integrals}{}
  Calculate the values of the following parametric integrals
  \begin{equation}
    \int_0^1 \frac{x^b - x^a}{\ln x}\mathrm{d} x.
  \end{equation}
\end{example}

\begin{solution}
  Notice that $\int_a^b x^y \mathrm{d} y = \frac{x^b - x^a}{\ln x}$,
  then interchanging the order of the integral yields
  \begin{equation}
    \int_0^1 \frac{x^b - x^a}{\ln x}\mathrm{d} t
    = \int_0^1 \mathrm{d} x \int_a^b x^y \mathrm{d} y
    = \int_a^b \mathrm{d} y \int_0^1 x^y \mathrm{d} x
    = \int_a^b \frac{1}{y+1}\mathrm{d} y
    = \ln \frac{1+b}{1+a}. 
  \end{equation}
\end{solution}

\begin{proposition}{Properties of Proper Parametric Integrals over Irregular Regions}{}
  If $f(x, t), f_t(x, t)$ are continuous on $[\alpha(t), \beta(t)] \times [c, d]$,
  $\alpha(t), \beta(t)$ are differentiable on $[c, d]$, then
  \begin{equation}
    \lim \limits _{t \rightarrow t_0} \int_{\alpha(t)}^{\beta(t)}f(x,t)\mathrm{d} x
    = \int_{\alpha(t_0)}^{\beta(t_0)} f(x, t_0)\mathrm{d} x,
  \end{equation}
  \begin{equation}
    \frac{\mathrm{d}}{\mathrm{d}t} \int_{\alpha(t)}^{\beta(t)} f(x, t)\mathrm{d} x
    = f(\beta(t), t)\beta^{\prime}(t) - f(\alpha(t), t) \alpha^{\prime}(t)
    + \int_{\alpha(t)}^{\beta(t)}f_t(x, t)\mathrm{d} x.
  \end{equation}
\end{proposition}

\begin{example}{Find the Derivatives of Parametric Integrals}{}
  If $f(x)$ is continuous on $[a, b]$, find the second derivative of
  \begin{equation}
    F(y) = \int_a^b f(x) |y - x| \mathrm{d} x, \quad y \in [a, b].
  \end{equation}
\end{example}

\begin{solution}
  We can write it in the following form
  \begin{equation}
    F(y) = \int_a^y f(x)(y-x)\mathrm{d} x + \int_y^b f(x)(x-y)\mathrm{d} x.
  \end{equation}
  By the derivatives of parametric integrals, we get
  \begin{equation}
    F^{\prime}(y) = \int_a^y f(x)\mathrm{d} x- \int_y^bf(x)\mathrm{d}x, \quad
    F^{\prime\prime}(y) = 2f(y).
  \end{equation}
  Therefore the answer is $F^{\prime\prime}(y) = 2f(y)$.
\end{solution}

\subsection{Improper Parametric Integrals}

\begin{definition}{Improper Parametric Integrals}{}
  Let $f(x, t)$ be a function defined on $[c, +\infty) \times I$,
  and for each $t \in I$, the improper integral
  $\int_c^{+\infty}f(x,t)\mathrm{d}t$ converges.
  Then we say
  \begin{equation}
    \Phi(t) := \int_c^{+\infty} f(x, t)\mathrm{d} x
  \end{equation}
  to be an \emph{improper parametric integral}.
\end{definition}

\begin{definition}{Uniform Convergence for Improper Parametric Integrals}{}
  Given an improper parametric integral $\int_c^{+\infty}f(x, t)\mathrm{d} x$, if
  \begin{equation}
    \forall \epsilon > 0, \exists N \in \mathbb{Z}^+, \forall M > N, \forall t \in I, \quad
    \left| \int_M^{+\infty} f(x, t)\mathrm{d} x \right| < \epsilon,
  \end{equation}
  then $\int_c^{+\infty}f(x, t)\mathrm{d} x$ is said to \emph{uniformly converge
  to $\Phi(t)$}.
\end{definition}

\begin{proposition}{Cauchy Criterion for Improper Parametric Integrals}{}
  An improper parametric integral is uniformly convergent if and only if
  \begin{equation}
    \forall \epsilon > 0, \exists M \in \mathbb{Z}^+,
    \forall A_1, A_2 > M, \forall t \in I,
    \left| \int_{A_1}^{A_2}f(x,t)\mathrm{d}x \right| < \epsilon.
  \end{equation}
\end{proposition}

\begin{example}{Determine Uniform Convergence}{}
  Prove that the following improper parametric integrals are
  uniformly convergent on $[\delta, +\infty)$, and not uniformly convergent
  on $(0, +\infty)$.
  \begin{equation}
    \Phi(x) = \int_0^{+\infty} \frac{\sin xy}{y} \mathrm{d} y.
  \end{equation}
\end{example}

\begin{proof}
  (1) We first prove that it is uniformly convergent on $[\delta, +\infty)$.
  Apply change of variables $u = xy$, and we get
  \begin{equation}
    \int_A^{+\infty} \frac{\sin xy}{y} \mathrm{d} y = \int_{Ax}^{+\infty} \frac{\sin u}{u} \mathrm{d} u,
  \end{equation}
  and according to the Dirichlet test of improper integrals, we know the integral
  $\int_0^{+\infty} \frac{\sin u}{u} \mathrm{d} u$ is convergent, i.e.,
  \begin{equation}
    \forall \epsilon > 0, \exists M > 0, \forall A > M,
    \left| \int_A^{+\infty} \frac{\sin u}{u} \mathrm{d} u \right| < \epsilon.
  \end{equation}
  Take $B > \frac{M}{\delta}$, then for all $x \in [\delta, +\infty)$,
  \begin{equation}
    \left| \int_B^{+\infty} \frac{\sin xy}{y}\mathrm{d} y \right|
    \leq \left| \int_A^{+\infty} \frac{\sin u}{u} \mathrm{d} u \right| < \epsilon,
  \end{equation}
  which satisfies the definition of uniform convergence.

  (2) Next, we prove that it is not uniformly convergent on $(0, +\infty)$.
\end{proof}

\subsection{Applications of Improper Parametric Integrals}

\begin{proposition}{}{}
  Let $p$ be a positive real number, then
  \begin{equation}
    \int_0^{+\infty} e^{-px} \frac{\sin bx - \sin ax}{x} \mathrm{d} x
    = \arctan \frac{b}{p} - \arctan \frac{a}{p}.
  \end{equation}
\end{proposition}

\section{Double Integrals}

\subsection{Concept and Computation of Double Integrals}

\begin{definition}{Double Integrals}{}
  Let $D$ be a bounded area in $\mathbb{R}^2$,
  and $T = \{\sigma_1,\cdots,\sigma_n\}$ is a partition of $D$,
  then the \emph{double integral} is defined as
  \begin{equation}
    \iint_D f(x,y)\mathrm{d}x \mathrm{d}y = \lim \limits _{\|T\|\rightarrow 0} \sum\limits_{i = 1}^n f(\xi_i, \eta_i) s(\sigma_i),
  \end{equation}
  where $(\xi_i, \eta_i) \in \sigma_i$ and $s(\sigma_i)$ is the area of $\sigma_i$.
\end{definition}

\begin{example}{Determine the Integrability}{}
  Let $D = [0, 1]\times [0,1]$. For $x \in \mathbb{R}$,
  let $q_x$ denote the denominator of $x$ when expressed as a reduced fraction.
  Define the function $f(x,y)$ on $D$ as follows
  \begin{equation}
    f(x,y)=\begin{cases}
      \frac{1}{q_x}+\frac{1}{q_y},&\mathrm{if}\left(x,y\right)\text{is a rational point},\\
      0,&\mathrm{if}\left(x,y\right)\text{is not a rational point}.
    \end{cases}
  \end{equation}
  \begin{equation}
    g(x,y)=\begin{cases}
      1,&\mathrm{if}(x,y)\text{is a rational point and }q_x=q_y,\\
      0,&\text{otherwise.}
    \end{cases}
  \end{equation}
  Prove that (1) the double integral of $f(x,y)$ over $D$ exists, but the two
  iterated integrals do not exist.
  (2) the double integral of $g(x,y)$ does not exist, but the two iterated
  integrals exist.
\end{example}

\begin{proof}
  (1) For any $\epsilon > 0$, there exist a finite number of $(x, y)$
  such that $f(x,y) > \epsilon$.
  Let $s_1,\cdots,s_m$ be these subregions,
  and let $T$ be a partition such that $\Delta s_i < \frac{\epsilon}{2m}$.
  The difference between Darboux upper sum and lower sum is
  \begin{equation}
    S(T) - s(T) = \sum\limits_{i = 1}^n \omega_i \Delta s_i
    \leq \sum\limits_{i = 1}^m 2 \cdot \frac{\epsilon}{2m} +  \epsilon \sum\limits_{i = m+1}^n \Delta s_i
    \leq \epsilon + 1 \cdot \epsilon
    \leq 2 \epsilon.
  \end{equation}
  thus the limit $\lim \limits _{\|T\| \rightarrow 0} \sum\limits_{i = 1}^n
  \omega_i \Delta s_i$ exists, and is $0$. Therefore the double integral exists.

  Now we prove that the iterated integrals do not exist.
  Fix $y$, when $x$ takes irrational points then $f(x,y) = 0$,
  while $x$ takes rational points then $f(x,y)=\frac{1}{q_y}$.
  The oscillation of $f(x,y)$ over any subregion is always larger or equal to $\frac{1}{q_y}$.
  Then the iterated integrals do not exist.
\end{proof}

\begin{proposition}{Double Integrals and Iterated Integrals}{}
  For $D = [a, b] \times [c, d]$ or $D = [a, b] \times [f_1(x), f_2(x)]$,
  we have
  \begin{equation}
    \iint_Df(x,y)\mathrm{d}x\mathrm{d}y=\int_a^b\mathrm{d}x\int_c^df(x,y)\mathrm{d}y,\quad
    \iint_Df(x,y)\mathrm{d}x\mathrm{d}y=\int_a^b\mathrm{d}x\int_{f_1(x)}^{f_2(x)}f(x,y)\mathrm{d}y
  \end{equation}
\end{proposition}

\subsection{Change of Variables in Double Integrals}

\begin{proposition}{Change of Variables in Double Integrals}{}
  Suppose the original double-integral coordinate axes $x, y$ defined on the
  area $R$.
  By the substitution $x = x(u, v), y = y(u, v)$, the region becomes $(u, v) \in
  \Omega$, then we have
  \begin{equation}
    \iint_Rf(x,y)\mathrm{d}x\mathrm{d}y=\iint_\Omega f(x(u,v),y(u,v))\left|\frac{\partial(x,y)}{\partial(u,v)}\right|\mathrm{d}u\mathrm{d}v,
  \end{equation}
  where the $|\cdot|$ indicates the absolute value, and
  \begin{equation}
    J(u,v) = \frac{\partial (x, y)}{\partial (u, v)} := \left|
      \begin{array}{cc}
        x_u&y_u\\
        x_v&y_v
      \end{array}
    \right|.
  \end{equation}
\end{proposition}

\begin{proposition}{Polar Coordinate Transformation for Double Integrals}{}
  Making the polar coordinate transformation $x = r\cos\theta, y = r \sin
  \theta$,
  we have
  \begin{equation}
    \iint_R f(x, y)\mathrm{d} x \mathrm{d} y = \iint_{\Omega}f(r \cos \theta, r \sin \theta) r \mathrm{d} r \mathrm{d} \theta.
  \end{equation}
\end{proposition}

\begin{note}
  When integrating, we usually integrate with respect to $\mathrm{d} r$ first,
  and then $\mathrm{d} \theta$.
\end{note}


\section{Triple Integrals}








