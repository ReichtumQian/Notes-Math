
\section{Double Integrals}

\subsection{Concept and Computation of Double Integrals}

\begin{definition}{Double Integrals}{}
  Let $D$ be a bounded area in $\mathbb{R}^2$,
  and $T = \{\sigma_1,\cdots,\sigma_n\}$ is a partition of $D$,
  then the \emph{double integral} is defined as
  \begin{equation}
    \iint_D f(x,y)\mathrm{d}x \mathrm{d}y = \lim \limits _{\|T\|\rightarrow 0} \sum\limits_{i = 1}^n f(\xi_i, \eta_i) s(\sigma_i),
  \end{equation}
  where $(\xi_i, \eta_i) \in \sigma_i$ and $s(\sigma_i)$ is the area of $\sigma_i$.
\end{definition}

\begin{example}{Determine the Integrability}{}
  Let $D = [0, 1]\times [0,1]$. For $x \in \mathbb{R}$,
  let $q_x$ denote the denominator of $x$ when expressed as a reduced fraction.
  Define the function $f(x,y)$ on $D$ as follows
  \begin{equation}
    f(x,y)=\begin{cases}
      \frac{1}{q_x}+\frac{1}{q_y},&\mathrm{if}\left(x,y\right)\text{is a rational point},\\
      0,&\mathrm{if}\left(x,y\right)\text{is not a rational point}.
    \end{cases}
  \end{equation}
  \begin{equation}
    g(x,y)=\begin{cases}
      1,&\mathrm{if}(x,y)\text{is a rational point and }q_x=q_y,\\
      0,&\text{otherwise.}
    \end{cases}
  \end{equation}
  Prove that (1) the double integral of $f(x,y)$ over $D$ exists, but the two
  iterated integrals do not exist.
  (2) the double integral of $g(x,y)$ does not exist, but the two iterated
  integrals exist.
\end{example}

\begin{proof}
  (1) For any $\epsilon > 0$, there exist a finite number of $(x, y)$
  such that $f(x,y) > \epsilon$.
  Let $s_1,\cdots,s_m$ be these subregions,
  and let $T$ be a partition such that $\Delta s_i < \frac{\epsilon}{2m}$.
  The difference between Darboux upper sum and lower sum is
  \begin{equation}
    S(T) - s(T) = \sum\limits_{i = 1}^n \omega_i \Delta s_i
    \leq \sum\limits_{i = 1}^m 2 \cdot \frac{\epsilon}{2m} +  \epsilon \sum\limits_{i = m+1}^n \Delta s_i
    \leq \epsilon + 1 \cdot \epsilon
    \leq 2 \epsilon.
  \end{equation}
  thus the limit $\lim \limits _{\|T\| \rightarrow 0} \sum\limits_{i = 1}^n
  \omega_i \Delta s_i$ exists, and is $0$. Therefore the double integral exists.

  Now we prove that the iterated integrals do not exist.
  Fix $y$, when $x$ takes irrational points then $f(x,y) = 0$,
  while $x$ takes rational points then $f(x,y)=\frac{1}{q_y}$.
  The oscillation of $f(x,y)$ over any subregion is always larger or equal to $\frac{1}{q_y}$.
  Then the iterated integrals do not exist.
\end{proof}

\begin{proposition}{Double Integrals and Iterated Integrals}{}
  For $D = [a, b] \times [c, d]$ or $D = [a, b] \times [f_1(x), f_2(x)]$,
  we have
  \begin{equation}
    \iint_Df(x,y)\mathrm{d}x\mathrm{d}y=\int_a^b\mathrm{d}x\int_c^df(x,y)\mathrm{d}y,\quad
    \iint_Df(x,y)\mathrm{d}x\mathrm{d}y=\int_a^b\mathrm{d}x\int_{f_1(x)}^{f_2(x)}f(x,y)\mathrm{d}y
  \end{equation}
\end{proposition}

\subsection{Change of Variables in Double Integrals}

\begin{proposition}{Change of Variables in Double Integrals}{}
  Suppose the original double-integral coordinate axes $x, y$ defined on the
  area $R$.
  By the substitution $x = x(u, v), y = y(u, v)$, the region becomes $(u, v) \in
  \Omega$, then we have
  \begin{equation}
    \iint_Rf(x,y)\mathrm{d}x\mathrm{d}y=\iint_\Omega f(x(u,v),y(u,v))\left|\frac{\partial(x,y)}{\partial(u,v)}\right|\mathrm{d}u\mathrm{d}v,
  \end{equation}
  where the $|\cdot|$ indicates the absolute value, and
  \begin{equation}
    J(u,v) = \frac{\partial (x, y)}{\partial (u, v)} := \left|
      \begin{array}{cc}
        x_u&y_u\\
        x_v&y_v
      \end{array}
    \right|.
  \end{equation}
\end{proposition}

\begin{proposition}{Polar Coordinate Transformation for Double Integrals}{}
  Making the polar coordinate transformation $x = r\cos\theta, y = r \sin
  \theta$,
  we have
  \begin{equation}
    \iint_R f(x, y)\mathrm{d} x \mathrm{d} y = \iint_{\Omega}f(r \cos \theta, r \sin \theta) r \mathrm{d} r \mathrm{d} \theta.
  \end{equation}
\end{proposition}

\begin{note}
  When integrating, we usually integrate with respect to $\mathrm{d} r$ first,
  and then $\mathrm{d} \theta$.
\end{note}


\section{Triple Integrals}








