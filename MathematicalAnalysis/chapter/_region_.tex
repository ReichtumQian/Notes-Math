\message{ !name(univariate-differential-calculus.tex)}
\message{ !name(univariate-differential-calculus.tex) !offset(1) }
ential}

\begin{definition}{Differentiability}{}
  Let $f(x)$ be a function defined on a neighborhood of $x_0 \in \mathbb{R}$,
  if the limit
  \begin{equation}
     \lim \limits _{x \rightarrow x_0} \frac{f(x) - f(x_0)}{x - x_0}
  \end{equation}
  exists, then we say it is \emph{differentiable at $x_0$}.
\end{definition}

\begin{definition}{Derivative}{}
  Let $f(x)$ be a function that is differentiable at $x_0 \in \mathbb{R}$,
  then its \emph{derivative at $x_0$} is defined by
  \begin{equation}
    f^{\prime}(x_0) := \lim \limits _{x \rightarrow x_0} \frac{f(x) - f(x_0)}{x - x_0}.
  \end{equation}
\end{definition}

\begin{definition}{Left-hand and Right-hand Derivatives}{}
  Let $f(x)$ be a function defined on a neighborhood of $x_0 \in \mathbb{R}$,
  then its \emph{left-hand derivative at $x_0$} is defined by
  \begin{equation}
    f^{\prime}(x_0^-) := \lim \limits _{x \rightarrow x_0^-} \frac{f(x) - f(x_0)}{x - x_0}.
  \end{equation}
  Similarly, the \emph{right-hand derivative at $x_0$} is defined by
  \begin{equation}
    f^{\prime}(x_0^+) := \lim \limits _{x \rightarrow x_0^+} \frac{f(x) - f(x_0)}{x - x_0}.
  \end{equation}
\end{definition}

\begin{note}
  A function $f(x)$ is differentiable at $x_0$ if and only if both its left-hand and right-hand
  derivatives at $x_0$ exist and are equal.
\end{note}

\begin{example}{Use the Definition to Find the Derivative}{}
  Let $g(0) = g^{\prime}(0) = 0$, and $f(x)$ defined as follows, find $f^{\prime}(0)$:
  \begin{equation}
    f(x)=\begin{cases}
      g(x)\sin\frac{1}{x},&x\neq 0;\\
      0,&x=0.
    \end{cases}
  \end{equation}
\end{example}

\begin{solution}
  By the definition of derivatives
  \begin{equation}
    f^{\prime}(0)=\lim_{\Delta x\to0}\frac{f(\Delta x)-f(0)}{\Delta x}=\lim_{\Delta x\to0}\frac{g(\Delta x)\sin\frac{1}{\Delta x}}{\Delta x}=\lim_{\Delta x\to0}\frac{g(\Delta x)-g(0)}{\Delta x-0}\sin\frac{1}{\Delta x}=0,
  \end{equation}
  thus we have $f^{\prime}(0) = 0$.
\end{solution}

\begin{example}{The Non-Differentiability of the Riemann Function}{}
  Prove that the Riemann function $R(x)$ is non-differentiable everywhere in
  $(0, 1)$:
  \begin{equation}
    R(x)=\begin{cases}\frac{1}{q},&x=\frac{p}{q}\text{ (reduced form)}\\0,&x\text{ is irrational or }0,1\end{cases}
\end{equation}
\end{example}

\begin{proof}
  For rational points,
  the Riemann function is not even continuous,
  so it must not be differentiable.
  For irrational points,
  let $x_0 = 0.a_1a_2 \cdots$ be an irrational point.
  If we approach $x_0$ along an irrational number sequence,
  then the limit
  \begin{equation}
    \frac{f(x_n) - f(x_0)}{x_n - x_0} = 0
  \end{equation}
  exists. When approaching along a rational number sequence,
  take $y_n = 0.a_1a_2 \cdots a_n$,
  then $R(y_n) = \frac{1}{q} \geq \frac{1}{10^n}$,
  but $|y_n - x_0| \leq \frac{1}{10^n}$,
  thus
  \begin{equation}
    \left| \frac{R(y_n) - R(x_0)}{y_n - x_0} \right| \geq 1,
  \end{equation}
  which means that $\lim \limits _{n \rightarrow \infty} \frac{R(y_n)}{y_n -
    x_0} \neq 0$.
  By Heine Theorem we know that the limit $\lim \limits _{x \rightarrow x_0}
  \frac{R(x) - R(x_0)}{x - x_0}$ does not exist,
  so $R(x)$ is also non-differentiable at irrational points.
\end{proof}

\begin{definition}{Differential}{}
  Let $f(x)$ be a function defined in a neighborhood of $x_0 \in \mathbb{R}$,
  and it is differentiable at $x_0$.
  And its \emph{differential} is
  \begin{equation}
    \mathrm{d} f = f^{\prime}(x_0) \mathrm{d} x,
  \end{equation}
  where $\mathrm{d} x = \Delta x = x - x_0$.
\end{definition}

\subsection{Commonly Used Derivatives Formulas}


\subsection{Derivatives of Implicit Functions and Inverse Functions}

\begin{proposition}{Derivative of Implicit Functions}{}
  Differentiate both sides of a equation with respect to the same independent variable,
  and the equation still holds.
\end{proposition}

\begin{proposition}{Derivative of Inverse Functions}{}
  Suppose the inverse function of a function $y = y(x)$ is $x = x(y)$,
  then we have
  
\message{ !name(univariate-differential-calculus.tex) !offset(-115) }
