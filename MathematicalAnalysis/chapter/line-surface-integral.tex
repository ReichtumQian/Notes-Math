
\section{Line Integrals}

\subsection{Line Integral with Respect to Arc Length}

\begin{definition}{Line Integrals with Respect to Arc Length}{}
  Let $\Gamma$ be a rectifiable curve in $\mathbb{R}^3$,
  and $f: \Gamma \rightarrow \mathbb{R}$ be a function.
  Divide $\Gamma$ into segments $s_1,\cdots,s_n$, with the arc-length of each
  segment being $\Delta s_i$, and $(x_i, y_i, z_i) \in s_i$.
  Then the \emph{line integral with respect to arc length} is defined as
  \begin{equation}
    \int_{\Gamma} f(x,y,z)\mathrm{d} s = \lim \limits _{\Delta s_i \rightarrow 0}
    \sum\limits_{i = 1}^n f(x_i, y_i, z_i) \Delta s_i.
  \end{equation}
\end{definition}

\begin{proposition}{Calculation of Line Integrals with Respect to Arc Length}{}
  Suppose $\Gamma$ is a smooth curve in $\mathbb{R}^3$ and can be expressed as
  $\mathbf{r} = x(t)\mathbf{i} + y(t)\mathbf{j} + z(t)\mathbf{k}$ where $t \in
  [a, b]$, then the line integral is
  \begin{equation}
    \int_{\Gamma} f(x,y,z) \mathrm{d} s = \int_a^b f(x(t), y(t), z(t))
    \sqrt{x^{\prime 2}(t) + y^{\prime 2}(t) + z^{\prime 2}(t)}\mathrm{d} t.
  \end{equation}
\end{proposition}

\begin{example}{Calculation of Line Integrals with Respect to Arc Length}{}
  Calculate the following line integrals
  \begin{enumerate}
  \item $\int_L x^2 + 2y^2 \mathrm{d} s$ where $L$ is $x^2 + y^2 = 1$.
  \item $\int_L \sqrt{2y^2 + z^2}\mathrm{d}s$ where $L$ is the intersection
    between $x^2 + y^2 + z^2 = a^2$ and $x = y$.
  \end{enumerate}
\end{example}

\begin{solution}
  (1) Let $x = \cos \theta$ and $y = \sin \theta$, then
  \begin{equation}
    \int_L x^2 + 2y^2 \mathrm{d} s = \int_0^{2\pi} 1 + \sin^2 \theta \mathrm{d} \theta
    = 2\pi + \pi = 3\pi.
  \end{equation}
  The integral of the second term is gained by the Wallis formula.

  (2) $L$ satisfies $2y^2 + z^2 = a^2$,
  then
  \begin{equation}
    \int_L \sqrt{2y^2 + z^2}\mathrm{d} s = a \cdot 2a\pi = 2\pi a^2.
  \end{equation}
\end{solution}

\begin{example}{Applications of Symmetry}{}
  Calculate the following line integrals
  \begin{enumerate}
  \item $\int_L|y|\mathrm{d} s$ where $L$ is $(x^2 + y^2)^2 = a^2(x^2 - y^2)$;
  \item $\oint _L|x|^{\frac{1}{3}}\mathrm{d} s$ where
    $L$ is $x^{\frac{2}{3}} + y^{\frac{2}{3}} = a^{\frac{2}{3}}$.
  \item $\int_L x\mathrm{d} s$, $\int_L x^2\mathrm{d} s$, and
    $\int_Lyz\mathrm{d}s$, where $L$ is the intersection of
    $x+y+z=0$ and $x^2+y^2+z^2 = 1$.
  \end{enumerate}
\end{example}

\begin{solution}
  (1) $L$ is symmetric about both the $x$-axis and $y$-axis.
  So divide $L$ into four parts. Let $L_1$ be the part in the first quadrant.
  Then $\int_L|y|\mathrm{d}s = 4\int_{L_1}y \mathrm{d} s$.
  Consider $x = r \cos \theta$ and $y = r \sin \theta$, then $L_1$ is $r = a
  \sqrt{\cos 2\theta}$ with $\theta \in [0, \frac{\pi}{4}]$.
  Here $x = a \cos \theta \sqrt{\cos 2\theta}$ and $y = a \sin \theta \sqrt{\cos
  2\theta}$. Then
  \begin{equation}
    4\int_{L_1}y\mathrm{d} s
    = 4 \int_0^{\frac{\pi}{4}} a \sqrt{\cos 2\theta} \sin \theta \sqrt{a^2 \cos 2\theta + a^2 \frac{\sin^2 2\theta}{\cos 2\theta}} \mathrm{d} \theta
    = 4 a^2 \int_0^{\frac{\pi}{4}} \sin \theta \mathrm{d} \theta = (4 - 2 \sqrt{2})a^2
  \end{equation}

  (3) Due to symmetry, we know
  \begin{equation}
    \int_L x \mathrm{d} s = \frac{1}{3} \int_{L} x+y+z\mathrm{d} s = 0.
  \end{equation}
  \begin{equation}
    \int_Lx^2\mathrm{d} s = \frac{1}{3} \int_L x^2 + y^2 + z^2 \mathrm{d} s
    = \frac{2}{3}\pi.
  \end{equation}
  With the given condition,
  $xy + yz + xz = \frac{1}{2}\left[ (x+y+z)^2 - (x^2 + y^2 + z^2) \right] = - \frac{1}{2}$.
  That is
  \begin{equation}
    \int_L yz \mathrm{d} s = \frac{1}{3} \int_L xy + yz + xz \mathrm{d} s
    = - \frac{\pi}{3}.
  \end{equation}
\end{solution}

\begin{theorem}{Mean-Value Theorem for Line Integrals}{}
  Let $f(x, y)$ be a continuous function on the smooth curve
  $L: (x(t), y(t)), t \in [a, b]$.
  Denote $\Delta L$ as the arc-length of $L$.
  Then there exists $(x_0, y_0) \in L$ such that
  \begin{equation}
    \int_L f(x, y)\mathrm{d} s = f(x_0, y_0)\Delta L.
  \end{equation}
\end{theorem}

\subsection{Line Integral of a Vector Field}

\begin{proposition}{Calculation of Line Integrals of a Vector Field}{}
  If the curve $\Gamma$ is given by $\mathbf{r}(t) = x(t)\mathbf{i} +
  y(t)\mathbf{j} + z(t)\mathbf{k}, t \in [a, b]$,
  then
  \begin{equation}
    \int_{\Gamma}P \mathrm{d} x + Q\mathrm{d} y + R\mathrm{d} z
    = \int_a^b \left[ Px^{\prime}(t) + Qy^{\prime}(t) + Rz^{\prime}(t) \right]\mathrm{d} t.
  \end{equation}
\end{proposition}

\begin{example}{Calculation of Line Integral of a Vector Field}{}
  Calculate the following line integrals:
  \begin{enumerate}
  \item $\int_L x\mathrm{d} y - y\mathrm{d} x$, where $L$ is the
    counter-clockwise curve of the triangle with vertices $O(0, 0)$,
    $A(1, 0)$, and $B(1, 2)$.
  \item $\oint_L \frac{-x \mathrm{d} x + y\mathrm{d} y}{x^2 + y^2}$,
    where $L$ is the unit circle $x^2 + y^2 = 1$ in the counter-clockwise direction.
  \item $\oint_L xyz\mathrm{d}z$, where $L$ is the circle formed by
    the intersection of $x^2 + y^2 + z^2 = 1$ and $y = z$,
    and is oriented in the $1,2,7,8$ octants.
  \end{enumerate}
\end{example}

\begin{solution}
  (1) Consider three segments $L_1: OA, L_2:AB, L_3:BO$.
  For $L_1$, taking $x$ as the parameter, then $x = x, y = 0$.
  For $L_2$, taking $y$ as the parameter, then $x = 1, y = y$.
  For $L_3$, taking $x$ as the parameter, then $y = 2x$.
  Then
  \begin{equation}
    \int_{L_1} x\mathrm{d} y - y\mathrm{d} x = \int_0^1 0 - 0\mathrm{d}x = 0.
  \end{equation}
  \begin{equation}
    \int_{L_2} x\mathrm{d} y - y\mathrm{d} x = \int_0^2 1 - 0 \mathrm{d}y = 2.
  \end{equation}
  \begin{equation}
    \int_{L_3} x\mathrm{d} y - y\mathrm{d} x = \int_1^0 2x - 2x\mathrm{d} x = 0.
  \end{equation}
  The final answer is the sum of the above terms, being $2$.

  (2) Let $x = \cos \theta$, $y = \sin \theta$, where $\theta \in [0, 2\pi]$.
  Then
  \begin{equation}
    I = \int_0^{2\pi} \cos \theta \sin \theta + \sin \theta \cos \theta \mathrm{d} \theta = 0.
  \end{equation}

  (3) Since $x^2 + 2y^2 = 1$, let $x = \cos \theta$, $y = z =
  \frac{1}{\sqrt{2}}\sin \theta$.
  Starting from the first octant, $x, y, z$ are all non-negative, so $\theta \in
  [0, 2\pi]$. Then
  \begin{equation}
    I = \int_0^{2\pi} \frac{1}{2} \cos \theta \sin^2 \theta \cdot \frac{1}{\sqrt{2}} \cos \theta \mathrm{d} \theta =
    \frac{\sqrt{2}\pi}{16}.
  \end{equation}
\end{solution}

\begin{example}{Applications of Symmetry}{}
  Calculate the following line integrals
  \begin{enumerate}
  \item 
  \end{enumerate}
\end{example}

\subsection{Green's Theorem}

\begin{theorem}{Green's Theorem}{}
  Let $\Omega \subset \mathbb{R}^2$ be a closed region bounded by a piece-wise
  smooth simple closed curve.
  $P(x,y)$ and $Q(x,y)$ are continuous in $\Omega$ and have continuous partial derivatives.
  Along the positive direction of $\Omega$ (with $\Omega$ always on the
  left-hand side), then
  \begin{equation}
    \oint_{\partial \Omega} P \mathrm{d} x + Q \mathrm{d} y
    = \iint_{\Omega} \left( \frac{\partial Q}{\partial x} - \frac{\partial P}{\partial y} \right) \mathrm{d} x \mathrm{d}y.
  \end{equation}
\end{theorem}

\begin{example}{Applications of Green's Theorem}{}
  Calculate the following line integrals
  \begin{enumerate}
  \item $\int_L [e^x \sin y - b(x+y)]\mathrm{d} x + (e^x\cos y - ax\mathrm{d}
    y)$ where $L$ is the semi-circle $y = \sqrt{2ax - x^2}$ from $(2a, 0)$ to
    $(0, 0)$.
  \item $\oint_L \frac{x \mathrm{d} y - y \mathrm{d} x}{ax^2 + by^2}$ where
    $a, b> 0$, and
    $L$ is a simple, closed, smooth curve that does not pass through the origin,
    but the region enclosed by $L$ contains the origin.
  \end{enumerate}
\end{example}

\begin{solution}
  (2) Obviously, $P$ and $Q$ have continuous partial derivatives in the region
  that does not contain the origin
  \begin{equation}
    P_y = \frac{by^2 - ax^2}{(ax^2 + by^2)^2}, \quad
    Q_x = \frac{by^2 - ax^2}{(ax^2 + by^2)^2}.
  \end{equation}
  If the region $D$ enclosed by $L$ does not contain the origin,
  then according to $P_y - Q_x = 0$, the integral is $0$.
  If $D$ contains the origin, consider removing a small region around the
  origin.
  Its boundary $L_1$ is $ax^2 + by^2 = r^2$, with a counter-clockwise direction.
  Then
  \begin{equation}
    \int_L P \mathrm{d} x + Q \mathtt{d}y
    = \int_{L_1} P\mathrm{d} x + Q\mathrm{d} y
    = \frac{1}{r^2} \int_{L_1} x\mathrm{d} y - y\mathrm{d} x.
  \end{equation}
  Using Green's theorem again, the result is
  \begin{equation}
    \frac{1}{r^2} \iint_{D_1}2\mathrm{d} x \mathrm{d} y = \frac{2\pi}{\sqrt{ab}}.
  \end{equation}
\end{solution}





\section{Surface Integrals}

\subsection{Surface Integral with Respect to Area}

\subsection{Surface Integral of a Vector Field}



