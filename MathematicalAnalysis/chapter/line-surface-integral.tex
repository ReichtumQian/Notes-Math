
\section{Line Integrals}

\subsection{Line Integral with Respect to Arc Length}

\begin{definition}{Line Integrals with Respect to Arc Length}{}
  Let $\Gamma$ be a rectifiable curve in $\mathbb{R}^3$,
  and $f: \Gamma \rightarrow \mathbb{R}$ be a function.
  Divide $\Gamma$ into segments $s_1,\cdots,s_n$, with the arc-length of each
  segment being $\Delta s_i$, and $(x_i, y_i, z_i) \in s_i$.
  Then the \emph{line integral with respect to arc length} is defined as
  \begin{equation}
    \int_{\Gamma} f(x,y,z)\mathrm{d} s = \lim \limits _{\Delta s_i \rightarrow 0}
    \sum\limits_{i = 1}^n f(x_i, y_i, z_i) \Delta s_i.
  \end{equation}
\end{definition}

\begin{proposition}{Calculation of Line Integrals with Respect to Arc Length}{}
  Suppose $\Gamma$ is a smooth curve in $\mathbb{R}^3$ and can be expressed as
  $\mathbf{r} = x(t)\mathbf{i} + y(t)\mathbf{j} + z(t)\mathbf{k}$ where $t \in
  [a, b]$, then the line integral is
  \begin{equation}
    \int_{\Gamma} f(x,y,z) \mathrm{d} s = \int_a^b f(x(t), y(t), z(t))
    \sqrt{x^{\prime 2}(t) + y^{\prime 2}(t) + z^{\prime 2}(t)}\mathrm{d} t.
  \end{equation}
\end{proposition}

\begin{example}{Calculation of Line Integrals with Respect to Arc Length}{}
  
\end{example}


\begin{theorem}{Mean-Value Theorem for Line Integrals}{}
  Let $f(x, y)$ be a continuous function on the smooth curve
  $L: (x(t), y(t)), t \in [a, b]$.
  Denote $\Delta L$ as the arc-length of $L$.
  Then there exists $(x_0, y_0) \in L$ such that
  \begin{equation}
    \int_L f(x, y)\mathrm{d} s = f(x_0, y_0)\Delta L.
  \end{equation}
\end{theorem}

\subsection{Line Integral of a Vector Field}

\begin{proposition}{Calculation of Line Integrals of a Vector Field}{}
  If the curve $\Gamma$ is given by $\mathbf{r}(t) = x(t)\mathbf{i} +
  y(t)\mathbf{j} + z(t)\mathbf{k}, t \in [a, b]$,
  then
  \begin{equation}
    \int_{\Gamma}P \mathrm{d} x + Q\mathrm{d} y + R\mathrm{d} z
    = \int_a^b \left[ Px^{\prime}(t) + Qy^{\prime}(t) + Rz^{\prime}(t) \right]\mathrm{d} t.
  \end{equation}
\end{proposition}

\subsection{Green's Theorem}

\begin{theorem}{Green's Theorem}{}
  Let $\Omega \subset \mathbb{R}^2$ be a closed region bounded by a piece-wise
  smooth simple closed curve.
  $P(x,y)$ and $Q(x,y)$ are continuous in $\Omega$ and have continuous partial derivatives.
  Along the positive direction of $\Omega$ (with $\Omega$ always on the
  left-hand side), then
  \begin{equation}
    \oint_{\partial \Omega} P \mathrm{d} x + Q \mathrm{d} y
    = \iint_{\Omega} \left( \frac{\partial Q}{\partial x} - \frac{\partial P}{\partial y} \right) \mathrm{d} x \mathrm{d}y.
  \end{equation}
\end{theorem}






\section{Surface Integrals}

\subsection{Surface Integral with Respect to Area}

\subsection{Surface Integral of a Vector Field}



