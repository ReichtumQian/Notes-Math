
\section{Uniform Convergence of Function Sequences}

\subsection{Concept of Uniform Convergence}

\begin{definition}{Uniform Convergence}{}
  Let $\{f_n\}$ be a sequence of functions on the interval $I$.
  If for any $\epsilon > 0$, there exists a positive-integer $N$
  such that for all $x \in I, n > N$,
  \begin{equation}
    |f_n(x) - f(x)| < \epsilon,
  \end{equation}
  then we say that $\{f_n\}$ \emph{converges uniformly to $f$ on $I$}.
\end{definition}

\begin{theorem}{Cauchy's Criterion for Uniform Convergence}{}
  Let $\{f_n\}$ be a sequence of functions on the interval $I$.
  $\{f_n\}$ is uniformly convergent on $I$ if and only if
  for any $\epsilon > 0$, there exists a positive integer $N$ such that
  \begin{equation}
    \forall m,n > N, \sup \limits_{x \in I} |f_m(x) - f_n(x)| < \epsilon.
  \end{equation}
\end{theorem}

\subsection{Criteria for Uniform Convergence}

\begin{theorem}{Weierstrass M-test}{}
  If there exists a positive-term sequence $\{a_n\}$
  with $\lim \limits _{n \rightarrow \infty} a_n = 0$ such that
  \begin{equation}
    |f(x) - f_n(x)| \leq a_n, \forall x \in I,
  \end{equation}
  then the sequence of functions $\{f_n(x)\}$ converges uniformly to $f(x)$ on $I$.
\end{theorem}

\begin{example}{Application of M-test}{}
  Prove that
  (1) $f_n(x) = \sqrt{x^2 + \frac{1}{n^2}}$ converges uniformly on any interval $I$;
\end{example}

\begin{theorem}{Supremum Criterion}{}
  Let $\{f_n\}$ be a sequence of functions on the interval $I$, then
  $\{f_n(x)\}$ converges uniformly to $f(x)$ if and only if
  \begin{equation}
    \lim \limits _{n \rightarrow \infty} \sup \limits_{x \in I} |f_n(x) - f(x)| = 0.
  \end{equation}
\end{theorem}

\begin{corollary}{}{}
  Let $\{f_n\}$ be a sequence of functions on the interval $I$, then
  $\{f_n(x)\}$ does not converge uniformly to $f(x)$ if and only if
  there exists a sequence $\{x_n\}$ such that
  \begin{equation}
    b_n = f_n(x_n) - f(x_n)
  \end{equation}
  does not converge to $0$.
\end{corollary}

\subsection{Uniform Convergence in Arithmetic Operations and Composite Limits}

\begin{definition}{Uniform Boundedness}{}
  Let $\{f_n\}$ be a sequence of functions on $I$.
  If there exists a positive real-number $M$ such that for all $n \in \mathbb{Z}^+$ and $x \in I$,
  \begin{equation}
    |f_n(x)| \leq M,
  \end{equation}
  then it is said to be \emph{uniformly bounded on $I$}.
\end{definition}

\section{Properties of Uniformly Convergent Function Sequences}

\subsection{Continuity of the Limit Function}

\begin{theorem}{Interchange of Limit Orders}{}
  If the sequence of functions $\{f_n(x)\}$ converges uniformly to $f(x)$
  on $(a, x_0) \cup (x_0, b)$,
  and for each $n$, the limit $\lim \limits _{x \rightarrow x_0} f_n(x) = a_n$
  exists,
  then
  \begin{equation}
    \lim \limits _{n \rightarrow \infty} \lim \limits _{x \rightarrow x_0} f_n(x) = 
    \lim \limits _{x \rightarrow x_0} \lim \limits _{n \rightarrow \infty} f_n(x).
  \end{equation}
\end{theorem}

\begin{proof}
  
\end{proof}

\begin{theorem}{Local Continuity of the Limit Function}{}
  If $\{f_n(x)\}$ converges uniformly in the neighborhood of $x_0$,
  and $f_n(x)$ is continuous at $x_0$, then the limit function $f(x)$
  is also continuous at $x_0$.
\end{theorem}

\begin{theorem}{Uniform Continuity of the Limit Function}{}
  If $\{f_n(x)\}$ converges uniformly to $f(x)$ on the interval $I$,
  and each $f_n(x)$ is uniformly continuous,
  then the limit function $f(x)$ is also uniformly continuous on $I$.
\end{theorem}

\subsection{Integrability of the Limit Function}

\begin{theorem}{Integrability of the Limit Function}{}
  If $\{f_n(x)\}$ converges uniformly to $f(x)$ on $[a, b]$,
  and each $f_n(x)$ is integrable on $[a, b]$,
  then $f(x)$ is integrable on $[a, b]$, and
  \begin{equation}
    \lim \limits _{n \rightarrow \infty} \int_a^b f_n(x) \mathrm{d} x 
    = \int_a^b f(x)\mathrm{d} x = \int_a^b \lim \limits _{n \rightarrow \infty} f_n(x) \mathrm{d} x.
  \end{equation}
\end{theorem}



\subsection{Differentiability of the Limit Function}

\begin{theorem}{Differentiability of the Limit Function}{}
  If $\{f_n\}$ converges uniformly to $f(x)$ on $[a, b]$,
  and each $f_n(x)$ has a continuous derivative on $[a, b]$,
  with $f_n^{\prime}(x)$ converging uniformly to $g(x)$.
  If there exists $x_0 \in [a, b]$ such that $f_n(x_0)$ converges,
  then $f_n(x)$ converges uniformly to a continuously differentiable function $f(x)$
  on $[a, b]$, and $f^{\prime}(x) = g(x)$, that is
  \begin{equation}
    (\lim \limits _{n \rightarrow \infty} f_n(x))^{\prime}
    = \lim \limits _{n \rightarrow \infty} f_n^{\prime}(x).
  \end{equation}
\end{theorem}



\section{Function Series}

\section{Power Series}

\subsection{Convergence Interval and Radius of Convergence}

\begin{theorem}{Abel's Theorem}{}
  If the power series $\sum\limits_{n = 1}^{\infty} a_nx^n$
  converges at $x_0 \neq 0$,
  then for all $x$ with $|x| < |x_0|$,
  $\sum\limits_{n = 1}^{\infty} a_nx^n$ converges absolutely.
  If it diverges at $x_0$,
  then for all $x$ with $|x| > |x_0|$, it diverges.
\end{theorem}










