
\section{Function Limits}

\subsection{Definition of Function Limits}

\begin{definition}{Function Limit}{}
  Let $x_0 \in \mathbb{R}$, and the function $f$ be defined in the neighborhood
  of $x_0$ (it can be undefined at $x_0$).
  If a real number $A$ satisfies
  \begin{equation}
    \forall \epsilon > 0, \exists \delta > 0, \forall x \in (x_0 - \delta, x_0 + \delta) - \{x_0\},
    |f(x) - A| < \epsilon,
  \end{equation}
  then it is called the \emph{limit} of $f(x)$ at $x = x_0$.
\end{definition}


\begin{proposition}{Uniqueness of Function Limits}{}
  If the limit of a function $f(x)$ at $x = x_0$ exists, then it is unique.
\end{proposition}

\begin{proof}
  Suppose there are two limits, and take $\epsilon$ as half of the difference
  between them.
  The proof can be completed according to the definition.
\end{proof}

\subsection{Properties of Function Limits}

\begin{proposition}{Four Arithmetic Operations of Function Limits}{}
  If the limits of $f(x), g(x)$ at $x = x_0$ exists,
  then the limits of $f+g, f-g, f \times g, f/g$
  (limit of the denominator is non-zero for the division operation)
  exist,
  and the limits are the corresponding arithmetic operations of the limits.
\end{proposition}

\begin{theorem}{Heine Theorem}{}
  The limit of function $f(x)$ at $x = x_0$ is $A$ if and only if
  for any sequence $\{x_n\}$ that converges to $x_0$ satisfying
  \begin{equation}
    \lim \limits _{n \rightarrow \infty} f(x_n) = A.
  \end{equation}
\end{theorem}

\subsection{Calculation of Function Limits}


\section{Continuity of Functions}

\subsection{Definition of Function Continuity}

\begin{definition}{Point-Continuity}{}
  Consider a function $f: [a, b] \rightarrow \mathbb{R}$
  and a real number $x_0 \in (a, b)$.
  We say $f(x)$ is \emph{continuous at $x_0$} if
  \begin{equation}
    \lim \limits _{x \rightarrow x_0} f(x) = f(x_0).
  \end{equation}
\end{definition}

\begin{note}
  If a function $f(x)$ is continuous at any point in the interval $(a, b)$,
  then we say it is continuous on $(a, b)$.
\end{note}

\begin{example}{Discontinuity of the Dirichlet Function}{}
  Determine the continuity of the following Dirichlet function:
  \begin{equation}
    D(x) =
    \begin{cases}
      1, & x \text{ is rational}\\
      0, & x \text{ is irrational}.
    \end{cases}
  \end{equation}
\end{example}

\begin{solution}
  Due to the denseness of rational and irrational numbers,
  for any $x \in \mathbb{R}$,
  we can always find two points with a difference of $1$ in any small neighborhood of
  $x$.
  Thus it is discontinuous everywhere.
\end{solution}

\begin{example}{Continuity of the Riemann Function}{}
  Prove that the Riemann function is continuous at irrational points
  and discontinuous at rational points
  \begin{equation}
    R(x) =
    \begin{cases}
      \frac{1}{q}, & x = \frac{p}{q} \text{ is a reduced fraction};\\
      0, & x \text{ is irrational}.
    \end{cases}
  \end{equation}
\end{example}

\begin{proof}
  Let $x_0 = \frac{p}{q}$ be a rational point,
  $R(x_0) = \frac{1}{q}$.
  As a sequence of irrational points converges to $x_0$,
  the limit is $0$, which is different from the function value $R(x_0) = 0$.
  Thus $R(x)$ is discontinuous at rational points.

  Let $x_0$ be an irrational point, for all $\epsilon > 0$, the number of $q$ satisfying
  $\frac{1}{q} > \epsilon$ is finite.
  Let these points be $x_1, \cdots, x_n$,
  and take $\delta < \min \{|x_0 - x_i|\}$,
  then we have
  \begin{equation}
    \forall x \in (x_0 - \delta, x_0 + \delta) - \{x_0\}, |f(x_0) - f(x)| < \epsilon,
  \end{equation}
  thus $R(x)$ is continuous at irrational points.
\end{proof}

\begin{proposition}{Continuity of Elementary Functions}{}
  Suppose $f, g$ are continuous functions on $[a, b]$,
  then $f^2(x)$, $\sqrt{|f(x)|}$, $|f(x)|$, $\max\{f(x), g(x)\}$, $\min \{f(x), g(x)\}$
  are continuous on $[a, b]$.
\end{proposition}

\begin{proof}
  Hint: $|f(x)| = \sqrt{f^2(x)}$,
  $\max \{f(x), g(x)\} = \frac{1}{2} \left[f(x) + g(x) + |f(x) - g(x)| \right]$.
\end{proof}

\subsection{Intermediate-Value Theorem}

\begin{theorem}{Zero-Point Existence Theorem}{}
  If $f(x)$ is a continuous function on $[a, b]$ and satisfies $f(a)f(b) < 0$,
  then there exists $\xi \in (a, b)$ such that
  \begin{equation}
    f(\xi) = 0.
  \end{equation}
\end{theorem}

\begin{theorem}{Intermediate-Value Theorem}{}
  If $f(x)$ is a continuous function on $[a, b]$,
  then for two points $x_1, x_2$ satisfying $a \leq x_1 < x_2 \leq b$ and
  $f(x_1) \neq f(x_2)$,
  $f(x)$ can take any value between $f(x_1)$ and $f(x_2)$ on $[a, b]$.
\end{theorem}


\section{Uniform Continuity of Functions}

\subsection{Definition of Uniform Continuity}

\begin{definition}{Uniform Continuity}{}
  Let $I$ be an arbitrary interval,
  and let $f$ be a function $f: I \rightarrow \mathbb{R}$.
  If for any $\epsilon > 0$, there exists $\delta > 0$ such that for all $x_1,
  x_2 \in I$, when $|x_1 - x_2| < \delta$, we have
  \begin{equation}
    |f(x_1) - f(x_2)| < \epsilon,
  \end{equation}
  then $f$ is said to be \emph{uniformly continuous} on $I$.
\end{definition}

\begin{proposition}{Sequence Perspective}{}
  Let $f(x)$ be a function defined on $I$,
  then $f(x)$ is uniformly continuous on $I$ if and only if
  for any sequences $x_n, y_n$ in $I$,
  if $\lim \limits _{n \rightarrow \infty} (x_n - y_n) = 0$,
  then
  \begin{equation}
    \lim \limits _{n \rightarrow \infty} \left[ f(x_n) - f(y_n) \right] = 0.
  \end{equation}
\end{proposition}

\begin{proof}
  From left to right is obvious, so we only prove the reverse direction.
  Assume $f$ is not uniformly continuous,
  then there exists $\epsilon_0 > 0$, for any $\delta > 0$,
  there exist $x, y$ such that $|x - y| < \delta$, but
  \begin{equation}
    |f(x) - f(y)| \geq \epsilon_0.
  \end{equation}
  Take $\delta = 1, \frac{1}{2}, \cdots, \frac{1}{n}$, there exist $x_n, y_n$
  such that $|x_n - y_n| < \frac{1}{n}$ and $|f(x_n) - f(y_n)| \geq \epsilon_0$,
  which contradicts the given condition.
\end{proof}

\begin{note}
  In most cases, the above theorem is used to prove that
  a function is not uniformly continuous.
\end{note}

\begin{example}{Examples of Continuous but Not Uniformly Continuous Functions}{}
  Prove the following functions are not uniformly continuous:
  (1) $x^2$ on $(0, +\infty)$,
  (2) $\sin x^2$ on $[0, + \infty)$,
  (3) $\frac{1}{x}$ on $(0, 1)$.
\end{example}

\begin{proof}
  (1) Take $x_n = \sqrt{n}, y_n = \sqrt{n + 1}$;
  (2) Take $x_n = \sqrt{2n \pi}, y_n = \sqrt{2n \pi + \frac{\pi}{2}}$;
  (3) Take $x_n = \frac{1}{n}, y_n = \frac{1}{n+1}$.
\end{proof}

\begin{example}{Prove Uniform Continuity}{}
  Prove that the following functions are uniformly continuous:
  (1) $\cos \sqrt{x}$ on $[0, +\infty)$
  (2) $\sqrt{x}$ on $[0, +\infty)$
  (3) $\sin x$ on $[0, +\infty)$.
\end{example}

\begin{proof}
  
\end{proof}

\subsection{Cantor Theorem and Its Generalization}

\begin{theorem}{Cantor Theorem}{}
  If $f(x)$ is continuous on the closed interval $[a, b]$,
  then it is uniformly continuous on $[a, b]$.
\end{theorem}

\begin{proof}
  
\end{proof}

\begin{theorem}{Cantor Theorem on an Infinite Interval}{}
  If $f(x)$ is continuous on $[a, +\infty)$,
  and $\lim \limits _{x \rightarrow +\infty} f(x) = A$,
  then $f(x)$ is uniformly continuous on $[a, +\infty)$.
\end{theorem}

\subsection{Lipschitz Continuity}

\begin{definition}{Lipschitz Continuity}{}
  If a function $f(x): I \rightarrow \mathbb{R}$
  satisfies the Lipschitz condition, i.e.,
  there exists a positive real number $K$ such that
  for any $x_1, x_2 \in I$,
  \begin{equation}
    |f(x_1) - f(x_2)| \leq K |x_1 - x_2|.
  \end{equation}
  Then it is \emph{Lipschitz continuous} on $I$.
\end{definition}

\begin{proposition}{Lipschitz Continuity and Uniform Continuity}{}
  If $f(x)$ is Lipschitz continuous on the interval $I$,
  then it is uniformly continuous on $I$.
\end{proposition}

\begin{example}{Use Lipschitz Continuity to Prove Uniform Continuity}{}
  Determine the uniform continuity of
  (1) $\sin x$ on $\mathbb{R}$;
  (2) $x^k$ on $(1, +\infty)$;
  (3) $\cos \sqrt{x}$ on $(1, +\infty)$;
  (4) $x \ln x$ on $(0, +\infty)$.
\end{example}

\begin{solution}
  
\end{solution}

