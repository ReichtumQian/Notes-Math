
\section{Function Limits}

\subsection{Definition of Function Limits}

\begin{definition}{Function Limit}{}
  Let $x_0 \in \mathbb{R}$, and the function $f$ be defined in the neighborhood
  of $x_0$ (it can be undefined at $x_0$).
  If a real number $A$ satisfies
  \begin{equation}
    \forall \epsilon > 0, \exists \delta > 0, \forall x \in (x_0 - \delta, x_0 + \delta) - \{x_0\},
    |f(x) - A| < \epsilon,
  \end{equation}
  then it is called the \emph{limit} of $f(x)$ at $x = x_0$.
\end{definition}


\begin{proposition}{Uniqueness of Function Limits}{}
  If the limit of a function $f(x)$ at $x = x_0$ exists, then it is unique.
\end{proposition}

\begin{proof}
  Suppose there are two limits, and take $\epsilon$ as half of the difference
  between them.
  The proof can be completed according to the definition.
\end{proof}

\subsection{Properties of Function Limits}

\begin{proposition}{Four Arithmetic Operations of Function Limits}{}
  If the limits of $f(x), g(x)$ at $x = x_0$ exists,
  then the limits of $f+g, f-g, f \times g, f/g$
  (limit of the denominator is non-zero for the division operation)
  exist,
  and the limits are the corresponding arithmetic operations of the limits.
\end{proposition}

\begin{theorem}{Heine Theorem}{}
  The limit of function $f(x)$ at $x = x_0$ is $A$ if and only if
  for any sequence $\{x_n\}$ that converges to $x_0$ satisfying
  \begin{equation}
    \lim \limits _{n \rightarrow \infty} f(x_n) = A.
  \end{equation}
\end{theorem}

\begin{proof}
  
\end{proof}

\section{Calculation of Function Limits}

\subsection{Formulas of Equivalent Infinitesimals}


\subsection{Taylor Expansion Formulas}


\section{Continuity of Functions}


\section{Uniform Continuity}


