
\section{Concepts and Properties of Sequences Limits}

\begin{definition}{Limits of Sequences}{}
  Consider a sequence of real numbers $a_n$, and a real number $A$.
  We say that the sequence $\{a_n\}$ \emph{converges} to $A$ if
  \begin{equation}
    \forall \epsilon > 0, \exists N \in \mathbb{Z}^+, \forall n > N, |a_n - A| < \epsilon,
  \end{equation}
  denoting as $\lim \limits _{n \rightarrow \infty}a_n = A$.
\end{definition}

\begin{proposition}{Four Arithmetic Operations of Limits}{}
  If the limits $\lim \limits _{n \rightarrow \infty} a_n$ and $\lim \limits _{n \rightarrow \infty} b_n$
  both exist, then
  \begin{equation}
    \lim \limits _{n \rightarrow \infty} (a_n \ast b_n) = \lim \limits _{n \rightarrow \infty} a_n \ast \lim \limits _{n \rightarrow \infty} b_n,
  \end{equation}
  where $\ast$ represents any of four arithmetic operations
  (when it is division, an additional requirement is $\lim \limits _{n
    \rightarrow \infty} b_n \neq 0$).
\end{proposition}

\begin{theorem}{Cauchy Criterion for Convergence}{}
  A sequence $\{a_n\}$ converges if and only if
  \begin{equation}
    \forall \epsilon > 0, \exists N \in \mathbb{N}^+, ~ \mathrm{s.t.}~ \forall m,n > N, |a_n - a_m| < \epsilon.
  \end{equation}
\end{theorem}

\begin{exercise}{}{}
  
\end{exercise}
