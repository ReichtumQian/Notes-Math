
\section{Concepts and Properties of Sequence Limits}

\subsection{Definition of Sequence Limits}

\begin{definition}{Limits of Sequences}{}
  Consider a sequence of real numbers $a_n$, and a real number $A$.
  We say that the sequence $\{a_n\}$ \emph{converges} to $A$ if
  \begin{equation}
    \forall \epsilon > 0, \exists N \in \mathbb{Z}^+, \forall n > N, |a_n - A| < \epsilon,
  \end{equation}
  denoting as $\lim \limits _{n \rightarrow \infty}a_n = A$.
\end{definition}

\begin{proposition}{Four Arithmetic Operations of Limits}{}
  If the limits $\lim \limits _{n \rightarrow \infty} a_n$ and $\lim \limits _{n \rightarrow \infty} b_n$
  both exist, then
  \begin{equation}
    \lim \limits _{n \rightarrow \infty} (a_n \ast b_n) = \lim \limits _{n \rightarrow \infty} a_n \ast \lim \limits _{n \rightarrow \infty} b_n,
  \end{equation}
  where $\ast$ represents any of four arithmetic operations
  (when it is division, an additional requirement is $\lim \limits _{n
    \rightarrow \infty} b_n \neq 0$).
\end{proposition}

\begin{theorem}{Cauchy Criterion for Convergence}{}
  A sequence $\{a_n\}$ converges if and only if
  \begin{equation}
    \forall \epsilon > 0, \exists N \in \mathbb{Z}^+, \forall m,n > N, |a_n - a_m| < \epsilon.
  \end{equation}
\end{theorem}

\begin{corollary}{Contrapositive of Cauchy Criterion for Convergence}{}
  A sequence $\{a_n\}$ diverges if and only if
  \begin{equation}
    \exists \epsilon > 0, \forall N \in \mathbb{Z}^+, \exists m, n \geq N, |a_n - a_m| \geq \epsilon.
  \end{equation}
\end{corollary}

\begin{example}{Apply the Cauchy Criterion to Prove Divergence}{}
  \begin{enumerate}
  \item Prove that the sequence $\sin n$ diverges.
  \item Prove that the sequence $a_n = 1 + \frac{1}{2} + \cdots + \frac{1}{n}$ diverges.
  \end{enumerate}
\end{example}

\begin{proof}
  (1) Notice that the interval $[2k\pi + \frac{\pi}{4}, 2k\pi + \frac{3\pi}{4}]$
  has a length greater than $1$, so it contains an integer.
  Let $n$ be an integer in this interval
  and $m$ be an integer in $[2k\pi + \pi, 2k\pi + 2\pi]$.
  As $k$ increases, both $n$ and $m$ can go to infinity.
  And we know that $|\sin n - \sin m| \geq \frac{\sqrt{2}}{2}$.
  By the Cauchy convergence criterion, the sequence $\sin n$ diverges.

  (2) For any given $N$, set $n = N$ and $m = 2N$.
  Then $|a_n - a_m| = \sum\limits_{i = N + 1}^{2N} \frac{1}{i} \geq \frac{N}{2N}
  = \frac{1}{2}$.
  Using the contrapositive of the Cauchy criterion criterion,
  we can conclude that the sequence $a_n$ diverges.
\end{proof}

\subsection{Properties of Convergent Sequences}

\begin{proposition}{Properties of Convergent Sequences}{}
  If a sequence $\{a_n\}$ converges to $A$, i.e.,
  $\lim \limits _{n \rightarrow \infty} a_n = A$, then
  \begin{enumerate}
  \item The sequence $a_n$ is bounded;
  \item If $A > a$, then as $n$ approaches infinity, $a_n > a$;
  \item Every sub-sequence of $\{a_n\}$ converges to $A$.
  \end{enumerate}
\end{proposition}

\begin{theorem}{Even and Odd Sub-sequence Convergence}{}
  A sequence $\{a_n\}$ converges to $A$ if and only if both its sub-sequences
  $\{a_{2k}\}$ and $\{a_{2k+1}\}$ converge to $A$.
\end{theorem}

\begin{proof}
  Forward direction is obvious due to the Cauchy criterion for convergence,
  so we only prove the reverse direction.
  Since $\lim \limits _{k \rightarrow \infty} a_{2k} = A$, we have
  \begin{equation}
    \forall \epsilon > 0, \exists K \in \mathbb{Z}^+, \forall k > \frac{K}{2}, |a_{2k} - A| < \epsilon.
  \end{equation}
  For this $\epsilon$, there exists $K^{\prime}$ such that for all $k >
  K^{\prime}$ satisfies $|a_{2k + 1} - A| < \epsilon$.
  Then we choose $N = \max \{K, K^{\prime}\}$, and for all $n > N$ satisfies
  $|a_n - A| < \epsilon$.
\end{proof}

\section{Some Important Sequence Limit Conclusions}

\subsection{Several Basic Limits}

\begin{proposition}{Common Basic Limits}{}
  If $a > 0, b > 1$, the relationships of basic limits are
  $\log n < n^a < b^n < n! < n^n$. Specifically:
  \begin{itemize}
  \item Power-form: $\lim \limits _{n \rightarrow \infty} \sqrt[n]{a} = 1$,
    $\lim \limits _{n \rightarrow \infty} \sqrt[n]{n} = 1$,
    $\lim \limits _{n \rightarrow \infty} \sqrt[n]{n!} = +\infty$.
  \item Ratio-form:
  \item Logarithmic-form:
  \item Exponential-form:
  \end{itemize}
\end{proposition}

\subsection{Stirling's Formula and Wallis' Formula}


