
\section{Indefinite Integrals}

\subsection{Basic Methods of Indefinite Integrals}

\begin{proposition}{Integration-by-Parts Formula}{}
  The integral of two functions $u$ and $v$ satisfies
  \begin{equation}
    \int u \mathrm{d} v = uv - \int v \mathrm{d} u.
  \end{equation}
\end{proposition}

\begin{corollary}{Tabular Method}{}
  The integral of functions $u$ and $v$ satisfy
  \begin{equation}
    \int u v^{(n+1)} \mathrm{d} x
    = uv^{(n)} - u^{\prime}v^{(n-1)} + u^{\prime\prime} v^{(n-2)}
    + \cdots + (-1)^n u^{(n)}v + (-1)^{n+1} \int u^{(n+1)}v \mathrm{d} x.
  \end{equation}
\end{corollary}

The specific operation can be carried out in the following table:
\begin{table}[htbp]
  \centering
  \begin{tabular}{c|c|c|c|c|c|c}
    \hline
    $u$'s successive derivatives&$u$&$u^{\prime}$&$u^{\prime\prime}$&$u^{\prime\prime\prime}$&$\cdots$&$u^{(n+1)}$ \\ \hline
    $v^{(n+1)}$'s successive antiderivatives&$v^{(n+1)}$&$v^{(n)}$&$v^{(n-1)}$&$v^{(n-2)}$&$\cdots$&$v$\\ \hline
  \end{tabular}
\end{table}

\begin{note}
  How to choose $u$: inverse trigonometric functions,  logarithmic functions,
  power functions, exponential functions, trigonometric functions.
\end{note}

\begin{example}{Tabular Method}{}
  Calculate $\int (x^2 + 2x + 6) e^{2x}\mathrm{d} x$.
\end{example}

\begin{solution}
  Define $u = x^2 + 2x + 6$ and $v = e^{2x}$,
  by the tabular method we have
  \begin{equation}
    \int u v \mathrm{d} x = (x^2 + 2x + 6) \frac{1}{2}e^{2x} - (2x + 2) \frac{1}{4}e^{2x}
    + \frac{1}{4} e^{2x} - 0 = \frac{1}{2}x^2 e^{2x} + \frac{1}{2} xe^{2x} + \frac{11}{4}e^{2x}.
  \end{equation}
\end{solution}

\subsection{Commonly Used Indefinite Integral Formulas}

\begin{proposition}{Commonly Used Indefinite Integral Formulas}{}
  The following formulas are commonly used and should be memorized.
  \begin{align}
      &\int \sin x \, \mathrm{d}x = -\cos x + c 
      && \int \cos x \, \mathrm{d}x = \sin x + c \\
      &\int \sec x \, \mathrm{d}x = \ln|\sec x + \tan x| + c 
      && \int \sec x \tan x \, \mathrm{d}x = \sec x + c \\
      &\int \sec^2 x \, \mathrm{d}x = \tan x + c 
      && \int \csc^2 x \, \mathrm{d}x = -\cot x + c \\
      &\int \frac{1}{x^2 + a^2} \, \mathrm{d}x = \frac{1}{a} \arctan \frac{x}{a} + c 
      && \int \frac{1}{x^2 - a^2} \, \mathrm{d}x = \frac{1}{2a} \ln \left| \frac{x-a}{x+a} \right| + c \\
      &\int \frac{1}{\sqrt{x^2 + a^2}} \, \mathrm{d}x = \ln \left| x + \sqrt{x^2 + a^2} \right| + c 
      && \int \frac{1}{\sqrt{x^2 - a^2}} \, \mathrm{d}x = \ln \left| x + \sqrt{x^2 - a^2} \right| + c \\
      &\int \frac{1}{\sqrt{a^2 - x^2}} \, \mathrm{d}x = \arcsin \frac{x}{a} + c 
      && {} \\
      &\int \ln x \, \mathrm{d}x = x \ln x - x + c 
      && {}
  \end{align}
\end{proposition}

\section{Definite Integrals}

\subsection{Integrability}

\begin{definition}{Riemann Integral}{}
  Let $f(x)$ be a function $[a, b]$, and divide $[a, b]$ into $n$ sub-intervals,
  representing the division by $T$.
  If
  \begin{equation}
    \lim \limits _{\|T\| \rightarrow 0} \sum\limits_{i = 1}^n f(\xi_i) \Delta x_i = J,
  \end{equation}
  where $J$ is a constant, then we say that $f(x)$ is \emph{integrable} on $[a, b]$,
  and denoted as
  \begin{equation}
    \int_a^b f(x) \mathrm{d} x = J.
  \end{equation}
\end{definition}

\begin{proposition}{Boundedness of Integrable Functions}{}
  If $f(x)$ is integrable on $[a, b]$, then it is bounded on $[a, b]$.
\end{proposition}

\begin{note}
  Note that bounded functions are not necessarily integrable, such as the
  Dirichlet function.
\end{note}

\begin{definition}{Darboux Upper and Lower Sum}{}
  On each sub-interval, take $M_i = \sup \limits_{x \in \Delta_i} f(x)$,
  and $m_i = \inf \limits_{x \in \Delta_i} f(x)$.
  Denote the oscillation $\omega_i = M_i - m_i$,
  then we define the \emph{Darboux upper sum} and \emph{Darboux lower sum} as:
  \begin{equation}
    S(T) := \sum\limits_{i = 1}^n M_i \Delta x_i, \quad
    s(T) := \sum\limits_{i = 1}^n m_i \Delta x_i.
  \end{equation}
\end{definition}

\begin{proposition}{Necessary and Sufficient Conditions for Integrability}{}
  A function $f(x)$ is integrable on $[a, b]$ if and only if
  for any $\epsilon > 0$, there exists a partition such that
  \begin{equation}
    S(T) - s(T) = \sum\limits_{i = 1}^n \omega_i \Delta x_i < \epsilon.
  \end{equation}
\end{proposition}

\begin{note}
  According to real analysis, we know that a function $f(x)$ is
  Riemann-integrable if and only if it is continuous almost everywhere.
\end{note}

\begin{example}{Non-integrability of the Dirichlet Function}{}
  Prove that Dirichlet function is not integrable on the interval $[a, b]$
  \begin{equation}
    D(x) =
    \begin{cases}
      1, & x \text{ is rational};\\
      0, & x \text{ is irrational}.
    \end{cases}
  \end{equation}
\end{example}

\begin{proof}
  For any partition $T$,
  the oscillation $\omega = 1$.
  So $S(T) - s(T) = \sum\limits_{i = 1}^n \Delta x_i = b - a$,
  which means that $f(x)$ is not integrable.
\end{proof}

\begin{example}{Integrability of the Riemann Function}{}
  Prove that the Riemann function $R(x)$ is integrable on $[0, 1]$,
  and satisfies
  \begin{equation}
    \int_0^1 R(x) \mathrm{d} x = 0.
  \end{equation}
\end{example}

\begin{proof}
  For any $\epsilon > 0$, there are only finitely many $x = \frac{p}{q}$ satisfying
  $\frac{1}{q} > \frac{\epsilon}{2}$,
  and their function values are less than or equal to $\frac{1}{2}$.
  List these finite points as $x_1,\cdots,x_k$.
  Each point can at most fall into two sub-intervals (when it is endpoint),
  so we take $\|T\| \leq \frac{\epsilon}{2k}$,
  then we have
  \begin{equation}
    \sum _T \omega_i \Delta x_i < \epsilon.
  \end{equation}
  This implies that the integral is zero.
\end{proof}

\begin{proposition}{Integrability of the Function after Elementary Operations}{}
  Suppose $f(x), g(x)$ are two integrable functions on $[a, b]$,
  then 
  \begin{equation}
    kf(x), \quad f(x) \pm g(x), \quad f(x)g(x)
  \end{equation}
  are integrable on $[a, b]$.
\end{proposition}

\begin{proof}
  The scalar multiplication and addition are direct,
  we only prove that $f(x)g(x)$ is integrable.
  Since both $f(x)$ and $g(x)$ are integrable, then they are all bounded.
  Assume $|f(x)| \leq M, |g(x)| \leq N$, then
  \begin{equation}
    \sum \omega_i^{fg} \Delta x_i \leq
    M \sum \omega_i^g \Delta x_i + N \sum \omega_i^f \Delta x_i.
  \end{equation}
  The limit is zero since both two terms go to zero.
\end{proof}

\begin{note}
  Note that even when $g(x) \neq 0$, the function
  \begin{equation}
    \varphi(x) = \frac{f(x)}{g(x)}
  \end{equation}
  is not necessarily integrable.
  For example $f(x) = 1$, and
  \begin{equation}
    g(x) =
    \begin{cases}
      x & x \in (0, 1];\\
      1 & x = 0.
    \end{cases}
  \end{equation}
  Then $f(x)/g(x)$ is unbounded, and therefore non-integrable.
\end{note}

\begin{proposition}{Integrability of Composition Functions}{}
  If $f(x)$ is continuous on the interval $[a, b]$,
  and $g(t)$ is integrable on the interval $[\alpha, \beta]$,
  satisfying $a \leq g(t) \leq b, t \in [\alpha, \beta]$,
  then
  \begin{equation}
    \varphi(t) = f(g(t))
  \end{equation}
  is integrable on $[\alpha, \beta]$.
\end{proposition}

\begin{proof}
  Since $f(x)$ is contunuous on a closed interval, then it is bounded and
  uniformly continuous. Let $M = \max f(x)$ and we have
  \begin{equation}
    \forall \epsilon > 0, \exists \delta > 0, \forall x_1, x_2, |x_1 - x_2| < \delta,
    |f(x_1) - f(x_2)| < \epsilon.
  \end{equation}
  Since $\varphi(x)$ is integrable, there exists a partition $T$ such that
  $\sum \limits_{\omega_i^{\varphi} \geq \delta} \Delta x_i < \epsilon$.
  Then
  \begin{align}
    \sum\limits_T \omega_i^{f \circ \varphi} \Delta t_i &= \sum\limits_{\omega_i^{\varphi} \geq \delta} \omega_i^{f \circ \varphi} \Delta t_i + \sum\limits_{\omega_i^{\varphi} < \delta} \omega_i^{f \circ \varphi} \Delta t_i\\
    &\leq 2M \sum\limits_{\omega_i^{\varphi} \geq \delta} \Delta t_i + \epsilon \sum\limits_{\omega_i^{\varphi} < \delta} \Delta t_i\\
    &< 2M \epsilon + (\beta - \alpha)\epsilon = \epsilon^{\prime}.
  \end{align}
  The conclusion follows from the condition for integrability.
\end{proof}

\subsection{Newton-Leibniz Formula}

\begin{theorem}{Newton-Leibniz Formula}{}
  Let $f(x)$ be an integrable function on $[a, b]$,
  and $F(x)$ be a continuous function on $[a, b]$.
  If except for finite number of points,
  $F^{\prime}(x) = f(x)$, then
  \begin{equation}
    \int_a^b f(x) \mathrm{d} x = F(b) - F(a).
  \end{equation}
\end{theorem}

\begin{theorem}{Fundamental Theorem of Calculus}{}
  Let $f(x)$ be an integrable function on $[a, b]$.
  Define $F(x) = \int_a^x f(t)\mathrm{d} t$,
  where $x \in [a, b]$. Then
  \begin{itemize}
  \item $F(x)$ is a continuous function on $[a, b]$;
  \item If $x_0 \in [a, b]$ is a continuous point of $f(x)$,
    then $F(x)$ is differentiable at $x_0$,
    and $F^{\prime}(x_0) = f(x_0)$.
  \item If $f(x)$ is a continuous function on $[a, b]$,
    then $F(x)$ is a continuously-differentiable function on $[a, b]$,
    and $F^{\prime}(x) = f(x)$.
  \end{itemize}
\end{theorem}

\begin{example}{Application of Fundamental Theorem of Calculus}{}
  Suppose a function $f(x)$ is integrable on the interval $[A, B]$,
  and $a, b \in (A, B)$ be two points at which $f(x)$ is continuous, prove that
  \begin{equation}
    \lim \limits _{h \rightarrow 0} \int_a^b \frac{f(x+h) - f(x)}{h}\mathrm{d} x = f(b) - f(a).
  \end{equation}
\end{example}

\begin{proof}
  Simplify the left-hand-side with the following steps
  \begin{align}
    \int_a^b \frac{f(x+h) - f(x)}{h}\mathrm{d} x &= \frac{1}{h} \left( \int_a^b f(x+h)\mathrm{d} x - \int_a^b f(x)\mathrm{d}x \right)\\
    &= \frac{1}{h} \left( \int_{a+h}^{b+h} f(x)\mathrm{d} x - \int_a^b f(x)\mathrm{d} x \right)\\
    &= \frac{1}{h} \left( \int _b^{b+h} f(x)\mathrm{d} x - \int_a^{a+h} f(x)\mathrm{d} x \right)
  \end{align}
  Define $F(x) = \int_A^xf(t)\mathrm{d}t$,
  then by the fundamental theorem of calculus
  \begin{equation}
    F^{\prime}(b) = \lim \limits _{h \rightarrow 0^+} \frac{F(b + h) - F(b)}{h} = f(b),
  \end{equation}
  similarly $F^{\prime}(a) = f(a)$.
\end{proof}

\subsection{The Mean Value Theorems for Integrals}

\begin{theorem}{The First Mean Value Theorem for Integrals}{}
  If $f(x), g(x)$ are continuous on $[a, b]$,
  and $g(x)$ does not change sign on $[a, b]$,
  then there exists $\xi \in (a, b)$ such that
  \begin{equation}
    \int_a^b f(x) g(x) \mathrm{d} x = f(\xi) \int_a^b g(x) \mathrm{d} x.
  \end{equation}
\end{theorem}

\begin{proof}
  Without loss of generality,
  assume $g(x) \geq 0$. Since $f(x)$ is continuous on $[a, b]$,
  we denote $M$ and $m$ the maximum and minimum values of $f(x)$ respectively.
  And we have
  \begin{equation}
    m g(x) \leq f(x)g(x) \leq Mg(x) \Rightarrow
    m \int_a^b g(x)\mathrm{d} x \leq \int_a^b f(x)g(x)\mathrm{d} x \leq M \int_a^b g(x)\mathrm{d} x.
  \end{equation}
  According to the intermediate-value property we can get the conclusion.
\end{proof}

\begin{theorem}{The Second Mean Value Theorem for Integrals}{}
  Suppose $f(x)$ is integrable on $[a, b]$ and $g(x)$ is monotonic on $[a, b]$.
  Then 
  \begin{itemize}
  \item If $g(x)$ is non-decreasing and non-negative: There exists $\xi \in (a,
    b)$ such that
    \begin{equation}
      \int_a^b f(x) g(x) \mathrm{d} x = g(b) \int_{\xi}^b f(x) \mathrm{d} x.
    \end{equation}
  \item If $g(x)$ is non-increasing and non-negative: There exists $\xi \in (a,
    b)$ such that
    \begin{equation}
      \int_a^b f(x) g(x) \mathrm{d} x = g(a) \int_{a}^{\xi} f(x) \mathrm{d} x.
    \end{equation}
  \end{itemize}
\end{theorem}

\section{Improper Integrals}

\subsection{The Concepts of Improper Integrals}

\begin{definition}{Improper Integrals over Infinite Intervals}{}
  Let $f$ be defined on $[a, +\infty)$.
  If the limit $\lim \limits _{b \rightarrow +\infty} \int_a^b f(x)\mathrm{d} x$
  exists,
  then we define
  \begin{equation}
    \int_a^{+\infty} f(x) \mathrm{d} x := \lim \limits _{b \rightarrow +\infty} \int_a^b f(x)\mathrm{d} x,
  \end{equation}
  and call it \emph{an improper integral over an infinite interval}.
\end{definition}

\begin{example}{Improper Integrals over Infinite Interval}{}
  prove that $\int_1^{+\infty} \frac{\mathrm{d}x}{x^p}$ converges when $p > 1$
  and diverges when $p \leq 1$.
\end{example}

\begin{proof}
  The antiderivative of $x^{-p}$ is
  \begin{equation}
    F(x) = \int x^{-p}\mathrm{d} x =
    \begin{cases}
      \frac{1}{1-p} x^{1-p}, & p \neq 1;\\
      \ln x, & p =1.
    \end{cases}
  \end{equation}
  Consider $F(+\infty) - F(1)$.
\end{proof}

\begin{example}{Properties of Convergent Improper Integrals}{}
  Prove the following propositions:
  \begin{enumerate}
  \item If $\int_a^{\infty}f(x)\mathrm{d} x$ is convergent,
    and $\lim \limits _{x \rightarrow \infty}f(x) = A$ exists,
    then $A = 0$.
  \item Find an example satisfies that $f(x)$ is continuous on $[a, +\infty)$,
    $\int_a^{\infty}f(x)\mathrm{d} x$ is convergent,
    but the limit $\lim \limits _{x \rightarrow \infty}f(x)$ does not exist.
  \item Find an example satisfies that $f(x)$ is integrable on an arbitrary interval,
    and $\int_a^{\infty}f(x)\mathrm{d} x$ is convergent, but $f(x)$ is not bounded.
  \item If $f(x)$ is uniformly continuous on $[a, \infty)$,
    and $\int_a^{+\infty}f(x)\mathrm{d} x$ is convergent,
    then $\lim \limits _{x \rightarrow \infty}f(x) = 0$.
  \end{enumerate}
\end{example}

\begin{proof}
  (1) If $A \neq 0$, without loss of generality, we assume $A > 0$.
  Then
  \begin{equation}
    \exists M \geq a, \forall x \geq M, f(x) \geq \frac{M}{2}.
  \end{equation}
  But $\int_a^{\infty}\frac{M}{2}\mathrm{d} x$ is not convergent,
  and the compare test implies $\int_a^{\infty}f(x)\mathrm{d} x$ also diverges.
  This contradicts the given condition.

  (2) $\int_1^{\infty} \sin x^2\mathrm{d} x$, substitue $x = \sqrt{t}$ and use Dirichlet
  test to prove the convergence. But the limit does not exist.

  (3) Construct a function $f(x)$ that takes $x$ when $x$ is an integer, and $0$ otherwise.

  (4) Assume $\lim \limits _{x \rightarrow \infty}f(x) \neq 0$, then there
  exists $\epsilon > 0$, $N > 0$ and a sequence $\{x_n\}$ such that $|f(x_n)| \geq
  \epsilon, n \geq N$.
  By the definition of uniformly continuous, there exists $\delta$ such that
  \begin{equation}
    \forall x, y, |x - y| \leq \delta, |f(x) - f(y)| < \frac{\epsilon}{2}.
  \end{equation}
  Then for each $x_n$, for all $x \in [x_n, x_n + \delta]$,
  $|f(x) - f(x_n)| < \frac{\epsilon}{2}$, we get
  \begin{equation}
    |f(x)| \geq |f(x_n)| - |f(x) - f(x_n)| > \frac{\epsilon}{2}.
  \end{equation}
  That means
  \begin{equation}
    \left| \int_{x_n}^{x_n + \delta} f(x)\mathrm{d} x \right|
    = \int_{x_n}^{x_n + \delta}|f(x)|\mathrm{d} x
    \geq \int_{x_n}^{x_n + \delta} \frac{\epsilon}{2} \mathrm{d} x
     = \frac{\epsilon \delta}{2} \not \rightarrow 0.
  \end{equation}
  This contradicts the given condition.
\end{proof}

\begin{definition}{Improper Integrals with Singular Points}{}
  Let $f(x)$ be defined $(a, b]$,
  and $\lim \limits _{x \rightarrow a^+}f(x) = \infty$.
  If the limit $\lim \limits _{\epsilon \rightarrow 0^+} \int_{a+\epsilon}^b
  f(x) \mathrm{d} x$ exists,
  then we define
  \begin{equation}
    \int_a^b f(x) \mathrm{d} x := \lim \limits _{\epsilon \rightarrow 0^+} \int_{a+\epsilon}^b
  f(x) \mathrm{d} x
  \end{equation}
  and call it \emph{an improper integral with a singular point $a$}.
\end{definition}

\begin{theorem}{Leibniz Formula for Improper Integrals}{}
  Suppose the limits exist, and $f$ has an antiderivative $F$.
  Then we have
  \begin{equation}
    \int_a^{+\infty} f(x) \mathrm{d} x = F(+\infty) - F(a), \quad
    \int_a^b f(x) \mathrm{d} x = F(b) - F(a+0).
  \end{equation}
\end{theorem}

\begin{example}{Improper Integrals with Singular Points}{}
  Prove that $\int_0^1 \frac{\mathrm{d} x}{x^p}$ converges when $p < 1$,
  and diverges when $p \geq 1$.
\end{example}

\begin{proof}
  The antiderivative of $x^{-p}$ is
  \begin{equation}
    F(x) = \int x^{-p}\mathrm{d} x =
    \begin{cases}
      \frac{1}{1-p} x^{1-p}, & p \neq 1;\\
      \ln x, & p =1.
    \end{cases}
  \end{equation}
  We observe that when $p < 1$, $F(1) = \frac{1}{1-p}$, and $F(\epsilon) =
  \frac{1}{1-p} \epsilon^{1-p}$. Therefore
  \begin{equation}
    \lim \limits _{\epsilon \rightarrow 0^+} F(1) - F(\epsilon) = \frac{1}{1-p},
  \end{equation}
  the limit exists. Similarly, it is not hard to prove that when $p \geq 1$, the
  limit does not exist.
\end{proof}

\subsection{Compare Tests for Improper Integrals}

\begin{proposition}{Compare Test for Improper Integrals}{}
  Let $f(x), g(x)$ be a non-negative function on $I$.
  If for any $x \in I$, $0 \leq f(x) \leq g(x)$, and $\int_a^bg(x)\mathrm{d} x$
  (Here $b$ can be infinity)
  is convergent, then $\int_a^b f(x)\mathrm{d} x$ is convergent.
\end{proposition}

\begin{example}{Applications of the Compare Test}{}
  Determine the convergence of the following improper integrals
  \begin{equation}
    (1) \int_0^{\infty} \frac{x \arctan x}{1 + x^p}\mathrm{d} x, \quad
    (2) \int_0^{\infty} \frac{x}{1-e^x}\mathrm{d} x, \quad
    (3) \int_0^{\infty} \frac{\ln x}{e^x}\mathrm{d}x, \quad
    (4) \int_3^{\infty} \frac{1}{x^p(\ln x)^q (\ln \ln x)^r}\mathrm{d} x.
  \end{equation}
\end{example}

\begin{solution}
  (1) $0$ is not a singular point, so we only consider $x \rightarrow \infty$.
  We know $\frac{x \arctan x}{1 + x^p} \sim \frac{\pi}{2} x^{1-p}$.
  Then the improper integral is convergent if and only if $p > 2$.

  (2) First consider point $0$, $1-e^x \sim -x$, thus $\lim \limits _{x
  \rightarrow 0} \frac{x}{1-e^x} \neq 0$, it is not a singular point.
  When $x$ approximates infinity, $e^x \sim x^{\infty}$, so it is convergent.

  (3) First consider point $0$, $e^0 = 1$ and $\ln x < x^{-\epsilon}$,
  $\epsilon < 1$ then it is convergent at $0$.
  Consider $x$ goes up to infinity, since $e^x \sim x^{\infty}$,
  then it is convergent.

  (4) If $p > 1$ then it is convergent, and $p<1$ it does not converge.
  If $p = 1$, then consider
  \begin{equation}
    \int_3^{+\infty} \frac{\mathrm{d} (\ln x)}{(\ln x)^q (\ln \ln x)^r}
    = \int_{\ln 3}^{\infty} \frac{\mathrm{d} u}{u^q (\ln u)^r}.
  \end{equation}
  It is convergent when $q > 1$, and not convergent when $q < 1$.
  If $q = 1$, it is similar to the discussion of $p = 1$.
\end{solution}

\begin{note}
  Why $\ln x < x^{-\epsilon}$ when $x \rightarrow 0$
  and $\ln x < x^{\epsilon}$ when $x \rightarrow \infty$:
  \begin{equation}
    \lim \limits _{x \rightarrow 0} \frac{\ln x}{x^{-\epsilon}}
    = \lim \limits _{x \rightarrow 0} \frac{x^{-1}}{-\epsilon x^{-\epsilon - 1}}
    = \lim \limits _{x \rightarrow 0} \frac{1}{-\epsilon x^{-\epsilon}} \rightarrow 0.
  \end{equation}
  \begin{equation}
    \lim \limits _{x \rightarrow \infty} \frac{\ln x}{x^{\epsilon}}
    = \lim \limits _{x \rightarrow \infty} \frac{x^{-1}}{\epsilon x^{\epsilon - 1}}
    = \lim \limits _{x \rightarrow \infty} \frac{1}{\epsilon x^{\epsilon}} \rightarrow 0.
  \end{equation}
\end{note}

\begin{definition}{Absolute Convergence and Conditional Convergence}{}
  Let $\int_a^bf(x)\mathrm{d}x$ be an improper integral,
  if the improper integral
  \begin{equation}
    \int_a^b |f(x)|\mathrm{d}x
  \end{equation}
  is convergent, then it is said to be \emph{absolutely convergent}.
  If $\int_a^bf(x)\mathrm{d}x$ is convergent but not absolutely convergent,
  then it is said to be \emph{conditionally convergent}.
\end{definition}

\begin{proposition}{Absolute Convergence Implies Convergence}{}
  If an improper integral $\int_a^b f(x)\mathrm{d} x$ is absolutely convergent,
  then it is convergent.
\end{proposition}

\begin{proof}
  Hint: $\int_a^b f(x)\mathrm{d} x \leq \int_a^b |f(x)|\mathrm{d} x$,
  and use the compare test.
\end{proof}


\subsection{Dirichlet and Abel Tests for Improper Integrals}

\begin{theorem}{Dirichlet Test}{}
  If $F(x) = \int_a^x f(t)\mathrm{d} t$ is bounded in $[a, +\infty)$,
  and $g(x)$ is decreasing monotonically to $0$ when $x$ goes up to infinity.
  Then
  \begin{equation}
    \int_a^{+\infty} f(x)g(x)\mathrm{d}x
  \end{equation}
  is convergent.
\end{theorem}

\begin{example}{Convergence of Imporper Integrals involving $\sin x$}{}
  Discuss the convergence of the following improper integrals
  \begin{equation}
    (1) \int_1^{+\infty} \frac{\sin x}{x^p}\mathrm{d}x, \quad
    (2) \int_0^{+\infty} \frac{\sin x}{x^p} \mathrm{d} x, \quad
    (3) \int_0^{+\infty} \sin x^2 \mathrm{d} x.
  \end{equation}
\end{example}

\begin{solution}
  (1) When $p \geq 1$, $\left| \frac{\sin x}{x^p} \right| \leq \frac{1}{x^p}$,
  and the comparison test implies it is absolutely convergent.
  When $0 < p \leq 1$, we know it is convergent by the Dirichlet test,
  now we prove that it is not absolutely convergent
  \begin{equation}
    \left| \frac{\sin x}{x^p} \right| \geq \frac{\sin^2 x}{x^p}
    = \frac{1}{2} \left( \frac{1}{x^p} - \frac{\cos 2x}{x^p} \right),
  \end{equation}
  since $\int_1^{+\infty} \frac{1}{x^p}\mathrm{d} x$ is not convergent,
  the improper integral is not absolutely convergent.
  When $p \leq 0$, take $u_1 = 2k\pi$ and $u_2 = 2k\pi + \frac{\pi}{2}$, we get
  \begin{equation}
    \left| \int_{u_1}^{u_2} \frac{\sin x}{x^p}\mathrm{d} x \right| \geq
    \int_{u_1}^{u_2} \sin x \mathrm{d} x = 1,
  \end{equation}
  so it is not convergent.

  (2) Notice $\int_0^{+\infty} f(x)\mathrm{d} x = \int_0^1 f(x)\mathrm{d} x +
  \int_1^{+\infty}f(x)\mathrm{d} x$, where the second term is discussed in (1).
  So we only consider the first term. When $x \rightarrow 0$, we know
  \begin{equation}
    \frac{\sin x}{x^p} \sim \frac{1}{x^{p-1}},
  \end{equation}
  which implies that when $p < 2$, it is absolutely convergent.
  When $p \geq 2$, it diverges.

  (3) Substitute $x$ with $\sqrt{t}$, the integral is 
  \begin{equation}
    \int_0^{+\infty} \sin x^2 \mathrm{d} x = \frac{1}{2} \int_0^{+\infty} \frac{\sin t}{\sqrt{t}}\mathrm{d} t,
  \end{equation}
  which is conditionally convergent.
\end{solution}

\begin{theorem}{Abel Test}{}
  If $\int_a^{+\infty} f(x)\mathrm{d} x$ is convergent,
  $g(x)$ is monotonic and bounded. Then
  \begin{equation}
    \int_a^{+\infty} f(x)g(x)\mathrm{d}x
  \end{equation}
  is convergent.
\end{theorem}

\subsection{Commonly Used Improper Integrals}

\begin{example}{Exponential and Trigonometric Functions}{}
  Calculate the following two improper integrals
  \begin{equation}
    (1) \int_0^{\infty} e^{-ax} \cos bx \mathrm{d} x, \quad
    (2) \int_0^{\infty} e^{-ax} \sin bx\mathrm{d} x.
  \end{equation}
\end{example}

\begin{solution}
  (1) We first find the indefinite integral of the function. By tabular method,
  we get
  \begin{equation}
    \int e^{-ax} \cos bx \mathrm{d} x
    = \frac{e^{-ax}}{b} \cos bx - \frac{a e^{-ax}}{b^2}\cos bx - \frac{a^2}{b^2}\int e^{-ax}\cos bx\mathrm{d}x,
  \end{equation}
  and it is $F(x) = \frac{e^{-ax}}{a^2 + b^2}(b \sin bx - a \cos bx)$.
  The value of the improper integral is $F(\infty) - F(0) = \frac{a}{a^2 + b^2}$.

  (2) The process is similar, the answer is $\frac{b}{a^2 + b^2}$.
\end{solution}

\begin{theorem}{Froullani Integrals}{}
  Let $f(x)$ be a continuous function on $[0, +\infty)$,
  $a, b$ be two positive real numbers.
  If $\lim \limits _{x \rightarrow +\infty}f(x)$ exists, then
  \begin{equation}
    \int_0^{+\infty} \frac{f(ax) - f(bx)}{x}\mathrm{d} x
    = [f(0) - f(+\infty)] \ln \frac{b}{a}.
  \end{equation}
  If there exists a positive real number $A$ such that
  $\int_A^{+\infty} \frac{f(x)}{x}\mathrm{d} x$ converges, then
  \begin{equation}
    \int_0^{+\infty} \frac{f(ax) - f(bx)}{x}\mathrm{d} x
    = f(0) \ln \frac{b}{a}.
  \end{equation}
\end{theorem}

\begin{example}{Examples of Froullani Integrals}{}
  Calculate the following improper integrals
  \begin{equation}
    (1) \int_0^{+\infty} \frac{\arctan(ax) - \arctan(bx)}{x}\mathrm{d} x, \quad
    (2) \int_0^{+\infty} \frac{e^{-ax} - e^{-bx}}{x}\mathrm{d} x, \quad
    (3) \int_0^{+\infty} \frac{\cos ax - \cos bx}{x} \mathrm{d} x.
  \end{equation}
\end{example}

\begin{solution}
  Apply the result of Froullani integral, we get 
  \begin{equation}
    (1) - \frac{\pi}{2} \ln \frac{b}{a}, \quad
    (2) \ln \frac{b}{a}, \quad
    (3) \ln \frac{b}{a},
  \end{equation}
  where (1)(2) are the first type of Froullani integral,
  (3) is the second type.
\end{solution}

\begin{theorem}{Dirichlet Integrals}{}
  Given an integer $p$, then
  \begin{equation}
    \int_0^{+\infty} \frac{\sin px}{x} \mathrm{d} x= \frac{\pi}{2} \operatorname{sgn}(p).
  \end{equation}
\end{theorem}

\begin{example}{Examples of Dirichlet Integrals}{}
  Calculate the following improper integrals
  \begin{equation}
    (1) \int_0^{+\infty} \frac{\sin x^2}{x}\mathrm{d} x, \quad
    (2) \int_0^{+\infty} \frac{\sin^2 x}{x^2} \mathrm{d} x.
  \end{equation}
\end{example}

\begin{solution}
  (1) Apply change of variable $t = x^2$:
  \begin{equation}
    \int_0^{\infty} \frac{\sin t}{\sqrt{t}}\mathrm{d} \sqrt{t}
    = \frac{1}{2} \int_0^{\infty} \frac{\sin t}{t}\mathrm{d} t
    = \frac{\pi}{4}.
  \end{equation}

  (2) First $\int_0^{\infty} \frac{\sin^2 x}{x^2}\mathrm{d} x =
  - \int_0^{\infty} \sin^2x \mathrm{d} \frac{1}{x}$, use the integration-by-part formula
  \begin{equation}
    - \int_0^{\infty} \sin^2 x \mathrm{d} \frac{1}{x}
    = - \frac{\sin^2 x}{x} \big|^{\infty}_0 + \int_0 ^{\infty} \frac{\sin 2x}{2x}\mathrm{d} 2x
    = \frac{\pi}{2}.
  \end{equation}
\end{solution}






