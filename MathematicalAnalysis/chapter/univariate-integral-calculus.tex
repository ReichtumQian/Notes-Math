
\section{Indefinite Integrals}

\subsection{Basic Methods of Indefinite Integrals}

\begin{proposition}{Integration-by-Parts Formula}{}
  The integral of two functions $u$ and $v$ satisfies
  \begin{equation}
    \int u \mathrm{d} v = uv - \int v \mathrm{d} u.
  \end{equation}
\end{proposition}

\begin{corollary}{Tabular Method}{}
  The integral of functions $u$ and $v$ satisfy
  \begin{equation}
    \int u v^{(n+1)} \mathrm{d} x
    = uv^{(n)} - u^{\prime}v^{(n-1)} + u^{\prime\prime} v^{(n-2)}
    + \cdots + (-1)^n u^{(n)}v + (-1)^{n+1} \int u^{(n+1)}v \mathrm{d} x.
  \end{equation}
\end{corollary}

The specific operation can be carried out in the following table:
\begin{table}[htbp]
  \centering
  \begin{tabular}{c|c|c|c|c|c|c}
    \hline
    $u$'s successive derivatives&$u$&$u^{\prime}$&$u^{\prime\prime}$&$u^{\prime\prime\prime}$&$\cdots$&$u^{(n+1)}$ \\ \hline
    $v^{(n+1)}$'s successive antiderivatives&$v^{(n+1)}$&$v^{(n)}$&$v^{(n-1)}$&$v^{(n-2)}$&$\cdots$&$v$\\ \hline
  \end{tabular}
\end{table}

\begin{note}
  How to choose $u$: inverse trigonometric functions,  logarithmic functions,
  power functions, exponential functions, trigonometric functions.
\end{note}

\begin{example}{Tabular Method}{}
  Calculate $\int (x^2 + 2x + 6) e^{2x}\mathrm{d} x$.
\end{example}

\begin{solution}
  Define $u = x^2 + 2x + 6$ and $v = e^{2x}$,
  by the tabular method we have
  \begin{equation}
    \int u v \mathrm{d} x = (x^2 + 2x + 6) \frac{1}{2}e^{2x} - (2x + 2) \frac{1}{4}e^{2x}
    + \frac{1}{4} e^{2x} - 0 = \frac{1}{2}x^2 e^{2x} + \frac{1}{2} xe^{2x} + \frac{11}{4}e^{2x}.
  \end{equation}
\end{solution}

\subsection{Commonly Used Indefinite Integral Formulas}

\subsection{Selected Examples from the Table of Indefinite Integrals}


\section{Definite Integrals}

\subsection{Integrability}

\begin{definition}{Riemann Integral}{}
  Let $f(x)$ be a function $[a, b]$, and divide $[a, b]$ into $n$ sub-intervals,
  representing the division by $T$.
  If
  \begin{equation}
    \lim \limits _{\|T\| \rightarrow 0} \sum\limits_{i = 1}^n f(\xi_i) \Delta x_i = J,
  \end{equation}
  where $J$ is a constant, then we say that $f(x)$ is \emph{integrable} on $[a, b]$,
  and denoted as
  \begin{equation}
    \int_a^b f(x) \mathrm{d} x = J.
  \end{equation}
\end{definition}

\begin{proposition}{Boundedness of Integrable Functions}{}
  If $f(x)$ is integrable on $[a, b]$, then it is bounded on $[a, b]$.
\end{proposition}

\begin{note}
  Note that bounded functions are not necessarily integrable, such as the
  Dirichlet function.
\end{note}

\begin{definition}{Darboux Upper and Lower Sum}{}
  On each sub-interval, take $M_i = \sup \limits_{x \in \Delta_i} f(x)$,
  and $m_i = \inf \limits_{x \in \Delta_i} f(x)$.
  Denote the oscillation $\omega_i = M_i - m_i$,
  then we define the \emph{Darboux upper sum} and \emph{Darboux lower sum} as:
  \begin{equation}
    S(T) := \sum\limits_{i = 1}^n M_i \Delta x_i, \quad
    s(T) := \sum\limits_{i = 1}^n m_i \Delta x_i.
  \end{equation}
\end{definition}

\begin{proposition}{Necessary and Sufficient Conditions for Integrability}{}
  A function $f(x)$ is integrable on $[a, b]$ if and only if
  for any $\epsilon > 0$, there exists a partition such that
  \begin{equation}
    S(T) - s(T) = \sum\limits_{i = 1}^n \omega_i \Delta x_i < \epsilon.
  \end{equation}
\end{proposition}

\begin{note}
  According to real analysis, we know that a function $f(x)$ is
  Riemann-integrable if and only if it is continuous almost everywhere.
\end{note}

\begin{example}{Non-integrability of the Dirichlet Function}{}
  Prove that Dirichlet function is not integrable on the interval $[a, b]$
  \begin{equation}
    D(x) =
    \begin{cases}
      1, & x \text{ is rational};\\
      0, & x \text{ is irrational}.
    \end{cases}
  \end{equation}
\end{example}

\begin{proof}
  For any partition $T$,
  the oscillation $\omega = 1$.
  So $S(T) - s(T) = \sum\limits_{i = 1}^n \Delta x_i = b - a$,
  which means that $f(x)$ is not integrable.
\end{proof}

\begin{example}{Integrability of the Riemann Function}{}
  Prove that the Riemann function $R(x)$ is integrable on $[0, 1]$,
  and satisfies
  \begin{equation}
    \int_0^1 R(x) \mathrm{d} x = 0.
  \end{equation}
\end{example}

\begin{proof}
  For any $\epsilon > 0$, there are only finitely many $x = \frac{p}{q}$ satisfying
  $\frac{1}{q} > \frac{\epsilon}{2}$,
  and their function values are less than or equal to $\frac{1}{2}$.
  List these finite points as $x_1,\cdots,x_k$.
  Each point can at most fall into two sub-intervals (when it is endpoint),
  so we take $\|T\| \leq \frac{\epsilon}{2k}$,
  then we have
  \begin{equation}
    \sum _T \omega_i \Delta x_i < \epsilon.
  \end{equation}
\end{proof}

\subsection{Newton-Leibniz Formula}

\begin{theorem}{Newton-Leibniz Formula}{}
  Let $f(x)$ be an integrable function on $[a, b]$,
  and $F(x)$ be a continuous function on $[a, b]$.
  If except for finite number of points,
  $F^{\prime}(x) = f(x)$, then
  \begin{equation}
    \int_a^b f(x) \mathrm{d} x = F(b) - F(a).
  \end{equation}
\end{theorem}

\begin{theorem}{Fundamental Theorem of Calculus}{}
  Let $f(x)$ be an integrable function on $[a, b]$.
  Define $F(x) = \int_a^x f(t)\mathrm{d} t$,
  where $x \in [a, b]$. Then
  \begin{itemize}
  \item $F(x)$ is a continuous function on $[a, b]$;
  \item If $x_0 \in [a, b]$ is a continuous point of $f(x)$,
    then $F(x)$ is differentiable at $x_0$,
    and $F^{\prime}(x_0) = f(x_0)$.
  \item If $f(x)$ is a continuous function on $[a, b]$,
    then $F(x)$ is a continuously-differentiable function on $[a, b]$,
    and $F^{\prime}(x) = f(x)$.
  \end{itemize}
\end{theorem}



\section{Improper Integrals}



