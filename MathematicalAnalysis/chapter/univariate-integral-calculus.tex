
\section{Indefinite Integrals}

\subsection{Basic Methods of Indefinite Integrals}

\begin{proposition}{Integration-by-Parts Formula}{}
  The integral of two functions $u$ and $v$ satisfies
  \begin{equation}
    \int u \mathrm{d} v = uv - \int v \mathrm{d} u.
  \end{equation}
\end{proposition}

\begin{corollary}{Tabular Method}{}
  The integral of functions $u$ and $v$ satisfy
  \begin{equation}
    \int u v^{(n+1)} \mathrm{d} x
    = uv^{(n)} - u^{\prime}v^{(n-1)} + u^{\prime\prime} v^{(n-2)}
    + \cdots + (-1)^n u^{(n)}v + (-1)^{n+1} \int u^{(n+1)}v \mathrm{d} x.
  \end{equation}
\end{corollary}

The specific operation can be carried out in the following table:
\begin{table}[htbp]
  \centering
  \begin{tabular}{c|c|c|c|c|c|c}
    \hline
    $u$'s successive derivatives&$u$&$u^{\prime}$&$u^{\prime\prime}$&$u^{\prime\prime\prime}$&$\cdots$&$u^{(n+1)}$ \\ \hline
    $v^{(n+1)}$'s successive antiderivatives&$v^{(n+1)}$&$v^{(n)}$&$v^{(n-1)}$&$v^{(n-2)}$&$\cdots$&$v$\\ \hline
  \end{tabular}
\end{table}

\begin{note}
  How to choose $u$: inverse trigonometric functions,  logarithmic functions,
  power functions, exponential functions, trigonometric functions.
\end{note}

\begin{example}{Tabular Method}{}
  Calculate $\int (x^2 + 2x + 6) e^{2x}\mathrm{d} x$.
\end{example}

\begin{solution}
  Define $u = x^2 + 2x + 6$ and $v = e^{2x}$,
  by the tabular method we have
  \begin{equation}
    \int u v \mathrm{d} x = (x^2 + 2x + 6) \frac{1}{2}e^{2x} - (2x + 2) \frac{1}{4}e^{2x}
    + \frac{1}{4} e^{2x} - 0 = \frac{1}{2}x^2 e^{2x} + \frac{1}{2} xe^{2x} + \frac{11}{4}e^{2x}.
  \end{equation}
\end{solution}

\subsection{Commonly Used Indefinite Integral Formulas}

\subsection{Selected Examples from the Table of Indefinite Integrals}


\section{Definite Integrals}

\subsection{Integrability}

\begin{definition}{Riemann Integral}{}
  Let $f(x)$ be a function $[a, b]$, and divide $[a, b]$ into $n$ sub-intervals,
  representing the division by $T$.
  If
  \begin{equation}
    \lim \limits _{\|T\| \rightarrow 0} \sum\limits_{i = 1}^n f(\xi_i) \Delta x_i = J,
  \end{equation}
  where $J$ is a constant, then we say that $f(x)$ is \emph{integrable} on $[a, b]$,
  and denoted as
  \begin{equation}
    \int_a^b f(x) \mathrm{d} x = J.
  \end{equation}
\end{definition}



\section{Improper Integrals}



