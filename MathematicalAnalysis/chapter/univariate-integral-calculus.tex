
\section{Indefinite Integrals}

\subsection{Basic Methods of Indefinite Integrals}

\begin{proposition}{Integration-by-Parts Formula}{}
  The integral of two functions $u$ and $v$ satisfies
  \begin{equation}
    \int u \mathrm{d} v = uv - \int v \mathrm{d} u.
  \end{equation}
\end{proposition}

\begin{corollary}{Tabular Method}{}
  The integral of functions $u$ and $v$ satisfy
  \begin{equation}
    \int u v^{(n+1)} \mathrm{d} x
    = uv^{(n)} - u^{\prime}v^{(n-1)} + u^{\prime\prime} v^{(n-2)}
    + \cdots + (-1)^n u^{(n)}v + (-1)^{n+1} \int u^{(n+1)}v \mathrm{d} x.
  \end{equation}
\end{corollary}

The specific operation can be carried out in the following table:
\begin{table}[htbp]
  \centering
  \begin{tabular}{c|c|c|c|c|c|c}
    \hline
    $u$'s successive derivatives&$u$&$u^{\prime}$&$u^{\prime\prime}$&$u^{\prime\prime\prime}$&$\cdots$&$u^{(n+1)}$ \\ \hline
    $v^{(n+1)}$'s successive antiderivatives&$v^{(n+1)}$&$v^{(n)}$&$v^{(n-1)}$&$v^{(n-2)}$&$\cdots$&$v$\\ \hline
  \end{tabular}
\end{table}

\begin{note}
  How to choose $u$: inverse trigonometric functions,  logarithmic functions,
  power functions, exponential functions, trigonometric functions.
\end{note}

\begin{example}{Tabular Method}{}
  Calculate $\int (x^2 + 2x + 6) e^{2x}\mathrm{d} x$.
\end{example}

\begin{solution}
  Define $u = x^2 + 2x + 6$ and $v = e^{2x}$,
  by the tabular method we have
  \begin{equation}
    \int u v \mathrm{d} x = (x^2 + 2x + 6) \frac{1}{2}e^{2x} - (2x + 2) \frac{1}{4}e^{2x}
    + \frac{1}{4} e^{2x} - 0 = \frac{1}{2}x^2 e^{2x} + \frac{1}{2} xe^{2x} + \frac{11}{4}e^{2x}.
  \end{equation}
\end{solution}

\subsection{Commonly Used Indefinite Integral Formulas}

\begin{proposition}{Commonly Used Indefinite Integral Formulas}{}
  The following formulas are commonly used and should be memorized.
  \begin{align}
      &\int \sin x \, \mathrm{d}x = -\cos x + c 
      && \int \cos x \, \mathrm{d}x = \sin x + c \\
      &\int \sec x \, \mathrm{d}x = \ln|\sec x + \tan x| + c 
      && \int \sec x \tan x \, \mathrm{d}x = \sec x + c \\
      &\int \sec^2 x \, \mathrm{d}x = \tan x + c 
      && \int \csc^2 x \, \mathrm{d}x = -\cot x + c \\
      &\int \frac{1}{x^2 + a^2} \, \mathrm{d}x = \frac{1}{a} \arctan \frac{x}{a} + c 
      && \int \frac{1}{x^2 - a^2} \, \mathrm{d}x = \frac{1}{2a} \ln \left| \frac{x-a}{x+a} \right| + c \\
      &\int \frac{1}{\sqrt{x^2 + a^2}} \, \mathrm{d}x = \ln \left| x + \sqrt{x^2 + a^2} \right| + c 
      && \int \frac{1}{\sqrt{x^2 - a^2}} \, \mathrm{d}x = \ln \left| x + \sqrt{x^2 - a^2} \right| + c \\
      &\int \frac{1}{\sqrt{a^2 - x^2}} \, \mathrm{d}x = \arcsin \frac{x}{a} + c 
      && {} \\
      &\int \ln x \, \mathrm{d}x = x \ln x - x + c 
      && {}
  \end{align}
\end{proposition}

\section{Definite Integrals}

\subsection{Integrability}

\begin{definition}{Riemann Integral}{}
  Let $f(x)$ be a function $[a, b]$, and divide $[a, b]$ into $n$ sub-intervals,
  representing the division by $T$.
  If
  \begin{equation}
    \lim \limits _{\|T\| \rightarrow 0} \sum\limits_{i = 1}^n f(\xi_i) \Delta x_i = J,
  \end{equation}
  where $J$ is a constant, then we say that $f(x)$ is \emph{integrable} on $[a, b]$,
  and denoted as
  \begin{equation}
    \int_a^b f(x) \mathrm{d} x = J.
  \end{equation}
\end{definition}

\begin{proposition}{Boundedness of Integrable Functions}{}
  If $f(x)$ is integrable on $[a, b]$, then it is bounded on $[a, b]$.
\end{proposition}

\begin{note}
  Note that bounded functions are not necessarily integrable, such as the
  Dirichlet function.
\end{note}

\begin{definition}{Darboux Upper and Lower Sum}{}
  On each sub-interval, take $M_i = \sup \limits_{x \in \Delta_i} f(x)$,
  and $m_i = \inf \limits_{x \in \Delta_i} f(x)$.
  Denote the oscillation $\omega_i = M_i - m_i$,
  then we define the \emph{Darboux upper sum} and \emph{Darboux lower sum} as:
  \begin{equation}
    S(T) := \sum\limits_{i = 1}^n M_i \Delta x_i, \quad
    s(T) := \sum\limits_{i = 1}^n m_i \Delta x_i.
  \end{equation}
\end{definition}

\begin{proposition}{Necessary and Sufficient Conditions for Integrability}{}
  A function $f(x)$ is integrable on $[a, b]$ if and only if
  for any $\epsilon > 0$, there exists a partition such that
  \begin{equation}
    S(T) - s(T) = \sum\limits_{i = 1}^n \omega_i \Delta x_i < \epsilon.
  \end{equation}
\end{proposition}

\begin{note}
  According to real analysis, we know that a function $f(x)$ is
  Riemann-integrable if and only if it is continuous almost everywhere.
\end{note}

\begin{example}{Non-integrability of the Dirichlet Function}{}
  Prove that Dirichlet function is not integrable on the interval $[a, b]$
  \begin{equation}
    D(x) =
    \begin{cases}
      1, & x \text{ is rational};\\
      0, & x \text{ is irrational}.
    \end{cases}
  \end{equation}
\end{example}

\begin{proof}
  For any partition $T$,
  the oscillation $\omega = 1$.
  So $S(T) - s(T) = \sum\limits_{i = 1}^n \Delta x_i = b - a$,
  which means that $f(x)$ is not integrable.
\end{proof}

\begin{example}{Integrability of the Riemann Function}{}
  Prove that the Riemann function $R(x)$ is integrable on $[0, 1]$,
  and satisfies
  \begin{equation}
    \int_0^1 R(x) \mathrm{d} x = 0.
  \end{equation}
\end{example}

\begin{proof}
  For any $\epsilon > 0$, there are only finitely many $x = \frac{p}{q}$ satisfying
  $\frac{1}{q} > \frac{\epsilon}{2}$,
  and their function values are less than or equal to $\frac{1}{2}$.
  List these finite points as $x_1,\cdots,x_k$.
  Each point can at most fall into two sub-intervals (when it is endpoint),
  so we take $\|T\| \leq \frac{\epsilon}{2k}$,
  then we have
  \begin{equation}
    \sum _T \omega_i \Delta x_i < \epsilon.
  \end{equation}
  This implies that the integral is zero.
\end{proof}

\begin{proposition}{Integrability of the Function after Elementary Operations}{}
  Suppose $f(x), g(x)$ are two integrable functions on $[a, b]$,
  then 
  \begin{equation}
    kf(x), \quad f(x) \pm g(x), \quad f(x)g(x)
  \end{equation}
  are integrable on $[a, b]$.
\end{proposition}

\begin{proof}
  The scalar multiplication and addition are direct,
  we only prove that $f(x)g(x)$ is integrable.
  Since both $f(x)$ and $g(x)$ are integrable, then they are all bounded.
  Assume $|f(x)| \leq M, |g(x)| \leq N$, then
  \begin{equation}
    \sum \omega_i^{fg} \Delta x_i \leq
    M \sum \omega_i^g \Delta x_i + N \sum \omega_i^f \Delta x_i.
  \end{equation}
  The limit is zero since both two terms go to zero.
\end{proof}

\begin{note}
  Note that even when $g(x) \neq 0$, the function
  \begin{equation}
    \varphi(x) = \frac{f(x)}{g(x)}
  \end{equation}
  is not necessarily integrable.
  For example $f(x) = 1$, and
  \begin{equation}
    g(x) =
    \begin{cases}
      x & x \in (0, 1];\\
      1 & x = 0.
    \end{cases}
  \end{equation}
  Then $f(x)/g(x)$ is unbounded, and therefore non-integrable.
\end{note}

\begin{proposition}{Integrability of Composition Functions}{}
  If $f(x)$ is continuous on the interval $[a, b]$,
  and $g(t)$ is integrable on the interval $[\alpha, \beta]$,
  satisfying $a \leq g(t) \leq b, t \in [\alpha, \beta]$,
  then
  \begin{equation}
    \varphi(t) = f(g(t))
  \end{equation}
  is integrable on $[\alpha, \beta]$.
\end{proposition}

\begin{proof}
  Since $f(x)$ is contunuous on a closed interval, then it is bounded and
  uniformly continuous. Let $M = \max f(x)$ and we have
  \begin{equation}
    \forall \epsilon > 0, \exists \delta > 0, \forall x_1, x_2, |x_1 - x_2| < \delta,
    |f(x_1) - f(x_2)| < \epsilon.
  \end{equation}
  Since $\varphi(x)$ is integrable, there exists a partition $T$ such that
  $\sum \limits_{\omega_i^{\varphi} \geq \delta} \Delta x_i < \epsilon$.
  Then
  \begin{align}
    \sum\limits_T \omega_i^{f \circ \varphi} \Delta t_i &= \sum\limits_{\omega_i^{\varphi} \geq \delta} \omega_i^{f \circ \varphi} \Delta t_i + \sum\limits_{\omega_i^{\varphi} < \delta} \omega_i^{f \circ \varphi} \Delta t_i\\
    &\leq 2M \sum\limits_{\omega_i^{\varphi} \geq \delta} \Delta t_i + \epsilon \sum\limits_{\omega_i^{\varphi} < \delta} \Delta t_i\\
    &< 2M \epsilon + (\beta - \alpha)\epsilon = \epsilon^{\prime}.
  \end{align}
  The conclusion follows from the condition for integrability.
\end{proof}

\subsection{Newton-Leibniz Formula}

\begin{theorem}{Newton-Leibniz Formula}{}
  Let $f(x)$ be an integrable function on $[a, b]$,
  and $F(x)$ be a continuous function on $[a, b]$.
  If except for finite number of points,
  $F^{\prime}(x) = f(x)$, then
  \begin{equation}
    \int_a^b f(x) \mathrm{d} x = F(b) - F(a).
  \end{equation}
\end{theorem}

\begin{theorem}{Fundamental Theorem of Calculus}{}
  Let $f(x)$ be an integrable function on $[a, b]$.
  Define $F(x) = \int_a^x f(t)\mathrm{d} t$,
  where $x \in [a, b]$. Then
  \begin{itemize}
  \item $F(x)$ is a continuous function on $[a, b]$;
  \item If $x_0 \in [a, b]$ is a continuous point of $f(x)$,
    then $F(x)$ is differentiable at $x_0$,
    and $F^{\prime}(x_0) = f(x_0)$.
  \item If $f(x)$ is a continuous function on $[a, b]$,
    then $F(x)$ is a continuously-differentiable function on $[a, b]$,
    and $F^{\prime}(x) = f(x)$.
  \end{itemize}
\end{theorem}

\begin{example}{Application of Fundamental Theorem of Calculus}{}
  Suppose a function $f(x)$ is integrable on the interval $[A, B]$,
  and $a, b \in (A, B)$ be two points at which $f(x)$ is continuous, prove that
  \begin{equation}
    \lim \limits _{h \rightarrow 0} \int_a^b \frac{f(x+h) - f(x)}{h}\mathrm{d} x = f(b) - f(a).
  \end{equation}
\end{example}

\begin{proof}
  Simplify the left-hand-side with the following steps
  \begin{align}
    \int_a^b \frac{f(x+h) - f(x)}{h}\mathrm{d} x &= \frac{1}{h} \left( \int_a^b f(x+h)\mathrm{d} x - \int_a^b f(x)\mathrm{d}x \right)\\
    &= \frac{1}{h} \left( \int_{a+h}^{b+h} f(x)\mathrm{d} x - \int_a^b f(x)\mathrm{d} x \right)\\
    &= \frac{1}{h} \left( \int _b^{b+h} f(x)\mathrm{d} x - \int_a^{a+h} f(x)\mathrm{d} x \right)
  \end{align}
  Define $F(x) = \int_A^xf(t)\mathrm{d}t$,
  then by the fundamental theorem of calculus
  \begin{equation}
    F^{\prime}(b) = \lim \limits _{h \rightarrow 0^+} \frac{F(b + h) - F(b)}{h} = f(b),
  \end{equation}
  similarly $F^{\prime}(a) = f(a)$.
\end{proof}

\subsection{The Mean Value Theorems for Integrals}

\begin{theorem}{The First Mean Value Theorem for Integrals}{}
  If $f(x), g(x)$ are continuous on $[a, b]$,
  and $g(x)$ does not change sign on $[a, b]$,
  then there exists $\xi \in (a, b)$ such that
  \begin{equation}
    \int_a^b f(x) g(x) \mathrm{d} x = f(\xi) \int_a^b g(x) \mathrm{d} x.
  \end{equation}
\end{theorem}

\begin{proof}
  Without loss of generality,
  assume $g(x) \geq 0$. Since $f(x)$ is continuous on $[a, b]$,
  we denote $M$ and $m$ the maximum and minimum values of $f(x)$ respectively.
  And we have
  \begin{equation}
    m g(x) \leq f(x)g(x) \leq Mg(x) \Rightarrow
    m \int_a^b g(x)\mathrm{d} x \leq \int_a^b f(x)g(x)\mathrm{d} x \leq M \int_a^b g(x)\mathrm{d} x.
  \end{equation}
  According to the intermediate-value property we can get the conclusion.
\end{proof}

\begin{theorem}{The Second Mean Value Theorem for Integrals}{}
  Suppose $f(x)$ is integrable on $[a, b]$ and $g(x)$ is monotonic on $[a, b]$.
  Then 
  \begin{itemize}
  \item If $g(x)$ is non-decreasing and non-negative: There exists $\xi \in (a,
    b)$ such that
    \begin{equation}
      \int_a^b f(x) g(x) \mathrm{d} x = g(b) \int_{\xi}^b f(x) \mathrm{d} x.
    \end{equation}
  \item If $g(x)$ is non-increasing and non-negative: There exists $\xi \in (a,
    b)$ such that
    \begin{equation}
      \int_a^b f(x) g(x) \mathrm{d} x = g(a) \int_{a}^{\xi} f(x) \mathrm{d} x.
    \end{equation}
  \end{itemize}
\end{theorem}

\section{Improper Integrals}

\subsection{The Concepts of Improper Integrals}

\begin{definition}{Improper Integrals over Infinite Intervals}{}
  Let $f$ be defined on $[a, +\infty)$.
  If the limit $\lim \limits _{b \rightarrow +\infty} \int_a^b f(x)\mathrm{d} x$
  exists,
  then we define
  \begin{equation}
    \int_a^{+\infty} f(x) \mathrm{d} x := \lim \limits _{b \rightarrow +\infty} \int_a^b f(x)\mathrm{d} x,
  \end{equation}
  and call it \emph{an improper integral over an infinite interval}.
\end{definition}

\begin{definition}{Improper Integrals with Singular Points}{}
  Let $f(x)$ be defined $(a, b]$,
  and $\lim \limits _{x \rightarrow a^+}f(x) = \infty$.
  If the limit $\lim \limits _{\epsilon \rightarrow 0^+} \int_{a+\epsilon}^b
  f(x) \mathrm{d} x$ exists,
  then we define
  \begin{equation}
    \int_a^b f(x) \mathrm{d} x := \lim \limits _{\epsilon \rightarrow 0^+} \int_{a+\epsilon}^b
  f(x) \mathrm{d} x
  \end{equation}
  and call it \emph{an improper integral with a singular point $a$}.
\end{definition}

\begin{theorem}{Leibniz Formula for Improper Integrals}{}
  Suppose the limits exist, and $f$ has an antiderivative $F$.
  Then we have
  \begin{equation}
    \int_a^{+\infty} f(x) \mathrm{d} x = F(+\infty) - F(a), \quad
    \int_a^b f(x) \mathrm{d} x = F(b) - F(a+0).
  \end{equation}
\end{theorem}

\begin{example}{Improper Integrals of $\frac{1}{x^p}$}{}
  Prove that (1) $\int_0^1 \frac{\mathrm{d} x}{x^p}$ converges when $p < 1$,
  and diverges when $p \geq 1$.
  (2) $\int_1^{+\infty} \frac{\mathrm{d}x}{x^p}$ converges when $p > 1$
  and diverges when $p \leq 1$.
\end{example}

\begin{proof}
  (1) We first find the antiderivative of $x^{-p}$:
  \begin{equation}
    F(x) = \int x^{-p}\mathrm{d} x =
    \begin{cases}
      \frac{1}{1-p} x^{1-p}, & p \neq 1;\\
      \ln x, & p =1.
    \end{cases}
  \end{equation}
  We observe that when $p < 1$, $F(1) = \frac{1}{1-p}$, and $F(\epsilon) =
  \frac{1}{1-p} \epsilon^{1-p}$. Therefore
  \begin{equation}
    \lim \limits _{\epsilon \rightarrow 0^+} F(1) - F(\epsilon) = \frac{1}{1-p},
  \end{equation}
  the limit exists. Similarly, it is not hard to prove that when $p \geq 1$, the
  limit does not exist.
\end{proof}

\subsection{Compare Tests for Improper Integrals}


\subsection{Dirichlet and Abel Tests for Improper Integrals}



