

\section{Measurable Functions}

\subsection{Concept of Measurable Functions}

\begin{definition}{Measurable Function}{}
  Let $f$ be a function defined on a measurable set $D$.
  If for any real number $\alpha$, the set
  \begin{equation}
    \{x \in D: f(x) > \alpha\}
  \end{equation}
  is measurable, then $f$ is said to be a \emph{measurable function on $D$}.
\end{definition}

\begin{note}
  It is equivalent to substitute the set in the definitio of measurable function with
  $\{f \geq \alpha\}$, $\{f = \alpha\}$, $\{f \leq \alpha\}$, $\{f < \alpha\}$.
\end{note}

\begin{proposition}{Closure of Measurable Functions under Operations}{}
  Let $f$, $g$, and $f_n$ be measurable functions, then
  \begin{equation}
    f \pm g, \quad f \times g, \quad f / g, \quad \sup f_n, \quad \inf f_n, \quad
    \limsup \limits_{n \rightarrow \infty}f_n, \quad
    \liminf \limits_{n \rightarrow \infty}f_n
  \end{equation}
  are measurable.
\end{proposition}

\subsection{Approximation by Simple Functions}

\begin{definition}{Simple Function}{}
  Let $f$ be a function defined on a measurable set $D$, satisfying
  \begin{equation}
    f(x) = a_k, \quad x \in E_k,
  \end{equation}
  where $a_k$ is a constant and $D = \cup _{k = 1}^n E_k$.
  Then $f$ is said to be a \emph{simple function on $D$}.
\end{definition}

\begin{theorem}{Measurable Function Approximated by Simple Functions}{}
  Let $f$ be a measurable function on $D$,
  then there exists a sequence of simple functions $\{f_n\}$ such that
  \begin{equation}
    \lim \limits _{n \rightarrow \infty} f_n(x) = f(x), \quad x \in D.
  \end{equation}
\end{theorem}

\subsection{Approximation by Continuous Functions}

\begin{theorem}{Egoroff Theorem (Pointwise Convergence vs Uniform Convergence)}{}
  Let $f$ and $f_n$ be measurable functions that are almost-everywhere finite
  on a set $D$ of finite measure.
  If $f_n \aeconv f$ on $D$,
  then
  \begin{equation}
    \forall \epsilon > 0, \exists F \subset D, m(D - F) < \epsilon, \quad
    f_n(x) \unifconv f(x), x \in F.
  \end{equation}
  where $F$ is a closed subset of $D$.
\end{theorem}

\begin{note}
  Egoroff theorem indicates that pointwise convergence is basically uniform convergence.
\end{note}

\begin{theorem}{Lusin Theorem (Measurable Function vs Continuous Function)}{}
  Let $f$ be a almost-everywhere finite measurable function on a measurable set
  $D$.
  Then
  \begin{equation}
    \forall \epsilon > 0, \exists f^{\ast} \in C(D), \quad
    m(\{f \neq f^{\ast}\}) < \epsilon.
  \end{equation}
\end{theorem}

\begin{note}
  Lusin theorem indicates that measurable functions are basically continuous functions.
\end{note}

\section{Convergence in Measure}

\begin{definition}{Convergence in Measure}{}
  Let $f$ and $f_n$ be measurable functions that are finite almost everywhere on
  a measurable set $D$. If for any $\delta > 0$,
  \begin{equation}
    \lim \limits _{n \rightarrow \infty} m(\{|f_n - f| \geq \delta\}) = 0,
  \end{equation}
  then we say that $f_n$ \emph{converges to $f$ in measure on $D$},
  denoted as $f_n \measconv f$.
\end{definition}

\begin{theorem}{Riesz Theorem (Convergence in Measure to Pointwise Convergence)}{}
  Let $f$ and $f_n$ be measurable functions that are finite almost everywhere on
  a measurable set $D$. Then
  \begin{equation}
    f_n \measconv f \quad \Rightarrow \quad \exists \{f_{n_k}\} \subset \{f_n\}, f_{n_k} \aeconv f.
  \end{equation}
\end{theorem}

\begin{theorem}{Lebesgue Theorem (Pointwise Convergence to Convergence in Measure)}{}
  Let $f$ and $f_n$ be measurable functions that are finite almost everywhere on
  a finite-measure measurable set $D$. Then
  \begin{equation}
    f_n \aeconv f \quad \Rightarrow \quad f_n \measconv f.
  \end{equation}
\end{theorem}










