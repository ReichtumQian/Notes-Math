

\section{Measurable Functions}

\subsection{Concept of Measurable Functions}

\begin{definition}{Measurable Function}{}
  Let $f$ be a function defined on a measurable set $D$.
  If for any real number $\alpha$, the set
  \begin{equation}
    \{x \in D: f(x) > \alpha\}
  \end{equation}
  is measurable, then $f$ is said to be a \emph{measurable function on $D$}.
\end{definition}

\subsection{Approximation by Simple Functions}

\subsection{Approximation by Continuous Functions}

\begin{theorem}{Egoroff Theorem}{}
  Let $f$ and $f_n$ be measurable functions that are almost-everywhere finite
  on a set $D$ of finite measure.
  If $f_n$ converges almost-everywhere to $f$ on $D$,
  then for any $\epsilon > 0$, there exists a closed subset $F$ of $D$,
  such that $m(D - F) < \epsilon$
  and $f_n$ converges uniformly to $f$ on $F$.
\end{theorem}

\begin{theorem}{Lusin Theorem}{}
  Let $f$ be a almost-everywhere finite measurable function on a measurable set
  $D$.
  Then for any $\epsilon > 0$, there exists a continuous function $f^{\ast}$ on
  $D$ such that
  \begin{equation}
    m(\{f \neq f^{\ast}\}) < \epsilon, \quad
    \sup \limits_{x \in D} |f^{\ast}(x)| \leq \sup \limits_{x \in D} |f(x)|.
  \end{equation}
\end{theorem}


\section{Convergence in Measure}












