

\section{Lebesgue Integral}

\subsection{Integral of Non-Negative Simple Functions}

\begin{definition}{Characteristic Function}{}
  Let $X$ be a set, and let $E \subset X$.
  The \emph{characteristic function of $E$} is
  \begin{equation}
    \chi_E(x) :=
    \begin{cases}
      1 & \text{if } x \in E;\\
      0 & \text{if } x \not \in E.
    \end{cases}
  \end{equation}
\end{definition}

\begin{definition}{Lebesgue Integral of Non-Negative Simple Functions}{}
  Let $f = \sum _{i = 1}^s a_i \chi_{E_i}(x)$ be a simple function,
  then \emph{Lebesgue integral of $f$} over $D = \cup _{i = 1}^s E_i$ is
  \begin{equation}
    \int_D f(x)\mathrm{d} x = \sum\limits_{i = 1}^s a_i m(E_i).
  \end{equation}
\end{definition}

\subsection{Integral of Non-negative Measurable Functions}

\begin{definition}{Lebesgue Integral of Non-negative Measurable Function}{}
  Let $f$ be a non-negative measurable function on a measurable set $D$,
  $f_n$ be a sequence of non-negative simple functions that increasingly
  converges to $f$ for each $x \in D$.
  Then the \emph{Lebesgue integral of $f$ on $D$} is
  \begin{equation}
    \int_D f \mathrm{d} x= \lim \limits _{n \rightarrow \infty} \int_D f_n \mathrm{d} x.
  \end{equation}
\end{definition}

\begin{theorem}{Levi's Monotone Convergence Theorem}{}
  Let $f$ and $f_n$ be non-negative measurable functions on a measurable set
  $D$.
  If for any $x \in D$, $f_n(x) \aeconv f(x)$ non-decreasingly, then
  \begin{equation}
    \int_D f(x)\mathrm{d} x = \lim \limits _{n \rightarrow \infty} \int_D f_n\mathrm{d} x.
  \end{equation}
\end{theorem}

\begin{theorem}{Fatou Theorem}{}
  Let $f_n$ be non-negative measurable functions on a measurable set $D$,
  then
  \begin{equation}
    \int_D \liminf \limits_{n \rightarrow \infty} f_n\mathrm{d} x
    \leq \liminf \limits_{n \rightarrow \infty} \int_D f_n\mathrm{d} x.
  \end{equation}
\end{theorem}

\subsection{Integral of Normal Measurable Functions}

\begin{definition}{Lebesgue Integral}{}
  Let $f$ be a measurable function on a measurable set $D$,
  and $f_+(x) := \max\{0, f(x)\}$, $f_-(x) := \max\{0, -f(x)\}$.
  Then the \emph{lebesgue integral of $f$ over $D$} is
  \begin{equation}
    \int_D f \mathrm{d} x 
    = \int_D f_+ \mathrm{d} x - \int_D f_-\mathrm{d} x
    := I^+ - I^-.
  \end{equation}
  If $I^+, I^- < \infty$, then $f$ is said to be \emph{lebesgue integrable over
    $D$}, denoted as $f \in L(D)$.
\end{definition}

\begin{theorem}{Countable Additivity of the Lebesgue Integral}{}
  Let $f$ be Lebesgue integrable over $D$,
  and $\{E_k\}$ be a partition of $D$. Then
  \begin{equation}
    \int_D f \mathrm{d} x = \sum\limits_{k = 1}^{\infty} \int_{E_k} f \mathrm{d} x.
  \end{equation}
\end{theorem}

\begin{theorem}{Absolute Continuity of the Lebesgue Integral}{}
  Let $f$ be Lebesgue integrable over $D$,
  then
  \begin{equation}
    \forall \epsilon > 0, \exists \delta > 0, \forall A \subset D,
    m(A) < \delta, \left| \int_A f \mathrm{d} x \right| < \epsilon.
  \end{equation}
\end{theorem}

\begin{theorem}{Lebesgue's Dominated Convergence Theorem (LDCT)}{}
  Let $f$ and $f_n$ be measurable functions on a measurable set $D$.
  If $f_n(x) \aeconv f(x)$ and there exists $g \in L(D)$,
  $|f_n(x)| \leq g(x)$ a.e. for any $n \geq 1$.
  Then $f, f_n \in L(D)$,
  \begin{equation}
    \lim \limits _{n \rightarrow \infty} \int_D f_n\mathrm{d} x
    = \int_D f\mathrm{d}x.
  \end{equation}
\end{theorem}

\section{Riemann Integral}

\begin{theorem}{Condition for Riemann Integral}{}
  Let $f(x)$ be a bounded function on $[a, b]$.
  Then $f(x)$ is Riemann integrable if and only if
  it is continuous a.e. on $[a, b]$.
\end{theorem}

\section{Multiple Integral and Iterated Integral}

\begin{theorem}{Fubini Theorem}{}
  Let $f(x, y)$ be integrable over $\mathbb{R}^p \times \mathbb{R}^q$. Then
  for $x, y$ a.e., $f(x,y)$ is integrable with respect to both $y$ and $x$
  respectively,
  \begin{equation}
    \int_{\mathbb{R}^p \times \mathbb{R}^q} f(x,y)\mathrm{d} x\mathrm{d}y 
    = \int_{\mathbb{R}}^p \left[ \int_{\mathbb{R}^q} f(x,y)\mathrm{d} y \right]\mathrm{d} x.
  \end{equation}
\end{theorem}


