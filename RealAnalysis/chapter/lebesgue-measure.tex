
\section{Lebesgue Measure and Measurable Sets}

\subsection{Lebesgue Outer Measure}

\begin{definition}{Lebesgue Outer Measure}{}
  For every subset $E$ of the set of real numbers,
  let $\{I_n\}$ be any sequence of open intervals satisfying
  $E \subset \cup _{n = 1}^{\infty} I_n$. Then we say
  \begin{equation}
    m^{\ast} (E) = \inf_{\{I_n\}} \left\{\sum_{n = 1}^{\infty} \ell(I_n)\right\}
  \end{equation}
  the \emph{lebesgue outer measure of $E$},
  where $\ell(I_n)$ is the length of the interval $I_n$.
\end{definition}

\begin{proposition}{Countable Sub-additivity}{}
  If $\{E_n\}$ is an arbitrary sequence of subsets of real numbers,
  then
  \begin{equation}
    m^{\ast} (\cup _{n = 1}^{\infty} E_n) \leq \sum _{n = 1}^{\infty} m^{\ast}(E_n).
  \end{equation}
\end{proposition}

\begin{proof}
  If $\sum_n m^{\ast}(E_n) = \infty$, then the conclusion holds naturally.
  Otherwise, for all $\epsilon > 0$, there exists $\{I_k^{(n)}\}$ such that
  $E_n \subset \cup_k I_k^{(n)}$ and
  \begin{equation}
    \sum \limits _k \ell(I_k^{(n)}) < m^{\ast}(E_n) + \frac{\epsilon}{2^n}.
  \end{equation}
  Since $\cup _n E_n \subset \cup _n \cup _k I_k^{(n)}$, then
  \begin{equation}
    m^{\ast}(\cup _n E_n)
    \leq \sum _n \sum _k \ell(I_k^{(n)})
    < \sum _n \left[ m^{\ast}(E_n) + \frac{\epsilon}{2^n} \right]
    = \sum _n m^{\ast}(E_n) + \epsilon.
  \end{equation}
  By the arbitrariness of $\epsilon$, the conclusion holds.
\end{proof}


\subsection{Lebesgue Measure}

\begin{definition}{Lebesgue Measurable Set}{}
  Let $E$ be a subset of real numbers.
  If for any subset $A$ of real numbers,
  \begin{equation}
    m^{\ast}(A) = m^{\ast}(A \cap E) + m^{\ast}(A \cap E^c).
  \end{equation}
  Then $E$ is said to be a \emph{lebesgue measurable set},
  and
  \begin{equation}
    m(A) := m^{\ast}(A)
  \end{equation}
  is said to be the \emph{measure of $A$}.
\end{definition}

\begin{note}
  Hereafter, we use the notation $\Omega$ to represent the set of measurable sets.
\end{note}

\begin{example}{Practice for Lebesgue Measure}{}
  Prove the following statements:
  \begin{enumerate}
  \item If $m^{\ast}(A) = 0$, then $A$ is measurable and $m(A) = 0$;
  \item If $m(A) = 0$ and $A$ is an open set, then $A = \emptyset$.
  \end{enumerate}
\end{example}

\begin{proof}
  (1) For any set of real numbers $A$,
  \begin{equation}
    0 \leq m^{\ast}(A \cap E) \leq m^{\ast}(E) = 0,
  \end{equation}
  thus $m^{\ast}(A \cap E) = 0$. On the other hand, $A \supset A \cap E^c$,
  then
  \begin{equation}
    m^{\ast}(A) \geq m^{\ast}(A \cap E^c) = m^{\ast}(A \cap E^c) + m^{\ast}(A \cap E),
  \end{equation}
  which implies that $A$ is measurable and its measure is zero.
\end{proof}

\begin{example}{Example of Non-measurable Set}{}
  Prove that for any $x \in [0, 1]$, the following set is non-measurable
\end{example}

\begin{proof}
  
\end{proof}

\subsection{Lebesgue Measurable Sets in Set Operations}

