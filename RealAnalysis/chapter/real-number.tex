
\section{Sets and Mappings}

\subsection{Sets and Operations of Sets}

\begin{definition}{Operations of Sets}{}
  Let $A$, $B$ be two sets,
  and $X$ the universal set.
  Define the following operations
  \begin{equation}
    A \cup B := \{x: x \in A \text{ or } x \in B\}, \quad
    A \cap B := \{x: x \in A, x \in B\};
  \end{equation}
  \begin{equation}
    A - B := \{x: x \in A, x \not\in B\}, \quad
    A \Delta B := (A \cup B) - (A \cap B), \quad
    A^c := X - A
  \end{equation}
  as the \emph{union, intersection, difference, symmetric difference, complement} respectively.
\end{definition}

\begin{proposition}{Properties of Operations of Sets}{}
  The operations of sets have the following properties:
  \begin{enumerate}
  \item $(A \cup B)\cap C = (A \cap C) \cup (B \cap C)$,
    $(A \cap B) \cup C = (A \cup C) \cap (B \cup C)$.
  \item $(A\cup B)^c = A^c \cap B^c$, $(A \cap B)^c = A^c \cup B^c$.
  \end{enumerate}
\end{proposition}

\begin{proof}
  
\end{proof}

\begin{definition}{Limits of Set Sequences}{}
  Let $\{A_n\}$ be a sequence of sets,
  define
  \begin{equation}
    \limsup \limits_{n \rightarrow \infty} A_n := \bigcap \limits_{n = 1}^{\infty}
    \bigcup \limits_{k = n} ^{\infty} A_k, \quad
    \liminf \limits_{n \rightarrow \infty} A_n := \bigcup \limits_{n = 1}^{\infty}
    \bigcap \limits_{k = n}^{\infty}A_k.
  \end{equation}
  as the \emph{upper limit and lower limit of $A_n$} respectively.
  If they are equal, then $\{A_n\}$ is said to have a \emph{limit}.
\end{definition}

\begin{proposition}{Properties of Set Sequence Limits}{}
  Let $\{A_n\}$ be a sequence of sets, then
  \begin{enumerate}
  \item $x \in \limsup \limits_{n \rightarrow \infty} A_n$ if and only if
    $\forall N > 0$, $\exists n \geq N$, $x \in A_n$.
  \item $x \in \liminf \limits_{n \rightarrow \infty} A_n$ if and only if
    $\exists N_x$, $\forall n \geq N_x$, $x \in A_n$.
  \item $\liminf \limits_{n \rightarrow \infty} A_n \subset \limsup \limits_{n
      \rightarrow \infty} A_n$.
  \end{enumerate}
\end{proposition}

\begin{proof}
  We only prove the first proposition, the second is similar,
  while the third is the corollary of the first and second.
  If $x \in \limsup\limits_{n \rightarrow \infty}A_n$,
  then for all $N > 0$, $x \in \bigcup \limits_{k = N}^{\infty}A_k$,
  that is, there exists $n \geq N$ such that $x \in A_n$.
  The reverse is also similar.
\end{proof}

\begin{example}{Examples of Set Sequence Limits}{}
  Let $A_n = \{\frac{m}{n}\}$, where $m$ is an integer. Prove
  \begin{equation}
    \limsup \limits_{n \rightarrow \infty} A_n = \mathbb{Q}, \quad
    \liminf \limits_{n \rightarrow \infty} A_n = \mathbb{Z}.
  \end{equation}
\end{example}

\begin{proof}
  For each $A_n$, it is obvious that $\mathbb{Z} \subset A_n \subset
  \mathbb{Q}$, then
  \begin{equation}
    \mathbb{Z} \subset \liminf \limits_{n \rightarrow \infty} A_n, \quad
    \limsup \limits_{n \rightarrow \infty} A_n \subset \mathbb{Q}.
  \end{equation}

  For all $x \in \liminf \limits_{n \rightarrow \infty} A_n$, there exists $n$
  such that $x \in A_n \cap A_{n+1}$.
  Then there exist integers $m_n$ and $m_{n+1}$ such that
  \begin{equation}
    x = \frac{m_n}{n} = \frac{m_{n+1}}{n+1} \Rightarrow x = m_{n+1} - m_n
  \end{equation}
  which is an integer, then $\liminf \limits_{n \rightarrow \infty} A_n \subset \mathbb{Z}$.

  For all $\frac{p}{q} \in \mathbb{Q}$ and $n \geq 1$,
  \begin{equation}
    \frac{q}{p} = \frac{nq}{np} \in \cup _{k = n}^{\infty} A_k
    \Rightarrow
    \frac{q}{p} \in \cap _{n = 1}^{\infty} \cup _{k = n}^{\infty}A_k,
  \end{equation}
  which means $\mathbb{Q} \subset \limsup \limits_{n \rightarrow \infty}A_n$.
\end{proof}



\subsection{Mappings}



\subsection{Cardinality of Sets}

\begin{definition}{Equivalence}{}
  If there exists a one-to-one and onto mapping between set $A$ and set $B$,
  then we say that $A$ and $B$ are \emph{equivalent}, denoted as
  \begin{equation}
    A \sim B.
  \end{equation}
\end{definition}

\begin{definition}{Set with Cardinality of Continuum}{}
  If a set $A$ is equivalent to $[0, 1]$,
  then we say it is a \emph{set with cardinality of continuum}.
\end{definition}

\section{The Sets of Real Numbers}

\subsection{Open Sets and Closed Sets}

\begin{definition}{Open Set}{}
  Let $E \subset \mathbb{R}^n$, if $E$ is the neighborhood of every point in
  $E$,
  then it is said to be an \emph{open set}.
\end{definition}

\begin{definition}{Closed Set}{}
  Let $E \subset \mathbb{R}^n$, if $E^c$ is an open set,
  then $E$ is said to be a \emph{closed set}.
\end{definition}

\subsection{Interior and Closure}





\subsection{Accumulation Points, Isolated Points and Perfect Sets}

\begin{definition}{Accumulation Point}{}
  Let $E \subset \mathbb{R}^n$ and $x \in \mathbb{R}^n$.
  If any neighborhood $V$ of $x$ satisfies
  \begin{equation}
    (V - \{x\}) \cap E \neq \emptyset,
  \end{equation}
  then $x$ is called an \emph{accumulation point of $E$},
  and the set of all accumulation points of $E$ is called the \emph{derived set
    of $E$}, denoted as $E^{\prime}$.
\end{definition}

\begin{example}{Practice for Derived Set}{}
  Let $A$ be a closed set in $\mathbb{R}^n$,
  determine the validity of the following statements:
  \begin{equation}
    (1) A = A^{\prime}, \quad
    (2) A^{\prime} \subset A.
  \end{equation}
\end{example}

\begin{solution}
  (1) Suppose $A$ consists of a single point,
  then $A^{\prime} = \emptyset$, thus the conclusion does not hold.

  (2) Correct. Since a closed set must contain all its accumulation points.
\end{solution}

\begin{definition}{Isolated Point}{}
  Let $E \subset \mathbb{R}^n$ and $x \in \mathbb{R}^n$.
  If $x \in E$ but $x$ is not an accumulation point of $E$,
  then it is called an \emph{isolated point of $E$}.
\end{definition}

\begin{definition}{Perfect Set}{}
  If a set $E \subset \mathbb{R}^n$ is closed and does have any isolated point,
  then it is called a \emph{perfect set}.
\end{definition}



\subsection{Cantor Set}

\begin{definition}{Cantor Set}{}
  Let $C_0 = [0, 1]$, and for each $n \geq 1$, define $C_n$ by removing the open
  middle third from each closed interval in $C_{n-1}$.
  Then the \emph{cantor set} is
  \begin{equation}
    \mathcal{C} = \cap _{n = 0}^{\infty}C_n.
  \end{equation}
\end{definition}


