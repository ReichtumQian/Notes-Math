
\section{Sets and Mappings}

\subsection{Sets and Operations of Sets}

\begin{definition}{Operations of Sets}{}
  Let $A$, $B$ be two sets,
  and $X$ the universal set.
  Define the following operations
  \begin{equation}
    A \cup B := \{x: x \in A \text{ or } x \in B\}, \quad
    A \cap B := \{x: x \in A, x \in B\};
  \end{equation}
  \begin{equation}
    A - B := \{x: x \in A, x \not\in B\}, \quad
    A \Delta B := (A \cup B) - (A \cap B), \quad
    A^c := X - A
  \end{equation}
  as the \emph{union, intersection, difference, symmetric difference, complement} respectively.
\end{definition}

\begin{proposition}{Properties of Operations of Sets}{}
  The operations of sets have the following properties:
  \begin{enumerate}
  \item Distributive Law: $(A \cup B)\cap C = (A \cap C) \cup (B \cap C)$,
    $(A \cap B) \cup C = (A \cup C) \cap (B \cup C)$.
  \item De Morgan's Law: $(A\cup B)^c = A^c \cap B^c$, $(A \cap B)^c = A^c \cup B^c$.
  \end{enumerate}
\end{proposition}

\begin{theorem}{De Morgan Formula}{}
  Let $\{A_{\lambda}: \lambda \in \Lambda\}$ be a family of subsets of $X$, then
  \begin{equation}
    \left(\cup_{\lambda\in\Lambda}A_{\lambda}\right)^{c}=\cap_{\lambda\in\Lambda}A_{\lambda}^{c},\quad\left(\cap_{\lambda\in\Lambda}A_{\lambda}\right)^{c}=\cup_{\lambda\in\Lambda}A_{\lambda}^{c}.
  \end{equation}
\end{theorem}

\begin{example}{Operations on Countable Sets}{}
  Let $f, g$ be mappings from $E$ to $\mathbb{R}$, where $E \subset \mathbb{R}$.
  Prove that
  \begin{equation}
    \{x: f(x) > g(x) \} = \cup _{n=1}^{+\infty} \{x: f(x) > g(x) + \frac{1}{n}\},
  \end{equation}
  \begin{equation}
    \{x: f(x) \geq g(x)\} = \cap _{n = 1}^{\infty} \{x: f(x) > g(x) - \frac{1}{n}\}.
  \end{equation}
\end{example}

\begin{proof}
  (1) For all $x$ from the lhs set,
  then $f(x) > g(x)$. There must exist $n$ such that $f(x) > g(x) + \frac{1}{n}$,
  which indicates $x$ is contained in the rhs set.
  Similarly, for all $x$ from the rhs set, there exists $n$ such that $x \in A_n$,
  which means $f(x) > g(x) + \frac{1}{n} > g(x)$.

  (2) Left-to-right is easy. For all $x$ from the rhs set,
  we have $x \in A_n$ for all $n \geq 1$, so $g(x) < f(x) + \frac{1}{n}$.
  This indicates that $g(x) \leq f(x)$.
\end{proof}

\begin{definition}{Supremum and Infimum}{}
  Let $S \subseteq \mathbb{R}$. A real number $\eta$ is the \emph{supremum of $S$},
  denoted as $\eta = \sup S$, if
  \begin{equation}
    \forall x \in S, x \leq \eta; \quad
    \forall \epsilon > 0, \exists x \in S, x > \eta - \epsilon.
  \end{equation}
  $\eta$ is the \emph{infimum of $S$}, denoted as $\eta = \inf S$, if
  \begin{equation}
    \forall x \in S, x \geq \eta; \quad
    \forall \epsilon > 0, \exists x \in S, x < \eta + \epsilon.
  \end{equation}
\end{definition}

\begin{example}{Union and Intersection Form of Function Supremum and Infimum}{}
  Given a sequence of functions $\{f_n\}$, where $f_n: E \rightarrow \mathbb{R}$,
  $E \subset \mathbb{R}$. Prove that for all $c \in \mathbb{R}$,
  \begin{equation}
    \{x: \inf_n \{f_n(x)\} < c\} = \cup _{n = 1}^{\infty} \{x: f_n(x) < c\},
  \end{equation}
  \begin{equation}
    \{x: \inf_n \{f_n(x)\} \geq c\} = \cap _{n = 1}^{\infty} \{x: f_n(x) \geq c\}.
  \end{equation}
  \begin{equation}
    \{x: \sup_n \{f_n(x)\} \leq c\} = \cap _{n = 1}^{\infty} \{x: f_n(x) \leq c\},
  \end{equation}
  \begin{equation}
    \{x: \sup_n \{f_n(x)\} > c\} = \cup _{n = 1}^{\infty} \{x: f_n(x) > c\}.
  \end{equation}
\end{example}

\begin{proof}
  We only prove the infimum case.

  (1) For any $x$ from the lhs set, denote $A = \inf_n \{f_n(x)\}$, then
  $A < c$ and
  \begin{equation}
    \forall \epsilon > 0, \exists n_0 > 0,
    f_{n_0}(x) < A + \epsilon.
  \end{equation}
  Choose $\epsilon < c - A$, then there exists $n_0$ such that $f_{n_0}(x) < c$,
  therefore
  \begin{equation}
    x \in \{x: f_{n_0}(x) < c\} \subset \cup _{n = 1}^{\infty} \{x: f_n(x) < c\}.
  \end{equation}
  For any $x$ from the rhs set, there exists $n_0$ such that $f_{n_0}(x) < c$, then
  \begin{equation}
    \inf_n f_n(x) \leq f_{n_0}(x) < c \Rightarrow x \in \{x: \inf_n f_n(x) < c\}.
  \end{equation}

  (2) $\inf_n \{f_n\} \geq c$ indicates that
  \begin{equation}
    \forall n \in \mathbb{Z}^+, f_n(x) \geq c
    \Leftrightarrow x \in \cap_{n = 1}^{\infty} \{x: f_n(x) \geq c\}.
  \end{equation}
\end{proof}

\subsection{Limit of Set Sequences}

\begin{definition}{Limits of Set Sequences}{}
  Let $\{A_n\}$ be a sequence of sets,
  define
  \begin{equation}
    \limsup \limits_{n \rightarrow \infty} A_n := \cap _{n = 1}^{\infty}
    \cup _{k = n} ^{\infty} A_k, \quad
    \liminf \limits_{n \rightarrow \infty} A_n := \cup _{n = 1}^{\infty}
    \cap _{k = n}^{\infty}A_k.
  \end{equation}
  as the \emph{upper limit and lower limit of $A_n$} respectively.
  If they are equal, then $\{A_n\}$ is said to have a \emph{limit}.
\end{definition}

\begin{proposition}{Properties of Set Sequence Limits}{}
  Let $\{A_n\}$ be a sequence of sets, then
  \begin{enumerate}
  \item $x \in \limsup \limits_{n \rightarrow \infty} A_n$ if and only if
    $\forall N > 0$, $\exists n \geq N$, $x \in A_n$.
  \item $x \in \liminf \limits_{n \rightarrow \infty} A_n$ if and only if
    $\exists N_x$, $\forall n \geq N_x$, $x \in A_n$.
  \item $\liminf \limits_{n \rightarrow \infty} A_n \subset \limsup \limits_{n
      \rightarrow \infty} A_n$.
  \end{enumerate}
\end{proposition}

\begin{proof}
  We only prove the first proposition, the second is similar,
  while the third is the corollary of the first and second.
  If $x \in \limsup\limits_{n \rightarrow \infty}A_n$,
  then for all $N > 0$, $x \in \bigcup \limits_{k = N}^{\infty}A_k$,
  that is, there exists $n \geq N$ such that $x \in A_n$.
  The reverse is also similar.
\end{proof}

\begin{example}{Examples of Set Sequence Limits}{}
  \begin{enumerate}
  \item Let $A_{2n-1} = (0, \frac{1}{n})$ and $A_{2n} = (0, n)$, find
    $\limsup \limits_{n \rightarrow \infty} A_n$ and $\liminf \limits_{n \rightarrow \infty} A_n$.
  \item Let $A_n = \{\frac{m}{n}: m \in \mathbb{Z}\}$. Prove
    \begin{equation}
      \limsup \limits_{n \rightarrow \infty} A_n = \mathbb{Q}, \quad
      \liminf \limits_{n \rightarrow \infty} A_n = \mathbb{Z}.
    \end{equation}
  \end{enumerate}
\end{example}

\begin{proof}
  (1) By proposition, $\limsup \limits_{n \rightarrow \infty} A_n = (0, +\infty)$ and
  $\liminf \limits_{n \rightarrow \infty} A_n = \emptyset$.

  (2) For each $A_n$, we know $\mathbb{Z} \subset A_n \subset \mathbb{Q}$, so
  $\mathbb{Z} \subset \liminf \limits_{n \rightarrow \infty} A_n$ and
  $\limsup \limits_{n \rightarrow \infty} A_n \subset \mathbb{Q}$.
  For all $x \in \liminf \limits_{n \rightarrow \infty} A_n$, there exists $n$
  such that $x \in A_n \cap A_{n+1}$.
  Then there exist integers $m_n$ and $m_{n+1}$ such that
  \begin{equation}
    x = \frac{m_n}{n} = \frac{m_{n+1}}{n+1} \Rightarrow x = m_{n+1} - m_n
  \end{equation}
  which is an integer, then $\liminf \limits_{n \rightarrow \infty} A_n \subset \mathbb{Z}$.

  For all $\frac{p}{q} \in \mathbb{Q}$ and $n \geq 1$,
  \begin{equation}
    \frac{q}{p} = \frac{nq}{np} \in \cup _{k = n}^{\infty} A_k
    \Rightarrow
    \frac{q}{p} \in \cap _{n = 1}^{\infty} \cup _{k = n}^{\infty}A_k,
  \end{equation}
  which means $\mathbb{Q} \subset \limsup \limits_{n \rightarrow \infty}A_n$.
\end{proof}

\begin{proposition}{Limit of Monotone Set Sequences}{}
  If the set sequence $\{A_n\}$ is monotonically increasing or decreasing,
  then $\lim \limits _{n \rightarrow \infty} A_n$ exists, and
  \begin{equation}
    \lim \limits _{n \rightarrow \infty} A_n = \cup _{n = 1}^{\infty} A_n ~~  \text{if increasing},
    \quad \quad
    \lim \limits _{n \rightarrow \infty} A_n = \cap _{n = 1}^{\infty} A_n ~~  \text{if decreasing}.
  \end{equation}
\end{proposition}

\begin{proof}
  It is direct to verify that the upper limit and lower limit are identical,
  and find the limit.
\end{proof}

\begin{example}{Limits of Monotone Set Sequence}{}
  Let $\{f_n\}$ be a sequence of functions, $f_n : E \rightarrow \mathbb{R}$, $E \subset \mathbb{R}$.
  For all $x \in E$, $f_n(x) \leq f_{n+1}(x)$. Prove that for all $c \in \mathbb{R}$,
  the limit of set sequence $A_n = \{x: f_n(x) > c\}$ exists and 
  \begin{equation}
    \lim \limits _{n \rightarrow \infty} A_n = \left\{ x:\lim \limits _{n \rightarrow \infty} f_n(x) > c \right\}.
  \end{equation}
\end{example}

\begin{proof}
  It is obvious that $A_n$ is monotonically increasing, so the limit exists.
  Then
  \begin{equation}
    \lim \limits _{n \rightarrow \infty} A_n = \cup _{n = 1}^{+\infty}A_n.
  \end{equation}
  It is easy to prove that for all $x \in \cup _{n = 1}^{+\infty}A_n$,
  $x \in \{x: \lim \limits _{n \rightarrow \infty} f_n(x) > c\}$.
  And for all $x \in \{x: \lim \limits _{n \rightarrow \infty} f_n(x) > c\}$,
  there must exist a sufficiently large $N$ such that $f_N(x) > c$,
  so $x \in A_N$.
\end{proof}

\subsection{Mappings}

\begin{definition}{Injective Mapping}{}
  A mapping $f:A \rightarrow B$ is \emph{injective} if
  \begin{equation}
    f(x_1) = f(x_2) \Rightarrow x_1 = x_2.
  \end{equation}
\end{definition}

\begin{definition}{Surjective/Onto Mapping}{}
  A mapping $f:A \rightarrow B$ is \emph{surjective} if
  \begin{equation}
    \forall y \in B, \exists x \in A, \quad f(x) = y.
  \end{equation}
\end{definition}

\begin{definition}{Bijective Mapping}{}
  A mapping $f:A \rightarrow B$ is \emph{bijective} if it is both injective and surjective.
\end{definition}

\subsection{Cardinality of Sets}

\begin{definition}{Equivalence}{}
  If there exists a bijective mapping between set $A$ and set $B$,
  then we say that $A$ and $B$ are \emph{equivalent}, denoted as
  \begin{equation}
    A \sim B.
  \end{equation}
\end{definition}

\begin{definition}{Countable Set and Uncountable Set}{}
  A set $S$ is \emph{countable} if it is either finite or there
  exists a bijection between $S$ and $\mathbb{N}$.
  If $S$ is not countable then we say it is \emph{uncountable}.
\end{definition}

\begin{definition}{Cardinality of Continuum}{}
  If a set $A$ is equivalent to $[0, 1]$,
  then we say it is a set with \emph{cardinality of continuum}.
\end{definition}

\section{Topology on $\mathbb{R}^n$}

\subsection{Neighborhood, Open Sets and Closed Sets}

\begin{definition}{Neighborhood}{}
  Let $x \in \mathbb{R}^n$, and $\epsilon > 0$, then 
  \begin{equation}
    V(x, \epsilon) := \{y \in \mathbb{R}^n: d(x,y) < \epsilon\}
  \end{equation}
  is said to be the \emph{$\epsilon$-neighborhood of $x$}.
\end{definition}

\begin{definition}{Open Set}{}
  Let $E \subset \mathbb{R}^n$, if $E$ is the neighborhood of every point in
  $E$,
  then it is said to be an \emph{open set}.
\end{definition}

\begin{theorem}{Structure of Open Sets in $\mathbb{R}$}{}
  Let $G$ be an open set of $\mathbb{R}$,
  then $G$ is the union of at most countably many pairwise-disjoint open intervals.
\end{theorem}

\begin{definition}{Closed Set}{}
  Let $E \subset \mathbb{R}^n$, if $E^c$ is an open set,
  then $E$ is said to be a \emph{closed set}.
\end{definition}

\subsection{Interior and Closure}

\begin{definition}{Interior}{}
  Let $S \subset \mathbb{R}^n$.
  A point $x \in S$ is called an \emph{interior point of $S$} if there exists a
  neighborhood $N$ of $x$ such that
  \begin{equation}
    N \subseteq S.
  \end{equation}
  The set composed of all the interior points of $S$ is called the \emph{interior of $S$}.
\end{definition}

\begin{definition}{Adherent Point and Closure}{}
  A point $x \in \mathbb{R}^n$ is called an \emph{adherent point of $E \subset \mathbb{R}^n$} if
  every neighborhood of $x$ intersects $E$, that is,
  \begin{equation}
    U \cap E \neq \emptyset.
  \end{equation}
  The set of all adherent points of $E$ is called the \emph{closure of $E$}, denoted as $\overline{E}$.
\end{definition}



\subsection{Accumulation Points, Isolated Points and Perfect Sets}

\begin{definition}{Accumulation Point}{}
  Let $E \subset \mathbb{R}^n$ and $x \in \mathbb{R}^n$.
  If any neighborhood $U$ of $x$ satisfies
  \begin{equation}
    (U - \{x\}) \cap E \neq \emptyset,
  \end{equation}
  then $x$ is called an \emph{accumulation point of $E$},
  and the set of all accumulation points of $E$ is called the \emph{derived set
    of $E$}, denoted as $E^{\prime}$.
\end{definition}

\begin{example}{Practice for Derived Set}{}
  Let $A$ be a closed set in $\mathbb{R}^n$,
  determine the validity of the following statements:
  \begin{equation}
    (1) A = A^{\prime}, \quad
    (2) A^{\prime} \subset A.
  \end{equation}
\end{example}

\begin{solution}
  (1) Suppose $A$ consists of a single point,
  then $A^{\prime} = \emptyset$, thus the conclusion does not hold.

  (2) Correct. Since a closed set must contain all its accumulation points.
\end{solution}

\begin{definition}{Isolated Point}{}
  Let $E \subset \mathbb{R}^n$ and $x \in \mathbb{R}^n$.
  If $x \in E$ but $x$ is not an accumulation point of $E$,
  then it is called an \emph{isolated point of $E$}.
\end{definition}

\begin{definition}{Perfect Set}{}
  If a set $E \subset \mathbb{R}^n$ is closed and does have any isolated point,
  then it is called a \emph{perfect set}.
\end{definition}

\subsection{Cantor Set}

\begin{definition}{Cantor Set}{}
  Let $C_0 = [0, 1]$, and for each $n \geq 1$, define $C_n$ by removing the open
  middle third from each closed interval in $C_{n-1}$.
  Then the \emph{cantor set} is
  \begin{equation}
    \mathcal{C} := \cap _{n = 0}^{\infty}C_n.
  \end{equation}
\end{definition}

\begin{proposition}{Properties of the Cantor Set}{}
  The Cantor set satisfies
  (1) It is a perfect set;
  (2) It has no interior points;
  (3) Its measure is zero.
\end{proposition}


